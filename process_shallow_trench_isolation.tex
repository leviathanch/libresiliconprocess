\subsection{Shallow trench isolation}\label{sti}
Shallow trench isolation (STI), also known as box isolation technique, is an integrated circuit feature which prevents electric current leakage between adjacent semiconductor device components.
STI is generally used on CMOS process technology nodes of 250 nanometers and smaller.
Older CMOS technologies and non-MOS technologies commonly use isolation based on LOCOS.
\footnote{Quirk, Michael \& Julian Serda (2001). Semiconductor Manufacturing Technology: Instructor's Manual Archived September 28, 2007, at the Wayback Machine., p. 25.} \\
The geometry of a substrate with STI implemented can be seen in \autoref{sti_target}.

\begin{figure}[H]
	\centering
	\begin{tikzpicture}[node distance = 3cm, auto, thick,scale=\CrossAndTopSectionBig, every node/.style={transform shape}]
		% substrate
\fill[substrate] (0,0) rectangle (20,2);
\node at (2,0.5) {Silicon substrate};
%trenches
\fill[isolationoxide] (0,0.75) rectangle (1,2);
\fill[isolationoxide] (8.5,0.75) rectangle (11.5,2);
\fill[isolationoxide] (19,0.75) rectangle (20,2);
	\end{tikzpicture}
	\begin{tikzpicture}[node distance = 3cm, auto, thick,scale=\CrossAndTopSectionBig, every node/.style={transform shape}]
		% substrate
\fill[YellowOrange] (0,0) rectangle (20,12);
% trench area
\fill[DarkGray] (0,0) rectangle (1,12);
\fill[DarkGray] (8.5,0) rectangle (11.5,12);
\fill[DarkGray] (19,0) rectangle (20,12);
\fill[DarkGray] (0,0) rectangle (20,1.25);
\fill[DarkGray] (0,7.5) rectangle (20,12);
	\end{tikzpicture}
	\caption{Shallow trench isolation target geometry}
	\label{sti_target}
\end{figure}

We choose the STI approach because we wanna scale the technology node down in the future below 250nm and wanna ensure backwards compatibility of our process.

As can be seen in \autoref{cmos_nutshell}, the n-well and the STI trench are supposed to have approximately the same depth.
Because the n-well will be $\approx 4 \mu m$ in depth (\autoref{well}) we have to match this with our trench depth.

\begin{figure}[H]
	\centering
	\begin{tikzpicture}[node distance = 3cm, auto, thick,scale=\VLSILayout, every node/.style={transform shape}]
		\fill[Goldenrod,opacity=0.2] (0.75,0.5) rectangle (8.75,7.75);
\fill[Goldenrod,opacity=0.2] (11.25,0.5) rectangle (19.25,7.75);

\draw[dotted] (20.5,0.5) rectangle (25,5.5);

\node at (22.25,5) {\textbf{Layers}};

\fill[Goldenrod,opacity=0.2] (21,1) rectangle (21.5,1.5);
\node at (22.25,1.25) {active};

\fill[orange,opacity=0.2] (21,1.5) rectangle (21.5,2);
\node at (22.25,1.75) {nwell};

\fill[blue,opacity=0.2] (21,2) rectangle (21.5,2.5);
\node at (22.25,2.25) {nimplant};

\fill[red,opacity=0.2] (21,2.5) rectangle (21.5,3);
\node at (22.25,2.75) {pimplant};

\fill[Emerald,opacity=0.2] (21,3) rectangle (21.5,3.5);
\node at (22.25,3.25) {gate};

\fill[Fuchsia,opacity=0.2] (21,3.5) rectangle (21.5,4);
\node at (22.25,3.75) {metal1};

\fill[DarkOrchid,opacity=0.2] (21,4) rectangle (21.5,4.5);
\node at (22.25,4.25) {via1};
	\end{tikzpicture}
	\caption{Shallow trench isolation layout}
	\label{sti_layout}
\end{figure}

\subsubsection{Initial cleaning}
In order to remove the initial naturally grown silicon dioxide from the wafer, acid is being applied to the wafer which leads to a pure silicon substrate wafer as in the process illustration shown below.

\begin{figure}[H]
	\centering
	\begin{tikzpicture}[node distance = 3cm, auto, thick,scale=\CrossSectionOnly, every node/.style={transform shape}]
		% substrate
\fill[substrate] (0,0) rectangle (20,2);
\node at (2,0.5) {Si (p-type)};
% oxide
\fill[isolationoxide] (0,2) rectangle (20,2.3);
\node at (2,2.1) {SiO};
	\end{tikzpicture} \\
	\includegraphics[scale=0.01]{down_arrow.png} \\
	\begin{tikzpicture}[node distance = 3cm, auto, thick,scale=\CrossSectionOnly, every node/.style={transform shape}]
		\input{tikz_process_steps/sti.1.b.tex}
	\end{tikzpicture}
	\caption{Initial cleaning}
\end{figure}

This needs to be done because the naturally grown initially existing silicon oxide is not pure and may contain contamination which may render the final product unusable.

\subsubsection{Sulfuric Cleaning}
The sulfuric acid mixture, $H_2 S O_4 + H_2 O_2$ is being applied to the wafer for 10 minutes at a temperature of 120 \degree C.

\subsubsection{HF dip}
After the sulfuric cleaning a HF (HF:$H_2O$,1:50) dip is being performed for one minute. \\
Hydrofluoric acid (HF) is used to remove native silicon dioxide from wafers. Since it acts quickly, one needs to only expose the wafer for a short time ("dip"). \\
After that the wafer needs to be dried and quickly processed further before new uncontrolled natural oxide can build up on the wafer through the contact with air.

\subsubsection{Pad oxide}
\subsubsection{Nitride layer}
\subsubsection{Patterning}
\subsubsection{Etching}
Dry etching (RIE)
\subsubsection{Deep oxidation}
