\subsection{NMOS threshold}\label{nmos_gate_dimensioning}
First we take a look at the worst case of 4 stacked NMOS transistors, which is the highest stacking amount which will be possible in technologies relying on this process.

\input{process_design_graphics/schematic_AND4.tex}
As shown in  \autoref{TTL_logic_levels} our acceptable voltages for our CMOS "ON" state range from 2V to VDD (which typically is around $5V\pm0.25V$)

\begin{equation}
V_{on} \geq 2V
\end{equation}

Because there are four transistors dividing the voltage by being in series, the power supply voltage is being divided by the amount of transistors in series.
In order to match the threshold voltages of all of the transistors, which is needed for a working digital logic, the following equation need to be satisfied

\begin{equation}
V_{on} > 4 \cdot V_{Tn}
\end{equation}

Lets assume the worst case with

\begin{equation}
V_{on} = 2V
\end{equation}

Which leads to the required worst case threshold tolerance value

\begin{equation}
4 \cdot V_{Tn} < 2V
\Rightarrow
V_{Tn} < 500mV
\end{equation}
 
Using the equation from \autoref{transistor_threshold_calculation} we can now set in this value and play around with the parameters.
  
With the variables and constants being the following we now can put the formula together:
\begin{itemize}
\item $N_A \approx 9\times10^{14}cm^{-3} = 9\times10^{20}m^{-3}$ is the substrate doping (of our chosen basis substrate. also see \autoref{process_overview})
\item $N_i$ is the carrier concentration in intrinsic (undoped) silicon. $N_i$ is equal to $1.45 \times 10^{10} cm^{-3} = 1.45 \times 10^{16} m^{-3}$ at 300\degree K
\item $\phi_M = 4.1 eV$ is the "work function" of our metal at the gate (Aluminum)
\item $E_g(T) = E_g(0) - \frac{\alpha T^2}{T+\beta} = 1.166 - 4.73 \cdot 10^{-4} \cdot \frac{T^2}{T+636} [eV]$ is the band gap energy of silicon at a given temperature\footnote{https://ecee.colorado.edu/~bart/book/eband5.htm} for which the parameters can be taken from \autoref{band_gap_parameters}
\begin{table}[H]
\centering
\begin{tabular}{|c|c|c|c|}
\hline
{} &
\textbf{Germanium} &
\textbf{Silicon} &
\textbf{GaAs} \\
\hline
$Eg(0) [eV]$ &
0.7437 &
1.166 &
1.519 \\
\hline
$\alpha [eV/K]$ &
4.77 x 10-4 &
4.73 x 10-4 &
5.41 x 10-4 \\
\hline
$\beta [K]$ &
235 &
636 &
204 \\
\hline
\end{tabular}
\caption{Band cap energy parameters}
\label{band_gap_parameters}
\end{table}
\item $C_{ox} \left[\frac{F}{m^2}\right]$ is the capacity of the gate oxide
\item $\epsilon_0 = 8.85 \cdot 10^{-14} \frac{F}{cm}.= 8.85 \cdot 10^{-12} \frac{F}{m} $ is the electric permittivity in vacuum
\item $\epsilon_{Si} =11.68 \cdot \epsilon_0$ is the relative permittivity of silicon
\item $\epsilon_{ox} = 3.9 \cdot \epsilon_0$ is the relative permittivity of silicon oxide
\item $t_{ox} [cm]$ is the thickness of the oxide layer in cm
\item $\chi = 4.05 eV$ is the electron affinity of a silicon crystal surface\footnote{\url{https://en.wikipedia.org/wiki/Electron_affinity}}
\item $q=1.602 \cdot 10^{-19} C$ is the elementary charge
\end{itemize} 
 
 \begin{equation}
V_T= \frac{1.66 \cdot 10^{-7}}{C_{ox}}-0.094
\end{equation}

For safety reasons we take 10\% of the worst case as a error margin which leads us to the following new equation to solve
 \begin{equation}
V_T = 450mV
\end{equation}

This gives us the constraint of
\begin{equation}
0.45 V = \left( \frac{1.66 \cdot 10^{-7}}{C_{ox}}-0.094 \right) V
\end{equation}

We solve for the target capacity (in $\frac{F}{cm^2}$) which gives us the desired threshold voltage:
\begin{equation}
\Rightarrow
C_{ox}
=
3.057 \cdot {10}^{-7} \frac{F}{cm^2}
\end{equation}
\begin{equation}
\Rightarrow
C_{ox}
=
305.7 \frac{nF}{cm^2}
\end{equation}