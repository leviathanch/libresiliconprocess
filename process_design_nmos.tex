\subsection{NMOS threshold}\label{nmos_dimensioning}
First we take a look at the worst case of 4 stacked NMOS transistors, which is the highest stacking amount which will be possible in technologies relying on this process.

\input{process_design_graphics/schematic_AND4.tex}
As shown in  \autoref{TTL_logic_levels} our acceptable voltages for our CMOS "ON" state range from 2V to VDD (which typically is around $5V\pm0.25V$)

\begin{equation}
V_{on} \geq 2V
\end{equation}

Because there are four transistors dividing the voltage by being in series, the power supply voltage is being divided by the amount of transistors in series.
In order to match the threshold voltages of all of the transistors, which is needed for a working digital logic, the following equation need to be satisfied

\begin{equation}
V_{on} > 4 \cdot V_{Tn}
\end{equation}

Lets assume the worst case with

\begin{equation}
V_{on} = 2V
\end{equation}

Which leads to the required worst case threshold tolerance value

\begin{equation}
4 \cdot V_{Tn} < 2V
\Rightarrow
V_{Tn} < 500mV
\end{equation}
 
With the values derived from \autoref{gate_dimensioning} and a surface concentration for our P-well of $10^{22}\frac{1}{m^3}$ we are already set because $\approx 0.28V$ are already better than we need.

So we target a surface concentration of $10^{22}\frac{1}{m^3}$ which gives us margin of error in case of temperature variations and too long diffusion.
  
