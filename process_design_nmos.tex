\subsection{NMOS threshold}\label{nmos_gate_dimensioning}
First we take a look at the worst case of 4 stacked NMOS transistors, which is the highest stacking amount which will be possible in technologies relying on this process.

\input{process_design_graphics/schematic_AND4.tex}
As shown in  \autoref{TTL_logic_levels} our acceptable voltages for our CMOS "ON" state range from 2V to VDD (which typically is around $5V\pm0.25V$)

\begin{equation}
V_{on} \geq 2V
\end{equation}

Because there are four transistors dividing the voltage by being in series, the power supply voltage is being divided by the amount of transistors in series.
In order to match the threshold voltages of all of the transistors, which is needed for a working digital logic, the following equation need to be satisfied

\begin{equation}
V_{on} > 4 \cdot V_{Tn}
\end{equation}

Lets assume the worst case with

\begin{equation}
V_{on} = 2V
\end{equation}

Which leads to the required worst case threshold tolerance value

\begin{equation}
4 \cdot V_{Tn} < 2V
\Rightarrow
V_{Tn} < 500mV
\end{equation}
 
 Using the equation from \autoref{transistor_threshold_calculation} we can now set in this value and play around with the parameters.
 
 \begin{equation}
V_T= \frac{1.66 \cdot 10^{-7}}{C_{ox}}-0.094
\end{equation}

For safety reasons we take 10\% of the worst case as a error margin which leads us to the following new equation to solve
 \begin{equation}
V_T = 450mV
\end{equation}

This gives us the constraint of
\begin{equation}
0.45 V = \left( \frac{1.66 \cdot 10^{-7}}{C_{ox}}-0.094 \right) V
\end{equation}

We solve for the target capacity (in $\frac{F}{cm^2}$) which gives us the desired threshold voltage:
\begin{equation}
\Rightarrow
C_{ox}
=
3.057 \cdot {10}^{-7} \frac{F}{cm^2}
\end{equation}
\begin{equation}
\Rightarrow
C_{ox}
=
305.7 \frac{nF}{cm^2}
\end{equation}