\subsection{Diffusion (Doping)}

Although the diffusion process of donors and acceptors into the silicon crystal is a three dimensional process for simplicity we first only discuss the one dimensional mathematics for it in order to get a "simple" equation for the depth-time-temperature relation.

We start with Ficks law (for all German speakers: Yes that's his name, not me shouting about this problem) where the dopant concentration N is coupled with time and place
\begin{equation}
\frac{\partial N}{\partial t} = D \cdot \frac{\partial^2 N}{\partial x^2}
\end{equation}

The diffusion coefficient is as well material as well as temperature dependent  and can be calculated with the following equation:
\begin{equation}
D = D_0 \cdot \exp\left(-\frac{E_a}{k \cdot T}\right)
\end{equation}
With $k=8.62 \cdot 10^{-5} \frac{eV}{K}$ being the Boltzman constant and in \autoref{absolute_diffusion_coefficients} we can see the $D_0$ and $E_a$ values for the most common materials\footnote{ISBN 3-8023-1588:Hoppe Bernhard, Mikroelektronik 2, Page 24, Table 2.1} which we can use within the further calculations for our well dimensioning phases. The temperature usually is in the area of $1000\degree C$ or in Kelvin $1273.15\degree K$.
\begin{table}[H]
	\centering
	\begin{tabular}{|c|c|c|}
		\hline
		Element &
		$D_0$ $\left[\frac{cm^2}{s}\right]$ &
		$E_a$ $\left[eV\right]$ \\
		\hline
		P &
		10.50 &
		3.69 \\
		\hline
		As &
		0.32 &
		3.56 \\
		\hline
		Sb &
		5.60 &
		3.95 \\
		\hline
		B &
		10.50 &
		3.69 \\
		\hline
		Al &
		8.00 &
		3.47 \\
		\hline
		Ga &
		3.60 &
		3.51 \\
		\hline
		Cu &
		0.0025 &
		0.65 \\
		\hline
	\end{tabular}
	\label{absolute_diffusion_coefficients}
	\caption{$D_0$ and $E_a$ values for Boron and Phosphorus}
\end{table}

The law stated above
\begin{equation}
\frac{\partial N}{\partial t} = D \cdot \frac{\partial^2 N}{\partial x^2} 
\end{equation}
has the same form as the temperature conductivity equation (Laplace) for which we already have a general solution
\begin{equation}
\frac{\partial u}{\partial t} = a^2 \cdot \frac{\partial^2 u}{\partial x^2} 
\end{equation}

Which means that we can map the general solution for the temperature conductivity equations after Laplace
\begin{equation}
u(x,t) = \frac{1}{2 \cdot a \cdot \sqrt{\pi \cdot t}} \cdot \int_{-\infty}^{\infty}{f(a)\cdot\exp\left(\frac{-(x-a)^2}{4 \cdot a^2 \cdot t^2}\right)}da
\end{equation}
to our Ficks law with $a=\sqrt{D}$ und $u=N$
\begin{equation}
N(x,t) = \frac{1}{2 \cdot \sqrt{D} \cdot \sqrt{\pi \cdot t}} \cdot \int_{-\infty}^{\infty}{f(\sqrt{D})\cdot\exp\left(\frac{-(x-\sqrt{D})^2}{4 \cdot D \cdot t^2}\right)}da
\end{equation}
