\subsection{PMOS threshold}\label{pmos_dimensioning}
Now we take a look at the worst case of 4 stacked PMOS transistors, which is the highest stacking amount which will be possible in technologies relying on this process.
\input{process_design_graphics/schematic_OR4.tex}

$\approx 4 \mu m$ come mainly from the need to fulfill the condition from \autoref{physics_drive_in}

\begin{equation}
x_e = 2 \cdot \sqrt{D_e \cdot t_e} \gg 2 \cdot \sqrt{D_v \cdot t_v} = x_v
\end{equation}

We already got the background ($N_B \approx 7 \cdot 10^{14} \frac{1}{cm^3}=7 \cdot 10^{20} \frac{1}{m^3}$) concentration from the specs of the basis substrate.

We use a drive in temperature of $1150 \degree C$ which is  $T = 1423.15 \degree K$ in Kelvin which gives us the diffusion coefficient $D=9.1 \cdot 10^{-17}  \frac{m^2}{s}$

Now using
\begin{equation}
N(x,t)
=
\frac{Q}{\sqrt{\pi\cdot D \cdot t}} \cdot \exp\left(\frac{-x^2}{4 \cdot D \cdot t}\right)
\end{equation}

We set the conditions and get the required diffusion time as well as the initial dosage in one shot:
\begin{equation}
N(0,t)
=
\frac{Q}{\sqrt{\pi\cdot D \cdot t}}
=
N_p-N_B
=
7 \cdot 10^{20} \frac{1}{m^3}
\end{equation}
\begin{equation}
x
=
2 \cdot \sqrt{D \cdot t}
=
4 \mu m
=
4 \cdot 10^{-6} m
\end{equation}
\begin{equation}
\Rightarrow
Q
=
7 \cdot 10^{20} \frac{1}{m^3} \cdot \sqrt{\pi\cdot D \cdot t}
=
7 \cdot 10^{20} \frac{1}{m^3} \cdot \sqrt{\pi} \cdot 2 \cdot 10^{-6} m
\approx
\underline{2.48 \cdot 10^{15} \frac{1}{m^2}}
\end{equation}

TODO