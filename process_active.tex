\subsection{Active}\label{active}
\begin{figure}[H]
	\centering
	\begin{tikzpicture}[node distance = 3cm, auto, thick,scale=\CrossAndTopSectionBig, every node/.style={transform shape}]
		\input{tikz_process_steps/well.a.tex}
% gate oxide
\fill[LightGray] (4.8,2) rectangle (6.7,2.3);
\fill[LightGray] (13.3,2) rectangle (15.2,2.3);
% gate poly
\fill[BrickRed] (4.8,2.3) rectangle (6.7,2.6);
\fill[BrickRed] (13.3,2.3) rectangle (15.2,2.6);
	\end{tikzpicture}
	\begin{tikzpicture}[node distance = 3cm, auto, thick,scale=\CrossAndTopSectionBig, every node/.style={transform shape}]
		\input{tikz_process_steps/active.b.tex}
	\end{tikzpicture}
	\caption{Active area geometry target}
\end{figure}

\subsubsection{Gate oxide growth}
The thickness of the gate oxide depends the capacity (slew rate) of the transistor.
The thinner the layer is, the steeper the edges of the CMOS circuitry will be, however also the threshold voltage will be reduced the thinner the gate oxide gets.

\begin{figure}[H]
	\centering
	\begin{tikzpicture}[node distance = 3cm, auto, thick,scale=\CrossSectionOnly, every node/.style={transform shape}]
		\input{tikz_process_steps/well.a.tex}
	\end{tikzpicture} \\
	\includegraphics[scale=0.01]{down_arrow.png} \\
	\begin{tikzpicture}[node distance = 3cm, auto, thick,scale=\CrossSectionOnly, every node/.style={transform shape}]
		\input{tikz_process_steps/well.a.tex}
\fill[LightGray] (0,2) rectangle (20,2.3);

	\end{tikzpicture}
	\caption{Gate oxide layer growth}
\end{figure}

\subsubsection{Polysilicon growth}
\begin{figure}[H]
	\centering
	\begin{tikzpicture}[node distance = 3cm, auto, thick,scale=\CrossSectionOnly, every node/.style={transform shape}]
		\input{tikz_process_steps/active.2.a.tex}
	\end{tikzpicture} \\
	\includegraphics[scale=0.01]{down_arrow.png} \\
	\begin{tikzpicture}[node distance = 3cm, auto, thick,scale=\CrossSectionOnly, every node/.style={transform shape}]
		\input{tikz_process_steps/well.a.tex}
\fill[LightGray] (0,2) rectangle (20,2.3);
\fill[BrickRed] (0,2.3) rectangle (20,2.6);
	\end{tikzpicture}
	\caption{Polysilicon layer growth}
\end{figure}

\subsubsection{Patterning}
\begin{figure}[H]
	\centering
	\begin{tikzpicture}[node distance = 3cm, auto, thick,scale=\CrossAndTopSection, every node/.style={transform shape}]
		\input{tikz_process_steps/well.a.tex}

% gate oxide layer
\fill[LightGray] (0,2) rectangle (20,2.3);
% poly layer
\fill[BrickRed] (0,2.3) rectangle (20,2.6);

	\end{tikzpicture}
	\begin{tikzpicture}[node distance = 3cm, auto, thick,scale=\CrossAndTopSection, every node/.style={transform shape}]
		\fill[BrickRed] (0,0) rectangle (20,10);
	\end{tikzpicture} \\
	\includegraphics[scale=0.01]{down_arrow.png} \\
	\begin{tikzpicture}[node distance = 3cm, auto, thick,scale=\CrossAndTopSection, every node/.style={transform shape}]
		% substrate
\fill[YellowOrange] (0,0) rectangle (20,2);
\node at (2,0.5) {Si (p-type)};
% n-well
\fill[Goldenrod] (1.25,0.75) rectangle (8.25,2);
\node at (4.75,1) {N-Well};
% gate oxide layer
\fill[LightGray] (0,2) rectangle (20,2.1);
% poly layer
\fill[BrickRed] (0,2.1) rectangle (20,2.2);
% resist
\fill[orange] (4.8,2.2) rectangle (6.7,2.8);
\fill[orange] (13.3,2.2) rectangle (15.2,2.8);
%field oxides:
\fill[DarkGray] (0,2) rectangle (1,4);
\fill[DarkGray] (8.5,2) rectangle (11.5,4);
\fill[DarkGray] (19,2) rectangle (20,4);
% channel stop
\fill[RedOrange] (0,1.5) rectangle (1,2);
\fill[RedOrange] (8.5,1.5) rectangle (11.5,2);
\fill[RedOrange] (19,1.5) rectangle (20,2);
	\end{tikzpicture}
	\begin{tikzpicture}[node distance = 3cm, auto, thick,scale=\CrossAndTopSection, every node/.style={transform shape}]
		\input{tikz_process_steps/active.3.bt.tex}
	\end{tikzpicture}
	\caption{Polysilicon contact mask}
\end{figure}

\subsubsection{Etching}
\begin{figure}[H]
	\centering
	\begin{tikzpicture}[node distance = 3cm, auto, thick,scale=\CrossAndTopSection, every node/.style={transform shape}]
		\input{tikz_process_steps/well.a.tex}

% gate oxide layer
\fill[LightGray] (0,2) rectangle (20,2.3);
% poly layer
\fill[BrickRed] (0,2.3) rectangle (20,2.6);

% resist
\fill[orange] (4.8,2.6) rectangle (6.7,3.2);
\fill[orange] (13.3,2.6) rectangle (15.2,3.2);
	\end{tikzpicture}
	\begin{tikzpicture}[node distance = 3cm, auto, thick,scale=\CrossAndTopSection, every node/.style={transform shape}]
		\input{tikz_process_steps/active.4.at.tex}
	\end{tikzpicture} \\
	\includegraphics[scale=0.01]{down_arrow.png} \\
	\begin{tikzpicture}[node distance = 3cm, auto, thick,scale=\CrossAndTopSection, every node/.style={transform shape}]
		\input{tikz_process_steps/well.a.tex}

% gate oxide
\fill[LightGray] (4.8,2) rectangle (6.7,2.3);
\fill[LightGray] (13.3,2) rectangle (15.2,2.3);
% gate poly
\fill[BrickRed] (4.8,2.3) rectangle (6.7,2.6);
\fill[BrickRed] (13.3,2.3) rectangle (15.2,2.6);

% resist
\fill[orange] (4.8,2.6) rectangle (6.7,3.2);
\fill[orange] (13.3,2.6) rectangle (15.2,3.2);
	\end{tikzpicture}
	\begin{tikzpicture}[node distance = 3cm, auto, thick,scale=\CrossAndTopSection, every node/.style={transform shape}]
		\input{tikz_process_steps/active.4.bt.tex}
	\end{tikzpicture}
	\caption{Polysilicon contact etched}
\end{figure}
Because the exact shape of the gate contact is required for a reproducible property characterization of the transistor geometry, dry etching is being used for etching the poly-oxide layer stack.

\subsubsection{Cleaning}
\begin{figure}[H]
	\centering
	\begin{tikzpicture}[node distance = 3cm, auto, thick,scale=\CrossAndTopSection, every node/.style={transform shape}]
		\input{tikz_process_steps/active.5.a.tex}
	\end{tikzpicture}
	\begin{tikzpicture}[node distance = 3cm, auto, thick,scale=\CrossAndTopSection, every node/.style={transform shape}]
		\input{tikz_process_steps/active.5.at.tex}
	\end{tikzpicture} \\
	\includegraphics[scale=0.01]{down_arrow.png} \\
	\begin{tikzpicture}[node distance = 3cm, auto, thick,scale=\CrossAndTopSection, every node/.style={transform shape}]
		\input{tikz_process_steps/active.5.b.tex}
	\end{tikzpicture}
	\begin{tikzpicture}[node distance = 3cm, auto, thick,scale=\CrossAndTopSection, every node/.style={transform shape}]
		\input{tikz_process_steps/active.5.bt.tex}
	\end{tikzpicture}
	\caption{Resist removal}
\end{figure}