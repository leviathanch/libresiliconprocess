\subsection{Well}
In order to build CMOS on the same substrate, an n-well is required for building the complementary P-channel transistor for a n-p-channel logic circuitry as shown above in the example section.
The cross section as well as the top view of the targeted geometry are shown in \autoref{nwell_target}
\begin{figure}[H]
	\centering
	\begin{tikzpicture}[node distance = 3cm, auto, thick,scale=\CrossAndTopSectionBig, every node/.style={transform shape}]
		\input{tikz_process_steps/2.a.tex}
	\end{tikzpicture}
	\begin{tikzpicture}[node distance = 3cm, auto, thick,scale=\CrossAndTopSectionBig, every node/.style={transform shape}]
		% substrate
\fill[YellowOrange] (0,0) rectangle (20,12);
% n-well
\fill[Goldenrod] (1.25,1.5) rectangle (8.25,7.25);
	\end{tikzpicture}
	\caption{N-well target geometry}
	\label{nwell_target}
\end{figure}
The n-well will serve us as an island of n-doped substrate within the p-doped basis substrate.
the n-well forms a natural p-n-junction to the later on implanted channel-stop which has a nice side effect of being an additional polarity protection.
The dopant dose will be: $2.5\times10^{12}cm^{-2}$

// TODO: Add expected resistance here

\subsubsection{Dioxide layer}
In order to selectively inject charge carrying atoms into the crystalline structure a protective dioxide ($SiO_2$) layer needs to be grown on top of a p-type substrate.
\begin{figure}[H]
	\centering
	\begin{tikzpicture}[node distance = 3cm, auto, thick,scale=\CrossSectionOnly, every node/.style={transform shape}]
		\input{tikz_process_steps/2.1.a.tex}
	\end{tikzpicture} \\
	\includegraphics[scale=0.01]{down_arrow.png} \\
	\begin{tikzpicture}[node distance = 3cm, auto, thick,scale=\CrossSectionOnly, every node/.style={transform shape}]
		\input{tikz_process_steps/2.1.b.tex}
	\end{tikzpicture}
	\caption{Dioxide layer growth}
\end{figure}
The industrial best practice is a layer of around (500nm$\approx$5000\normalfont\AA) thickness or more.
For this purpose the wafer is being oxidized for at least 90 minutes at 1000\degree C using wet oxidation which results in a dioxide layer at least 500nm($\approx$5000\normalfont\AA) in thickness.

\newpage
\subsubsection{Patterning}
The resist is being deposited using spin coating and then baked depending on the baking time for the specific resist.
The layout for being exposed onto the resist is being extracted from the "nwell" layer within the GDS2 file.
\begin{figure}[H]
	\centering
	\begin{tikzpicture}[node distance = 3cm, auto, thick,scale=\CrossAndTopSection, every node/.style={transform shape}]
		\input{tikz_process_steps/2.2.a.tex}
	\end{tikzpicture}
	\begin{tikzpicture}[node distance = 3cm, auto, thick,scale=\CrossAndTopSection, every node/.style={transform shape}]
		\input{tikz_process_steps/2.2.at.tex}
	\end{tikzpicture} \\
	\includegraphics[scale=0.01]{down_arrow.png} \\
	\begin{tikzpicture}[node distance = 3cm, auto, thick,scale=\CrossAndTopSection, every node/.style={transform shape}]
		\input{tikz_process_steps/2.2.b.tex}
	\end{tikzpicture}
	\begin{tikzpicture}[node distance = 3cm, auto, thick,scale=\CrossAndTopSection, every node/.style={transform shape}]
		% resist
\fill[orange] (0,0) rectangle (20,12);
% substrate
\fill[gray] (1.25,1.5) rectangle (8.25,7.25);
	\end{tikzpicture}
	\caption{Cross/top view of n-well layout on resist}
\end{figure}
The thickness of the resist layer and the backing duration will variate depending on the specific equipment for which this process will be implemented with.

\subsubsection{Etching}
We now need to open a window in the dioxide layer, through which we will inject carrier atoms into the silicon crystal structure.
\begin{figure}[H]
	\centering
	\begin{tikzpicture}[node distance = 3cm, auto, thick,scale=\CrossAndTopSection, every node/.style={transform shape}]
		\input{tikz_process_steps/2.3.a.tex}
	\end{tikzpicture}
	\begin{tikzpicture}[node distance = 3cm, auto, thick,scale=\CrossAndTopSection, every node/.style={transform shape}]
		\input{tikz_process_steps/2.3.at.tex}
	\end{tikzpicture} \\
	\includegraphics[scale=0.01]{down_arrow.png} \\
	\begin{tikzpicture}[node distance = 3cm, auto, thick,scale=\CrossAndTopSection, every node/.style={transform shape}]
		\input{tikz_process_steps/2.3.b.tex}
	\end{tikzpicture}
	\begin{tikzpicture}[node distance = 3cm, auto, thick,scale=\CrossAndTopSection, every node/.style={transform shape}]
		\input{tikz_process_steps/2.3.bt.tex}
	\end{tikzpicture}
	\caption{Cross/top view of n-well oxide window}
\end{figure}
Since the silicon dioxide layer is 500nm thick and we wanna reach the silicon below we can use wet etching as described in the chemistry chapter.
Using BHF (\autoref{BHF}) we can etch with a speed of 100 up to 250 nm/min at room  temperature because there is no danger of etching "too far", we can define 5 minutes as a safe etching time.

\subsubsection{Cleaning}
\begin{figure}[H]
	\centering
	\begin{tikzpicture}[node distance = 3cm, auto, thick,scale=\CrossSectionOnly, every node/.style={transform shape}]
		\input{tikz_process_steps/2.4.a.tex}
	\end{tikzpicture} \\
	\includegraphics[scale=0.01]{down_arrow.png} \\
	\begin{tikzpicture}[node distance = 3cm, auto, thick,scale=\CrossSectionOnly, every node/.style={transform shape}]
		% substrate
\fill[YellowOrange] (0,0) rectangle (20,2);
\node at (2,0.5) {Si (p-type)};
% oxide
\fill[gray] (0,2) rectangle (1.25,2.6);
\fill[gray] (8.25,2) rectangle (20,2.6);
	\end{tikzpicture}
	\caption{Resist removal}
\end{figure}

\subsubsection{Injection}
We now need to inject the carriers into the upper level of the n-channel area so that we can later on drive them into the crystal during the drive-in step.
\begin{figure}[H]
	\centering
	\begin{tikzpicture}[node distance = 3cm, auto, thick,scale=\CrossSectionOnly, every node/.style={transform shape}]
		% substrate
\fill[YellowOrange] (0,0) rectangle (20,2);
\node at (2,0.5) {Si (p-type)};
% oxide
\fill[gray] (0,2) rectangle (1.25,2.6);
\fill[gray] (8.25,2) rectangle (20,2.6);

\newcounter{ct}
\forloop{ct}{0}{\value{ct} < 21}
{
	\draw [->] (\value{ct},4) -- (\value{ct},3);
	\node at (\value{ct},4.2) {P$^{31}$};
}
	\end{tikzpicture} \\
	\includegraphics[scale=0.01]{down_arrow.png} \\
	\begin{tikzpicture}[node distance = 3cm, auto, thick,scale=\CrossSectionOnly, every node/.style={transform shape}]
		\input{tikz_process_steps/2.5.b.tex}
	\end{tikzpicture}
	\caption{Doping process}
\end{figure}
The n-well is implanted with a Phosphorus ($P^{31}$) dose of $2.5\times10^{12}cm^{-2}$ at an energy of 100 KeV.
The n-well is then annealed.

\subsubsection{Oxide for drive-in}
Now we need to cover the now doped and annealed areas with an oxide layer for the drive-in phase.
\begin{figure}[H]
	\centering
	\begin{tikzpicture}[node distance = 3cm, auto, thick,scale=\CrossSectionOnly, every node/.style={transform shape}]
		\input{tikz_process_steps/2.6.a.tex}
	\end{tikzpicture} \\
	\includegraphics[scale=0.01]{down_arrow.png} \\
	\begin{tikzpicture}[node distance = 3cm, auto, thick,scale=\CrossSectionOnly, every node/.style={transform shape}]
		\input{tikz_process_steps/2.6.b.tex}
	\end{tikzpicture}
	\caption{Oxide growth}
\end{figure}
The wafer is being oxidized for 32 minutes at 1000\degree C in order to achieve a cover silicon layer of 250nm thickness ($\approx$2500\normalfont\AA).

\subsubsection{Drive-in}
In order to drive the carrier atoms deeper into the crystalline structure the wafer needs to be driven in after predeposition.
\begin{figure}[H]
	\centering
	\begin{tikzpicture}[node distance = 3cm, auto, thick,scale=\CrossSectionOnly, every node/.style={transform shape}]
		\input{tikz_process_steps/2.7.a.tex}
	\end{tikzpicture} \\
	\includegraphics[scale=0.01]{down_arrow.png} \\
	\begin{tikzpicture}[node distance = 3cm, auto, thick,scale=\CrossSectionOnly, every node/.style={transform shape}]
		\input{tikz_process_steps/2.7.b.tex}
	\end{tikzpicture}
	\caption{Drive-in process}
\end{figure}
In this step the wafer is  driven-in for 960 minutes at 1150\degree C in an inert ambient.

\subsubsection{Oxide removal}
We want an oxide free wafer with the n-well accessible for the further process steps.
\begin{figure}[H]
	\centering
	\begin{tikzpicture}[node distance = 3cm, auto, thick,scale=\CrossSectionOnly, every node/.style={transform shape}]
		\input{tikz_process_steps/2.8.a.tex}
	\end{tikzpicture} \\
	\includegraphics[scale=0.01]{down_arrow.png} \\
	\begin{tikzpicture}[node distance = 3cm, auto, thick,scale=\CrossSectionOnly, every node/.style={transform shape}]
		\input{tikz_process_steps/2.8.b.tex}
	\end{tikzpicture}
	\caption{Oxide removal}
\end{figure}
We use hydrofluoric acid, because it doesn't etch silicon at all but is very aggressive towards $SiO_2$ while with non oxidized silicon there is no reaction.
