\section{Substrate}

The Hong University of science and technology (short HKUST) provides us with two types of wafers.

\begin{itemize}
	\item Prime Grade Silicon Wafer, [100] N-type
	\begin{itemize}
		\item Front-side polished, backside etched
		\item Dopant: Phosphorus
		\item Thickness: $525 \mu m \pm 25 \mu m$
		\item Resistivity:4 to 7ohm-cm
		\item Growth Method: CZ
		\item Diameter: 100mm +/- 0.5 mm
		\item Primary  \& secondary flat locations: (In compliance with the SEMI)
			\begin{itemize}
				\item Carbon concentration<2.5 x e16 atm/cc
				\item Oxygen concentration < $9.0 \cdot 10^{17} \frac{atm}{cc}$
				\item $TTV < 10 \mu m$
				\item $TIR < 6 \mu m$
				\item $Bow/Warp < 40 \mu m$
			\end{itemize}
	\end{itemize}
	\item Prime Grade Silicon Wafer, [100] P-type
	\begin{itemize}
		\item Front-side polished, backside etched
		\item Dopant: Boron
		\item Thickness: $525 \mu m \pm 25 \mu m$
		\item Resistivity:15 to 25 ohm-cm
		\item Growth Method: CZ
		\item Diameter: 100mm +/- 0.5 mm
		\item Primary  \& secondary flat locations: (In compliance with the SEMI)
			\begin{itemize}
				\item Carbon concentration<2.5 x e16 atm/cc
				\item Oxygen concentration < $9.0 \cdot 10^{17} \frac{atm}{cc}$
				\item $TTV < 10 \mu m$
				\item $TIR < 6 \mu m$
				\item $Bow/Warp < 40 \mu m$
			\end{itemize}
	\end{itemize}
\end{itemize}

For this process the p-doped mono crystalline silicon substrate is being used, but forks and modifications will be very well possible based on a Graphene substrate or alike, still under the LSPL.
The starting material is a p-doped <100> oriented mono crystalline silicon wafer\\

\textbf{Reasons for using p-doped substrate}:\begin{itemize}
\item We can't use two different substrates for our design because in the design both PMOS and NMOS is present.
We have to choose which is more beneficial from fabrication point of view.
In general or say it's true that NMOS devices are always more in the Semiconductor Industry in comparison to PMOS devices.
For your reference-SRAM requires 6 transistors (4 NMOS, 2 PMOS).
\item Another reason for more number of NMOS is because of difference of mobility of electron and holes.
Electron mobility is almost twice of holes mobility and because of this ON-RESISTANCE of n-channel device is half of p-channel device with the same geometry and under the same operating conditions.
That means to achieve same impedance size of n-channel transistors is almost half of p-channel devices.
Same thing I can say in the different way that for same size of wafer, we can have more number of NMOS (means can perform more logical operation) in comparison to PMOS.
\item Since we only have the choice between P and N doped substrate, we use P doped substrate, because of the carrier mobility
\end{itemize}

Using the method from \autoref{r_l_nd_relation} we get a doping concentration between $8.76 \cdot 10^{14} \frac{1}{cm^3}$ and $5.23 \cdot 10^{14} \frac{1}{cm^3}$.
The average of this range is $N_B = \frac{8.76+5.23 }{2} \cdot 10^{14} \frac{1}{cm^3} \approx 7 \cdot 10^{14} \frac{1}{cm^3}$