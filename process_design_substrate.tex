\subsection{Substrate}
For this process p-substrate is being used, but forks and modifications will be very well possible based on a Graphene substrate or alike, still under the LSPL.
The starting material is a p-type, <100> oriented silicon with a doping concentration of $\approx 9\times10^{14}cm^{-3}$.\\

\textbf{Reasons for using p-substrate}:\begin{itemize}
\item We can't use two different substrates for our design because in the design both PMOS and NMOS is present.
We have to choose which is more beneficial from fabrication point of view.
In general or say it's true that NMOS devices are always more in the Semiconductor Industry in comparison to PMOS devices.
For your reference-SRAM requires 6 transistors (4 NMOS, 2 PMOS).
\item Another reason for more number of NMOS is because of difference of mobility of electron and holes.
Electron mobility is almost twice of holes mobility and because of this ON-RESISTANCE of n-channel device is half of p-channel device with the same geometry and under the same operating conditions.
That means to achieve same impedance size of n-channel transistors is almost half of p-channel devices.
Same thing I can say in the different way that for same size of wafer, we can have more number of NMOS (means can perform more logical operation) in comparison to PMOS.
\end{itemize}