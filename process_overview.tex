Below the general flow chart of the overall process flow can be seen. These process steps will be discussed within the following sections.
\begin{center}
	\begin{tikzpicture}[node distance = 3cm, auto, thick,scale=0.5, every node/.style={transform shape}]
		% Place nodes
		\node [block] (well) {Well};
		\node [block, below of=well] (implant) {Implant};
		\node [block, below of=implant] (select) {Select};
		\node [block, below of=select] (active) {Active};
		\node [block, below of=active] (metal) {Metal 1-6};
		% Draw edges
		\path [line] (well) -- (implant);
		\path [line] (implant) -- (select);
		\path [line] (select) -- (active);
		\path [line] (active) -- (metal);
		
		\draw[dotted] (-2,-10.5) rectangle (2,1.5);
		\node at (3.5,-5) {CMOS process};
	\end{tikzpicture}
\end{center}
The four starting overall process steps are part of an overall active part of the technology, while the final metal (respectively contact) layers will be used for making a contact between the logic gates and macro cells and making them available to the exterior world.
For this process p-substrate is the required basic substrate, but forks and modifications will be very well possible based on a Graphene substrate or alike, still under the LSPL.
The decision to use an n-well approach is based largely on the compatibility with the existing nMOS process.
The starting material is a p-type, <100> oriented silicon with a doping concentration of $\approx 9\times10^{14}cm^{-3}$.