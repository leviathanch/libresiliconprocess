\section{Process}

Below the general flow chart of the overall process flow can be seen in \autoref{full_flow}.
These process steps will be discussed within the following sections.

\begin{center}
	\begin{figure}[h]
		\begin{center}
			\begin{tikzpicture}[node distance = 3cm, auto, thick,scale=0.5, every node/.style={transform shape}]
				% Place nodes
				\node [block] (clean1) at (0,15) {Initial cleaning};
				\node [block, right of=clean1] (well) {Well};
				\node [block, below of=well] (fox) {Channel stop and field oxide};
				\node [block, below of=fox] (active) {Active};
				\node [block, below of=active] (pp) {p+ Implant};
				\node [block, below of=pp] (np) {n+ Implant};
				\node [block, right of=np] (metal) {Metal 1-6};
				% Draw edges
				\path [line] (clean1) -- (well);
				\path [line] (well) -- (fox);
				\path [line] (fox) -- (active);
				\path [line] (active) -- (pp);
				\path [line] (pp) -- (np);
				\path [line] (np) -- (metal);
				
				\draw[dotted] (1.5,2) rectangle (4.5,16);
				\node at (6,8) {CMOS process};
			\end{tikzpicture}
		\end{center}
		\caption{Frontend and backend process flow}
		\label{full_flow}
	\end{figure}
\end{center}

The five starting overall process steps are part of an overall active part of the technology, while the final metal (respectively contact) layers will be used for making a contact between the logic gates and macro cells and making them available to the exterior world.
For this process p-substrate is the required basic substrate, but forks and modifications will be very well possible based on a Graphene substrate or alike, still under the LSPL.
The decision to use an n-well approach is based largely on the compatibility with the existing nMOS process.
The starting material is a p-type, <100> oriented silicon with a doping concentration of $\approx 9\times10^{14}cm^{-3}$.
