\subsection{Gate}\label{gate}
Now we have to build the initial gate structure which contains of the 40nm thick dielectric (in our case just silicon dioxide) and the polysilicon electrode.

\begin{figure}[H]
	\centering
	\begin{tikzpicture}[node distance = 3cm, auto, thick,scale=\CrossAndTopSectionBig, every node/.style={transform shape}]
		\input{tikz_process_steps/pwell.a.tex}
% n-well
\fill[nwell] (1.25,0.75) rectangle (8.5,2);
\node at (5.75,1) {N-Well};
% p-well
\fill[pwell] (11.75,0.25) rectangle (18.75,2);
\node at (15.25,1) {P-Well};
% p-well
\fill[pwell] (11.5,0.75) rectangle (19,2);
\node at (14.25,1) {P-Well};
\fill[gateoxide] (4.8,2) rectangle (6.7,2.3);
\fill[gateoxide] (13.3,2) rectangle (15.2,2.3);
\fill[gatemetal] (4.8,2.3) rectangle (6.7,2.6);
\fill[gatemetal] (13.3,2.3) rectangle (15.2,2.6);
		\node at (3,3) {Gate oxide};
		\draw[->] (3,2.8) -- (5,2.2);
		\node at (3,4) {Polysilicon};
		\draw[->] (3,3.8) -- (5,3.2);
	\end{tikzpicture}
	\begin{tikzpicture}[node distance = 3cm, auto, thick,scale=\CrossAndTopSectionBig, every node/.style={transform shape}]
		\input{tikz_process_steps/sti.b.tex}
\fill[nwell] (1.25,1) rectangle (8.25,7.25);
\fill[pwell] (11.5,1.25) rectangle (19,7.5);

% gate metal
\fill[gatemetal] (5,0) rectangle (6.5,9);
\fill[gatemetal] (13.5,0) rectangle (15,9);
\fill[gatemetal] (5,8) rectangle (15,10);
	\end{tikzpicture}
	\caption{Aluminum gate contacts with gate oxide}
\end{figure}

The line spacing of the polysilicon electrode shape has to be at least 0.5\um because of the resolution of the stepper and also because of the etching process which has 0.5\um as the minimum line spacing.

\begin{figure}[H]
	\centering
	\begin{tikzpicture}[node distance =1cm, auto, thick,scale=\VLSILayout, every node/.style={transform shape}]
		\input{tikz_process_steps/gate.layout.tex}

% n+
\fill[nimplant,opacity=\OpacityLayout] (1.5,2) rectangle (3,6.5);
\fill[nimplant,opacity=\OpacityLayout] (12,2) rectangle (13.5,6.5);
\fill[nimplant,opacity=\OpacityLayout] (15,2) rectangle (16.5,6.5);


% p+
\fill[pimplant,opacity=\OpacityLayout] (3.5,2) rectangle (5,6.5);
\fill[pimplant,opacity=\OpacityLayout] (6.5,2) rectangle (8,6.5);
\fill[pimplant,opacity=\OpacityLayout] (17,2) rectangle (18.5,6.5);
% gate metal
\fill[gatemetal,opacity=\OpacityLayout] (4.8,1.75) rectangle (6.7,8);
\fill[gatemetal,opacity=\OpacityLayout] (13.3,1.75) rectangle (15.2,8);
\fill[gatemetal,opacity=\OpacityLayout] (4.8,8) rectangle (15.2,10);

	\end{tikzpicture}
	\caption{Gate layout}
	\label{gate_layout}
\end{figure}

In \autoref{gate_layout} we can see the layout honoring the 0.5\um spacing design rule for the gate structure shape and poly-layer interconnect between NMOS and PMOS.

\subsubsection{Gate oxide deposition}

\begin{figure}[H]
	\centering
	\begin{tikzpicture}[node distance = 3cm, auto, thick,scale=\CrossSectionOnly, every node/.style={transform shape}]
		\input{tikz_process_steps/gate.1.a.tex}
	\end{tikzpicture} \\
	\includegraphics[scale=0.01]{down_arrow.png} \\
	\begin{tikzpicture}[node distance = 3cm, auto, thick,scale=\CrossSectionOnly, every node/.style={transform shape}]
		\input{tikz_process_steps/gate.1.b.tex}
	\end{tikzpicture}
	\caption{Thin oxide}
\end{figure}

\subsubsection{Polysilicon deposition}

Now we need to add the polysilicon layer for forming the gate structure after etching.

\begin{figure}[H]
	\centering
	\begin{tikzpicture}[node distance = 3cm, auto, thick,scale=\CrossSectionOnly, every node/.style={transform shape}]
		\input{tikz_process_steps/gate.2.a.tex}
	\end{tikzpicture} \\
	\includegraphics[scale=0.01]{down_arrow.png} \\
	\begin{tikzpicture}[node distance = 3cm, auto, thick,scale=\CrossSectionOnly, every node/.style={transform shape}]
		\input{tikz_process_steps/pwell.a.tex}
% n-well
\fill[nwell] (1.25,0.75) rectangle (8.5,2);
\node at (5.75,1) {N-Well};
% p-well
\fill[pwell] (11.75,0.25) rectangle (18.75,2);
\node at (15.25,1) {P-Well};
% p-well
\fill[pwell] (11.5,0.75) rectangle (19,2);
\node at (14.25,1) {P-Well};
\fill[gateoxide] (0,2) rectangle (20,2.3);
\fill[gatemetal] (0,2.3) rectangle (20,3);
	\end{tikzpicture}
	\caption{Polysilicon}
\end{figure}

We use the LPCVD machine(\autoref{lpcvd_machine}) and deposit a layer of around 600nm polysilicon\footnote{\url{https://people.rit.edu/lffeee/LPCVD_Recipes.pdf}}.

We set the temperatue to 650\degreesC, the gas will be Silane ($Si H_4$ ($Si + 2H_2$)), the pressure will be set to 300 mTorr with a flow of 90sccm.

This will give us a growth rate of roughly 23.5 nm per minute, so for 600nm we let it grow half an hour.

\subsubsection{Patterning}

The resist is being deposited using spin coating and then baked depending on the baking time for the specific resist.
The layout for being exposed onto the resist is being extracted from the "poly" layer within the GDS2 file onto a bright field mask.

\begin{figure}[H]
	\centering
	\begin{tikzpicture}[node distance = 3cm, auto, thick,scale=\CrossSectionOnly, every node/.style={transform shape}]
		\input{tikz_process_steps/gate.3.a.tex}
	\end{tikzpicture} \\
	\includegraphics[scale=0.01]{down_arrow.png} \\
	\begin{tikzpicture}[node distance = 3cm, auto, thick,scale=\CrossSectionOnly, every node/.style={transform shape}]
		\input{tikz_process_steps/pwell.a.tex}
% n-well
\fill[nwell] (1.25,0.75) rectangle (8.5,2);
\node at (5.75,1) {N-Well};
% p-well
\fill[pwell] (11.75,0.25) rectangle (18.75,2);
\node at (15.25,1) {P-Well};
% p-well
\fill[pwell] (11.5,0.75) rectangle (19,2);
\node at (14.25,1) {P-Well};
\fill[gateoxide] (0,2) rectangle (20,2.3);
\fill[poly] (0,2.3) rectangle (20,3);

\fill[resist] (5,3) rectangle (6.5,3.6);
\fill[resist] (13.5,3) rectangle (15,3.6);
	\end{tikzpicture}
	\caption{Resist}
\end{figure}

\subsubsection{Etching}

\begin{figure}[H]
	\centering
	\begin{tikzpicture}[node distance = 3cm, auto, thick,scale=\CrossSectionOnly, every node/.style={transform shape}]
		\input{tikz_process_steps/gate.4.a.tex}
	\end{tikzpicture} \\
	\includegraphics[scale=0.01]{down_arrow.png} \\
	\begin{tikzpicture}[node distance = 3cm, auto, thick,scale=\CrossSectionOnly, every node/.style={transform shape}]
		\input{tikz_process_steps/pwell.a.tex}
% n-well
\fill[nwell] (1.25,0.75) rectangle (8.5,2);
\node at (5.75,1) {N-Well};
% p-well
\fill[pwell] (11.75,0.25) rectangle (18.75,2);
\node at (15.25,1) {P-Well};
% p-well
\fill[pwell] (11.5,0.75) rectangle (19,2);
\node at (14.25,1) {P-Well};
\fill[gateoxide] (5,2) rectangle (6.5,2.3);
\fill[gateoxide] (13.5,2) rectangle (15,2.3);
\fill[gatemetal] (5,2.3) rectangle (6.5,3);
\fill[gatemetal] (13.5,2.3) rectangle (15,3);
\fill[resist] (5,3) rectangle (6.5,3.6);
\fill[resist] (13.5,3) rectangle (15,3.6);
	\end{tikzpicture}
	\caption{Resist}
\end{figure}

\subsubsection{Cleaning}

\begin{figure}[H]
	\centering
	\begin{tikzpicture}[node distance = 3cm, auto, thick,scale=\CrossSectionOnly, every node/.style={transform shape}]
		\input{tikz_process_steps/gate.5.a.tex}
	\end{tikzpicture} \\
	\includegraphics[scale=0.01]{down_arrow.png} \\
	\begin{tikzpicture}[node distance = 3cm, auto, thick,scale=\CrossSectionOnly, every node/.style={transform shape}]
		\input{tikz_process_steps/gate.5.b.tex}
	\end{tikzpicture}
	\caption{Resist}
\end{figure}