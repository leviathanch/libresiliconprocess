\section{Via}\label{via}

Now we have to build an additional set of vias connecting the first metal layer to the next metal layer.
These vias are already part of the front-end process.

\begin{figure}[H]
	\centering
	\begin{tikzpicture}[node distance = 3cm, auto, thick,scale=\CrossSectionOnly, every node/.style={transform shape}]
		\fill[isolationoxide] (0,\LowerMetal) rectangle (55,\LowerMoreMetal);

\fill[white] (3.0,\UpperMetal) rectangle (4.0,\LowerMoreMetal);
\fill[white] (5.75,\UpperMetal) rectangle (6.75,\LowerMoreMetal);
\fill[white] (8.5,\UpperMetal) rectangle (9.5,\LowerMoreMetal);
\fill[white] (11.25,\UpperMetal) rectangle (12.25,\LowerMoreMetal);
\fill[white] (14.0,\UpperMetal) rectangle (15.0,\LowerMoreMetal);
\fill[white] (20.5,\UpperMetal) rectangle (21.5,\LowerMoreMetal);
\fill[white] (22.75,\UpperMetal) rectangle (23.75,\LowerMoreMetal);
\fill[white] (24.5,\UpperMetal) rectangle (25.5,\LowerMoreMetal);

\fill[white] (27.0,\UpperMetal) rectangle (27.8,\LowerMoreMetal);
\fill[white] (28.6,\UpperMetal) rectangle (28.9,\LowerMoreMetal);
\fill[white] (29.85,\UpperMetal) rectangle (30.4,\LowerMoreMetal);
\fill[white] (31.35,\UpperMetal) rectangle (31.65,\LowerMoreMetal);
\fill[white] (32.6,\UpperMetal) rectangle (33.4,\LowerMoreMetal);

\fill[white] (43.0,\UpperMetal) rectangle (44.0,\LowerMoreMetal);
\fill[white] (47.0,\UpperMetal) rectangle (48.0,\LowerMoreMetal);

\fill[white] (48.75,\UpperMetal) rectangle (49.75,\LowerMoreMetal);
\fill[white] (53.25,\UpperMetal) rectangle (54.25,\LowerMoreMetal);

\fill[isolationoxide] (0.0,2.0) rectangle (55.0,\LowerMetal);

\paintcontacts{blue}{brown}{blue}

\input{tikz_process_steps/pimplant.a.tex}

\filldraw[line width=0, isolationoxide] (5.75,2.0) -- (5.5,2.0) -- (5.75,3.0);
\filldraw[line width=0, isolationoxide] (6.75,2.0) -- (7.0,2.0) -- (6.75,3.0);

\filldraw[line width=0, isolationoxide] (11.25,2.0) -- (11.0,2.0) -- (11.25,3.0);
\filldraw[line width=0, isolationoxide] (12.5,2.0) -- (12.25,2.0) -- (12.25,3.0);

\filldraw[line width=0, isolationoxide] (22.75,2.4) -- (22.6,2.4) -- (22.75,3.4);
\filldraw[line width=0, isolationoxide] (23.75,2.4) -- (23.9,2.4) -- (23.75,3.4);

\fill[silicide] (1.5,1.9) rectangle (2.5,2);
\fill[silicide] (4.5,1.9) rectangle (5.5,2);
\fill[silicide] (5.75,2.9) rectangle (6.75,3.0);
\fill[silicide] (7,1.9) rectangle (8,2);

\fill[silicide] (10.0,1.9) rectangle (11.0,2);
\fill[silicide] (11.25,2.9) rectangle (12.25,3.0);
\fill[silicide] (12.5,1.9) rectangle (13.5,2);
\fill[silicide] (15.5,1.9) rectangle (16.5,2);

\fill[silicide] (19.0,1.9) rectangle (20.0,2);

\fill[silicide] (22.0,1.9) rectangle (22.6,2.0);
\fill[silicide] (22.75,3.3) rectangle (23.75,3.4);
\fill[silicide] (23.9,1.9) rectangle (24.5,2.0);

\fill[silicide] (27.0,1.9) rectangle (27.75,2.0);
\fill[silicide] (28.5,1.9) rectangle (29.0,2.0);
\fill[silicide] (29.75,1.9) rectangle (30.5,2.0);
\fill[silicide] (31.25,1.9) rectangle (31.75,2.0);
\fill[silicide] (32.5,1.9) rectangle (33.5,2.0);

\fill[silicide] (35.75,1.9) rectangle (36.5,2.0);
\fill[silicide] (37.25,1.9) rectangle (37.75,2.0);
\fill[silicide] (38.5,1.9) rectangle (39.0,2.0);
\fill[silicide] (39.75,1.9) rectangle (40.25,2.0);
\fill[silicide] (41.0,1.9) rectangle (41.75,2.0);

% diode contacts
\fill[silicide] (43.0,3.4) rectangle (44.0,3.5);
\fill[silicide] (47.0,3.4) rectangle (48.0,3.5);

% resistor contacts
\fill[silicide] (49.0,3.4) rectangle (49.5,3.5);
\fill[silicide] (53.5,3.4) rectangle (54.0,3.5);



	\end{tikzpicture}
	\caption{Contact geometry target}
	\label{via_cross_sections}
\end{figure}

As can be seen in \autoref{via_cross_sections}, the goal of this step is purely to deposit a layer of isolation oxide, get the holes into it, down to the metal layer below in order to form wires later on.

\begin{figure}[H]
	\centering
	\begin{tikzpicture}[node distance =1cm, auto, thick,scale=\VLSILayout, every node/.style={transform shape}]
		\input{tikz_process_steps/silicification.layout.tex}
%vias

\fill[contact,opacity=\OpacityLayout] (7,1.5) rectangle (8,2.5);
\fill[contact,opacity=\OpacityLayout] (7,3.5) rectangle (8,4.5);
\fill[contact,opacity=\OpacityLayout] (7,5.5) rectangle (8,6.5);

\fill[contact,opacity=\OpacityLayout] (12,1.5) rectangle (13,2.5);
\fill[contact,opacity=\OpacityLayout] (12,3.5) rectangle (13,4.5);
\fill[contact,opacity=\OpacityLayout] (12,5.5) rectangle (13,6.5);

\fill[contact,opacity=\OpacityLayout] (1.5,1.5) rectangle (2.5,2.5);
\fill[contact,opacity=\OpacityLayout] (1.5,3.5) rectangle (2.5,4.5);
\fill[contact,opacity=\OpacityLayout] (1.5,5.5) rectangle (2.5,6.5);

\fill[contact,opacity=\OpacityLayout] (3.5,1.5) rectangle (4.5,2.5);
\fill[contact,opacity=\OpacityLayout] (3.5,3.5) rectangle (4.5,4.5);
\fill[contact,opacity=\OpacityLayout] (3.5,5.5) rectangle (4.5,6.5);

\fill[contact,opacity=\OpacityLayout] (15.5,1.5) rectangle (16.5,2.5);
\fill[contact,opacity=\OpacityLayout] (15.5,3.5) rectangle (16.5,4.5);
\fill[contact,opacity=\OpacityLayout] (15.5,5.5) rectangle (16.5,6.5);

\fill[contact,opacity=\OpacityLayout] (17.5,1.5) rectangle (18.5,2.5);
\fill[contact,opacity=\OpacityLayout] (17.5,3.5) rectangle (18.5,4.5);
\fill[contact,opacity=\OpacityLayout] (17.5,5.5) rectangle (18.5,6.5);

\fill[contact,opacity=\OpacityLayout] (5.5,8.5) rectangle (6.5,9.5); % contact out
\fill[contact,opacity=\OpacityLayout] (7.5,8.5) rectangle (8.5,9.5); % contact out
\fill[contact,opacity=\OpacityLayout] (9.5,8.5) rectangle (10.5,9.5); % contact out
\fill[contact,opacity=\OpacityLayout] (11.5,8.5) rectangle (12.5,9.5); % contact out
\fill[contact,opacity=\OpacityLayout] (13.5,8.5) rectangle (14.5,9.5); % contact out

\draw[|<->|] (7.5,8.25) -- (8.5,8.25);
\node at (8,7.75) {$\lambda$};

\draw[|<->|] (7.25,8.5) -- (7.25,9.5);
\node[rotate=90] at (6.75,8.75) {$\lambda$};

\fill[metal1,opacity=\OpacityLayout] (7,8) rectangle (13,12);
\fill[metal1,opacity=\OpacityLayout] (1.0,0) rectangle (5.0,12);
\fill[metal1,opacity=\OpacityLayout] (6.5,1) rectangle (13.5,7);
\fill[metal1,opacity=\OpacityLayout] (8,0) rectangle (12,1);
\fill[metal1,opacity=\OpacityLayout] (15.0,0) rectangle (19.0,12);

\node at (16,11.5) {VDD};
\node at (2.5,11.5) {GND};
\node at (10,11.5) {Input};
\node at (10,0.5) {Output};
\fill[via1,opacity=\OpacityLayout] (2,9) rectangle (4,11);
\fill[via1,opacity=\OpacityLayout] (16,9) rectangle (18,11);
	\end{tikzpicture}
	\caption{First via layout}
	\label{via_layout}
\end{figure}

It should be noted again that the via placement and dimensions in \autoref{via_layout} are solely for demonstration purposes for the process and are in no way the actual standard cell design for the final standard cell lib. \\

In a later iterations of this process we might be switching to Copper as the metal material for this step which will result in a variation of this step because the usage of damascene method.

\newpage

\subsection{Isolation dioxide layer}

We now need to grow a layer of thick oxide in order to isolate the Aluminum interconnect layer from the active area.

\begin{figure}[H]
	\centering
	\begin{tikzpicture}[node distance = 3cm, auto, thick,scale=\CrossSectionOnly, every node/.style={transform shape}]
		\fill[isolationoxide] (0.0,2.0) rectangle (55.0,\LowerMetal);

\paintcontacts{blue}{brown}{blue}

\input{tikz_process_steps/pimplant.a.tex}

\filldraw[line width=0, isolationoxide] (5.75,2.0) -- (5.5,2.0) -- (5.75,3.0);
\filldraw[line width=0, isolationoxide] (6.75,2.0) -- (7.0,2.0) -- (6.75,3.0);

\filldraw[line width=0, isolationoxide] (11.25,2.0) -- (11.0,2.0) -- (11.25,3.0);
\filldraw[line width=0, isolationoxide] (12.5,2.0) -- (12.25,2.0) -- (12.25,3.0);

\filldraw[line width=0, isolationoxide] (22.75,2.4) -- (22.6,2.4) -- (22.75,3.4);
\filldraw[line width=0, isolationoxide] (23.75,2.4) -- (23.9,2.4) -- (23.75,3.4);

\fill[silicide] (1.5,1.9) rectangle (2.5,2);
\fill[silicide] (4.5,1.9) rectangle (5.5,2);
\fill[silicide] (5.75,2.9) rectangle (6.75,3.0);
\fill[silicide] (7,1.9) rectangle (8,2);

\fill[silicide] (10.0,1.9) rectangle (11.0,2);
\fill[silicide] (11.25,2.9) rectangle (12.25,3.0);
\fill[silicide] (12.5,1.9) rectangle (13.5,2);
\fill[silicide] (15.5,1.9) rectangle (16.5,2);

\fill[silicide] (19.0,1.9) rectangle (20.0,2);

\fill[silicide] (22.0,1.9) rectangle (22.6,2.0);
\fill[silicide] (22.75,3.3) rectangle (23.75,3.4);
\fill[silicide] (23.9,1.9) rectangle (24.5,2.0);

\fill[silicide] (27.0,1.9) rectangle (27.75,2.0);
\fill[silicide] (28.5,1.9) rectangle (29.0,2.0);
\fill[silicide] (29.75,1.9) rectangle (30.5,2.0);
\fill[silicide] (31.25,1.9) rectangle (31.75,2.0);
\fill[silicide] (32.5,1.9) rectangle (33.5,2.0);

\fill[silicide] (35.75,1.9) rectangle (36.5,2.0);
\fill[silicide] (37.25,1.9) rectangle (37.75,2.0);
\fill[silicide] (38.5,1.9) rectangle (39.0,2.0);
\fill[silicide] (39.75,1.9) rectangle (40.25,2.0);
\fill[silicide] (41.0,1.9) rectangle (41.75,2.0);

% diode contacts
\fill[silicide] (43.0,3.4) rectangle (44.0,3.5);
\fill[silicide] (47.0,3.4) rectangle (48.0,3.5);

% resistor contacts
\fill[silicide] (49.0,3.4) rectangle (49.5,3.5);
\fill[silicide] (53.5,3.4) rectangle (54.0,3.5);


	\end{tikzpicture}
	\drawStepArrow{LTO\\deposition}
	\begin{tikzpicture}[node distance = 3cm, auto, thick,scale=\CrossSectionOnly, every node/.style={transform shape}]
		\input{tikz_process_steps/via.oxide_growth.b.tex}
	\end{tikzpicture}
	\caption{Oxide layer}
\end{figure}

\textbf{Possible approaches}:
\begin{itemize}
	\item \textbf{"LPCVD-B3 LTO (CVD-B3)" from HKUST} \\
	At HKUST we have a chemical vapor deposition unit which gives us better control over the layer thicknes. \\
	These steps are needed to arrive with the desired geometry\footnote{\url{http://memslab.blogspot.com/2013/01/lto-lpcvd.html}}
	\begin{enumerate}
		\item Set the growth rate to 14 nm/min
		\item Run for 140 minutes
	\end{enumerate}
	\item \textbf{In a furnace ("a hack around")} \\
	In case of a lack of LPCVD equipment one might also resort to "hack together" a solution for LTO deposition using a furnace\footnote{\url{https://www.sciencedirect.com/science/article/pii/0167577X89900062}}
		\begin{enumerate}
			\item Deposit tetraethyl orthosilicate ($Si C_8 H_{20} O_4$)
			\item React for 20 minutes at 1050\degreesC in $N_2$ environment in a furnace
	\end{enumerate}
\end{itemize}

\newpage

\subsection{Etching}\label{via_etching}

We now need to open a window in the dioxide layer, through which we will inject carrier atoms into the silicon crystal structure.

\begin{figure}[H]
	\centering
	\begin{tikzpicture}[node distance = 3cm, auto, thick,scale=\CrossAndTopSection, every node/.style={transform shape}]
		\input{tikz_process_steps/via.etching.a.tex}
	\end{tikzpicture}
	\begin{tikzpicture}[node distance = 3cm, auto, thick,scale=\CrossAndTopSection, every node/.style={transform shape}]
		\input{tikz_process_steps/via.etching.at.tex}
	\end{tikzpicture}
	\drawStepArrow{}
	\begin{tikzpicture}[node distance = 3cm, auto, thick,scale=\CrossAndTopSection, every node/.style={transform shape}]
		\input{tikz_process_steps/via.etching.b.tex}
	\end{tikzpicture}
	\begin{tikzpicture}[node distance = 3cm, auto, thick,scale=\CrossAndTopSection, every node/.style={transform shape}]
		\input{tikz_process_steps/via.etching.bt.tex}
	\end{tikzpicture}
	\caption{N+ region opened}
\end{figure}

Since the silicon dioxide layer is 100nm thick and we wanna reach the silicon below we can use wet etching as described in the chemistry chapter.\\

\textbf{Possible approaches}:
\begin{itemize}
	\item \textbf{"AOE Etcher (DRY-AOE)" from HKUST} \\
	We can use anisotropic plasma etching for sharper borders.
	\item \textbf{Chemical solution} \\
	We can use buffered hydrofluoric acid (BOE (1:6)) at room temperature ($\approx$508 nm/min) for around 4 minutes in order to get through the 2\um of oxide.\\
	Too long over 4 minutes might cause under-etch however!
\end{itemize}
