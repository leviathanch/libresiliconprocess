\subsection{Metal 1}\label{chapter_metal1}

Now we've got to build the first interconnect wires, connecting the contact vias to the "metal1" wires, which will provide a way to contact to them with the via1 contact layout.

\begin{figure}[H]
	\centering
	\begin{tikzpicture}[node distance = 3cm, auto, thick,scale=\CrossAndTopSectionBig, every node/.style={transform shape}]
		\fill[nitride] (0.0,\LowerMetal) rectangle (55.0,\LowerMetal+0.5);
\fill[isolationoxide] (0.0,2.0) rectangle (55.0,\LowerMetal+0.25);

\paintscaledmetal{nitride}{0.25}{0.5}
\paintscaledmetal{isolationoxide}{0.0}{0.25}

\input{tikz_process_steps/nimplant.a.tex}
\fill[pimplant] (3.5,1) rectangle (5,2);
\node at (4,1.5) {p+};
\fill[pimplant] (6.5,1) rectangle (8,2);
\node at (7,1.5) {p+};
\fill[pimplant] (17,1) rectangle (18.5,2);
\node at (18,1.5) {p+};

\filldraw[line width=0, isolationoxide] (5.75,2.0) -- (5.5,2.0) -- (5.75,3.0);
\filldraw[line width=0, isolationoxide] (6.75,2.0) -- (7.0,2.0) -- (6.75,3.0);

\filldraw[line width=0, isolationoxide] (11.25,2.0) -- (11.0,2.0) -- (11.25,3.0);
\filldraw[line width=0, isolationoxide] (12.5,2.0) -- (12.25,2.0) -- (12.25,3.0);

\filldraw[line width=0, isolationoxide] (22.75,2.4) -- (22.6,2.4) -- (22.75,3.4);
\filldraw[line width=0, isolationoxide] (23.75,2.4) -- (23.9,2.4) -- (23.75,3.4);

\fill[silicide] (1.5,1.9) rectangle (2.5,2);
\fill[silicide] (4.5,1.9) rectangle (5.5,2);
\fill[silicide] (5.75,2.9) rectangle (6.75,3.0);
\fill[silicide] (7,1.9) rectangle (8,2);

\fill[silicide] (10.0,1.9) rectangle (11.0,2);
\fill[silicide] (11.25,2.9) rectangle (12.25,3.0);
\fill[silicide] (12.5,1.9) rectangle (13.5,2);
\fill[silicide] (15.5,1.9) rectangle (16.5,2);

\fill[silicide] (19.0,1.9) rectangle (20.0,2);

\fill[silicide] (22.0,1.9) rectangle (22.6,2.0);
\fill[silicide] (22.75,3.3) rectangle (23.75,3.4);
\fill[silicide] (23.9,1.9) rectangle (24.5,2.0);

\fill[silicide] (27.0,1.9) rectangle (27.75,2.0);
\fill[silicide] (28.5,1.9) rectangle (29.0,2.0);
\fill[silicide] (29.75,1.9) rectangle (30.5,2.0);
\fill[silicide] (31.25,1.9) rectangle (31.75,2.0);
\fill[silicide] (32.5,1.9) rectangle (33.5,2.0);

\fill[silicide] (35.75,1.9) rectangle (36.5,2.0);
\fill[silicide] (37.25,1.9) rectangle (37.75,2.0);
\fill[silicide] (38.5,1.9) rectangle (39.0,2.0);
\fill[silicide] (39.75,1.9) rectangle (40.25,2.0);
\fill[silicide] (41.0,1.9) rectangle (41.75,2.0);

% diode contacts
\fill[silicide] (43.0,3.4) rectangle (44.0,3.5);
\fill[silicide] (47.0,3.4) rectangle (48.0,3.5);

% resistor contacts
\fill[silicide] (49.0,3.4) rectangle (49.5,3.5);
\fill[silicide] (53.5,3.4) rectangle (54.0,3.5);


\paintcontacts{brown}{gray}{brown}

	\end{tikzpicture}
	\caption{Metal geometry target}
	\label{metal_target}
\end{figure}

As can be seen in \autoref{metal_target}, the goal of this step is purely to etch the wire structure for the first metal layer into the in \autoref{metal_deposition} deposited metal layer, and form wires by doing so.

\begin{figure}[H]
	\centering
	\begin{tikzpicture}[node distance =1cm, auto, thick,scale=\VLSILayout, every node/.style={transform shape}]
		\input{tikz_process_steps/pimplant.layout.tex}
%vias

\fill[contact,opacity=\OpacityLayout] (7,1.5) rectangle (8,2.5);
\fill[contact,opacity=\OpacityLayout] (7,3.5) rectangle (8,4.5);
\fill[contact,opacity=\OpacityLayout] (7,5.5) rectangle (8,6.5);

\fill[contact,opacity=\OpacityLayout] (12,1.5) rectangle (13,2.5);
\fill[contact,opacity=\OpacityLayout] (12,3.5) rectangle (13,4.5);
\fill[contact,opacity=\OpacityLayout] (12,5.5) rectangle (13,6.5);

\fill[contact,opacity=\OpacityLayout] (1.5,1.5) rectangle (2.5,2.5);
\fill[contact,opacity=\OpacityLayout] (1.5,3.5) rectangle (2.5,4.5);
\fill[contact,opacity=\OpacityLayout] (1.5,5.5) rectangle (2.5,6.5);

\fill[contact,opacity=\OpacityLayout] (3.5,1.5) rectangle (4.5,2.5);
\fill[contact,opacity=\OpacityLayout] (3.5,3.5) rectangle (4.5,4.5);
\fill[contact,opacity=\OpacityLayout] (3.5,5.5) rectangle (4.5,6.5);

\fill[contact,opacity=\OpacityLayout] (15.5,1.5) rectangle (16.5,2.5);
\fill[contact,opacity=\OpacityLayout] (15.5,3.5) rectangle (16.5,4.5);
\fill[contact,opacity=\OpacityLayout] (15.5,5.5) rectangle (16.5,6.5);

\fill[contact,opacity=\OpacityLayout] (17.5,1.5) rectangle (18.5,2.5);
\fill[contact,opacity=\OpacityLayout] (17.5,3.5) rectangle (18.5,4.5);
\fill[contact,opacity=\OpacityLayout] (17.5,5.5) rectangle (18.5,6.5);

\fill[contact,opacity=\OpacityLayout] (5.5,8.5) rectangle (6.5,9.5); % contact out
\fill[contact,opacity=\OpacityLayout] (7.5,8.5) rectangle (8.5,9.5); % contact out
\fill[contact,opacity=\OpacityLayout] (9.5,8.5) rectangle (10.5,9.5); % contact out
\fill[contact,opacity=\OpacityLayout] (11.5,8.5) rectangle (12.5,9.5); % contact out
\fill[contact,opacity=\OpacityLayout] (13.5,8.5) rectangle (14.5,9.5); % contact out

\draw[|<->|] (7.5,8.25) -- (8.5,8.25);
\node at (8,7.75) {$\lambda$};

\draw[|<->|] (7.25,8.5) -- (7.25,9.5);
\node[rotate=90] at (6.75,8.75) {$\lambda$};

\fill[metal1,opacity=\OpacityLayout] (7,8) rectangle (13,12);
\fill[metal1,opacity=\OpacityLayout] (1.0,0) rectangle (5.0,12);
\fill[metal1,opacity=\OpacityLayout] (6.5,1) rectangle (13.5,7);
\fill[metal1,opacity=\OpacityLayout] (8,0) rectangle (12,1);
\fill[metal1,opacity=\OpacityLayout] (15.0,0) rectangle (19.0,12);

\node at (16,11.5) {VDD};
\node at (2.5,11.5) {GND};
\node at (10,11.5) {Input};
\node at (10,0.5) {Output};
	\end{tikzpicture}
	\caption{First metal layout}
	\label{metal_layout}
\end{figure}

It should be noted again that the via placement and dimensions in \autoref{metal_layout} are solely for demonstration purposes for the process and are in no way the actual standard cell design for the final standard cell lib. \\

In later iterations of this process we might be switching to Tungsten as the metal material for this step so the etching method might change in further releases.

\newpage

\subsubsection{Metal deposition}\label{metal_deposition}

Now we somehow have got to get the metal onto our silicon oxide in a fashion so that it fills the holes we've etched in \autoref{contact_holes_etch} and touches down onto the silicide/polycide, thus making a contact to the active area.

\begin{figure}[H]
	\centering
	\begin{tikzpicture}[node distance = 3cm, auto, thick,scale=\CrossSectionOnly, every node/.style={transform shape}]
		\input{tikz_process_steps/metal.metal_deposition.a.tex}
	\end{tikzpicture}
	\drawStepArrow{}
	\begin{tikzpicture}[node distance = 3cm, auto, thick,scale=\CrossSectionOnly, every node/.style={transform shape}]
		\input{tikz_process_steps/metal.metal_deposition.b.tex}
	\end{tikzpicture}
	\caption{Metal deposition}
\end{figure}

In order to reach the target of filling the holes in the oxide and having at least another depth worth of space in order to have an enough low resistance of the wire interconnect.
We end up with a target thickness of 4\um.

\textbf{Possible approaches}:
\begin{itemize}
	\item \textbf{"Varian 3180 Sputter (SPT-3180)" from HKUST} \\
	The deposition speed is 16nm/s which gives us a required deposition time of 250 seconds for 4\um.
	\item \textbf{Add your solution here!}
\end{itemize}

\newpage

\subsubsection{Etching}\label{metal_wire_etch}

Now we've got to etch the Aluminum which hasn't been covered yet by the resist in order to get the desired wire structures.

\begin{figure}[H]
	\centering
	\begin{tikzpicture}[node distance = 3cm, auto, thick,scale=\CrossAndTopSection, every node/.style={transform shape}]
		\input{tikz_process_steps/metal.etching.a.tex}
	\end{tikzpicture}
	\begin{tikzpicture}[node distance = 3cm, auto, thick,scale=\CrossAndTopSection, every node/.style={transform shape}]
		\fill[metal1] (0,0) rectangle (20,12);
\fill[resist] (7,8) rectangle (13,12);
\fill[resist] (1.0,0) rectangle (5.0,12);
\fill[resist] (6.5,1) rectangle (13.5,7);
\fill[resist] (8,0) rectangle (12,1);
\fill[resist] (15.0,0) rectangle (19.0,12);

	\end{tikzpicture}
	\drawStepArrow{Mask: metal1}
	\begin{tikzpicture}[node distance = 3cm, auto, thick,scale=\CrossAndTopSection, every node/.style={transform shape}]
		\input{tikz_process_steps/metal.etching.b.tex}
	\end{tikzpicture}
	\begin{tikzpicture}[node distance = 3cm, auto, thick,scale=\CrossAndTopSection, every node/.style={transform shape}]
		\input{tikz_process_steps/metal.etching.bt.tex}
	\end{tikzpicture}
	\caption{Etching first wires}
\end{figure}

\textbf{Possible approaches}:
\begin{itemize}
	\item \textbf{"Oxford Aluminum Etcher (DRY-Metal-2)" from HKUST} \\
	The normal etch rate for Aluminum is 180 nm/min with this machines \\
	We've deposited 4\um Aluminum in \autoref{metal_deposition} which means we've got to etch for around 22 minutes and 13 seconds
	\item \textbf{Chemical solution} \\
	Please specify here!
\end{itemize}
