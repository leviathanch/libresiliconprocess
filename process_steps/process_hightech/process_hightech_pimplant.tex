\subsection{p+ Implant}\label{pimplant_chapter}
For the bulk of the NMOS transistors and for the source and drain of the PMOS transistors highly doped  p+ areas are required.
In this step we're going to build these.

\begin{figure}[H]
	\centering
	\begin{tikzpicture}[node distance = 3cm, auto, thick,scale=\CrossAndTopSectionBig, every node/.style={transform shape}]
		% substrate
\fill[YellowOrange] (0,0) rectangle (20,2);
\node at (2,0.5) {Si (p-type)};
% n-well
\fill[Goldenrod] (1.25,0.75) rectangle (8.25,2);
\node at (5.75,1) {N-Well};
% body
\fill[ProcessBlue] (1.5,1) rectangle (3,2);
\node at (2,1.5) {n+};
% source
\fill[RedOrange] (3.5,1) rectangle (5,2);
\node at (4,1.5) {p+};
% drain
\fill[RedOrange] (6.5,1) rectangle (8,2);
\node at (7,1.5) {p+};
%% gate:
% gate oxide
\fill[LightGray] (4.8,2) rectangle (6.7,2.1);
% gate poly
\fill[BrickRed] (4.8,2.1) rectangle (6.7,2.2);

%field oxides:
\fill[DarkGray] (0,2) rectangle (1,4);
\fill[DarkGray] (8.5,2) rectangle (11.5,4);
\fill[DarkGray] (19,2) rectangle (20,4);

\fill[RedOrange] (0,1.5) rectangle (1,2);
\fill[RedOrange] (8.5,1.5) rectangle (11.5,2);
\fill[RedOrange] (19,1.5) rectangle (20,2);

\node at (0.5,1.75) {p+};
\node at (9.5,1.75) {p+};
\node at (19.5,1.75) {p+};

%%% nmos:
% body
\fill[RedOrange] (17,1) rectangle (18.5,2);
\node at (18,1.5) {p+};
% source
\fill[ProcessBlue] (15,1) rectangle (16.5,2);
\node at (16,1.5) {n+};
% drain
\fill[ProcessBlue] (12,1) rectangle (13.5,2);
\node at (13,1.5) {n+};

%% gate:
% gate oxide
\fill[LightGray] (13.3,2) rectangle (15.2,2.1);
% gate poly
\fill[BrickRed] (13.3,2.1) rectangle (15.2,2.2);
\fill[pimplant] (3.5,1) rectangle (5,2);
\node at (4,1.5) {p+};
\fill[pimplant] (6.5,1) rectangle (8,2);
\node at (7,1.5) {p+};
\fill[pimplant] (17,1) rectangle (18.5,2);
\node at (18,1.5) {p+};
	\end{tikzpicture}
	\begin{tikzpicture}[node distance = 3cm, auto, thick,scale=\CrossAndTopSectionBig, every node/.style={transform shape}]
		\fill[YellowOrange] (0,0) rectangle (20,10);

% n-well
\fill[Goldenrod] (1,1.25) rectangle (8.5,7.5);

% p+
\fill[RedOrange] (3.5,2) rectangle (5,6.5);
\fill[RedOrange] (6.5,2) rectangle (8,6.5);
\fill[RedOrange] (17,2) rectangle (18.5,6.5);

% n+
\fill[ProcessBlue] (1.5,2) rectangle (3,6.5);
\fill[ProcessBlue] (12,2) rectangle (13.5,6.5);
\fill[ProcessBlue] (15,2) rectangle (16.5,6.5);

% trench area
\fill[DarkGray] (0,0) rectangle (1,12);
\fill[DarkGray] (8.5,0) rectangle (11.5,12);
\fill[DarkGray] (19,0) rectangle (20,12);
\fill[DarkGray] (0,0) rectangle (20,1.25);
\fill[DarkGray] (0,7.5) rectangle (20,12);

% poly
\fill[BrickRed] (4.8,1.75) rectangle (6.7,9);
\fill[BrickRed] (13.3,1.75) rectangle (15.2,9);
\fill[BrickRed] (4.8,8) rectangle (15.2,9);
	\end{tikzpicture}
	\caption{P+ implant geometry target}
\end{figure}

The tricky thing here is to have a reasonable implant depth but not too deep because the deeper the junction, the higher the junction capacity which in turn limits the switching performance of the CMOS circuitry.

\begin{figure}[H]
	\centering
	\begin{tikzpicture}[node distance =1cm, auto, thick,scale=\VLSILayout, every node/.style={transform shape}]
		\input{tikz_process_steps/pimplant.layout.tex}
% gate metal
\fill[gatemetal,opacity=\OpacityLayout] (4.8,1.75) rectangle (6.7,8);
\fill[gatemetal,opacity=\OpacityLayout] (13.3,1.75) rectangle (15.2,8);
\fill[gatemetal,opacity=\OpacityLayout] (4.8,8) rectangle (15.2,10);


% n+
\fill[nimplant,opacity=\OpacityLayout] (1.5,2) rectangle (3,6.5);
\fill[nimplant,opacity=\OpacityLayout] (12,2) rectangle (13.5,6.5);
\fill[nimplant,opacity=\OpacityLayout] (15,2) rectangle (16.5,6.5);


% p+
\fill[pimplant,opacity=\OpacityLayout] (3.5,2) rectangle (5,6.5);
\fill[pimplant,opacity=\OpacityLayout] (6.5,2) rectangle (8,6.5);
\fill[pimplant,opacity=\OpacityLayout] (17,2) rectangle (18.5,6.5);
	\end{tikzpicture}
	\caption{P+ layout}
	\label{pimplant_layout}
\end{figure}

An example layout of p-implants can be seen in \autoref{pimplant_layout}, the mask is being extracted from the layer "p\_plus\_select"

Also important to notice is that this example layout is just for demonstration purposes only, please have a look at the standard cell documentation for the actual layouts. 

The pselect is implanted with a Boron ($B^{11}$) dose of $2.5\times10^{12}cm^{-2}$ at an energy of 20 keV  (43nm$\pm$18nm deep)

\newpage
