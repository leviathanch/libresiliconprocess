\section{Tripple Well}\label{tripple_well_chapter}

In order to build BiCMOS we need nested wells for getting the vertical diode structures which form the bi junction transistors.

A vertical isolation, which allows us to have some bulk areas on a higher potential than others, and isolated FETs come along from this tripple well architecture for free.

The cross section of the targeted geometry are shown in \autoref{tripple_well_target}

\begin{figure}[H]
	\centering
	\begin{tikzpicture}[node distance = 3cm, auto, thick,scale=\CrossAndTopSectionBig, every node/.style={transform shape}]
		% substrate
\fill[substrate] (0,0) rectangle (20,1.25);
\node at (2,0.5) {Silicon substrate};
\fill[substrate] (0.25,1.25) rectangle (19.75,2);

\def\welldepthAnbase{1.25}
\def\welldepthBnbase{1.5}
\def\welldepthCnbase{1.75}

% normal wells
\shade[upper left = nwell, upper right = nwell, lower right = substrate, lower left = substrate,] (1.25,\welldepthAnbase) rectangle (8.25,2.0);
\shade[upper left = pwell, upper right = pwell, lower right = substrate, lower left = substrate,] (9.75,\welldepthAnbase) rectangle (16.75,2.0);
\shade[upper left = nwell, upper right = nwell, lower right = substrate, lower left = substrate,] (18.25,\welldepthAnbase) rectangle (25.25,2.0);
\shade[upper left = nwell, upper right = nwell, lower right = substrate, lower left = substrate,] (26.75,\welldepthAnbase) rectangle (33.75,2.0);
\shade[upper left = nwell, upper right = nwell, lower right = substrate, lower left = substrate,] (35.25,\welldepthAnbase) rectangle (42.25,2.0);

% isolation nwell pwell in nwell for SONOS flash
\shade[upper left = pbase, upper right = pbase, lower right = nwell, lower left = nwell,] (18.5,\welldepthBnbase) rectangle (25.0,2.0);

% npn - pbase
\shade[upper left = pbase, upper right = pbase, lower right = nwell, lower left = nwell,] (28.25,\welldepthBnbase) rectangle (32.0,2.0);

% pnp emitter and collector ring
\shade[upper left = pbase, upper right = pbase, lower right = nwell, lower left = nwell,] (35.5,\welldepthBnbase) rectangle (37.0,2.0);
\shade[upper left = pbase, upper right = pbase, lower right = nwell, lower left = nwell,] (38.0,\welldepthBnbase) rectangle (39.5,2.0);
\shade[upper left = pbase, upper right = pbase, lower right = nwell, lower left = nwell,] (40.5,\welldepthBnbase) rectangle (42.0,2.0);

% n base for SONOS
\shade[upper left = nbase, upper right = nbase, lower right = pbase, lower left = pbase,] (18.75,\welldepthCnbase) rectangle (24.75,2.0);

% n base for NPN emitter
\shade[upper left = nbase, upper right = nbase, lower right = pbase, lower left = pbase,] (29.5,\welldepthCnbase) rectangle (30.75,2.0);

	\end{tikzpicture}
	\caption{Tripple well target geometry}
	\label{tripple_well_target}
\end{figure}

Since the diffusion constant variates with the concentration of background dopants, we have to make sure that the thermal budget has enough slack during every single tripple well formation step, in order to avoid the consumption of one of the wells during further processing.

\begin{figure}[H]
	\centering
	\begin{tikzpicture}[node distance =1cm, auto, thick,scale=\VLSILayout, every node/.style={transform shape}]
		\fill[Goldenrod,opacity=0.2] (0.75,0.5) rectangle (8.75,7.75);
\fill[Goldenrod,opacity=0.2] (11.25,0.5) rectangle (19.25,7.75);

\draw[dotted] (20.5,0.5) rectangle (25,5.5);

\node at (22.25,5) {\textbf{Layers}};

\fill[Goldenrod,opacity=0.2] (21,1) rectangle (21.5,1.5);
\node at (22.25,1.25) {active};

\fill[orange,opacity=0.2] (21,1.5) rectangle (21.5,2);
\node at (22.25,1.75) {nwell};

\fill[blue,opacity=0.2] (21,2) rectangle (21.5,2.5);
\node at (22.25,2.25) {nimplant};

\fill[red,opacity=0.2] (21,2.5) rectangle (21.5,3);
\node at (22.25,2.75) {pimplant};

\fill[Emerald,opacity=0.2] (21,3) rectangle (21.5,3.5);
\node at (22.25,3.25) {gate};

\fill[Fuchsia,opacity=0.2] (21,3.5) rectangle (21.5,4);
\node at (22.25,3.75) {metal1};

\fill[DarkOrchid,opacity=0.2] (21,4) rectangle (21.5,4.5);
\node at (22.25,4.25) {via1};
\fill[pwell,opacity=\OpacityLayout] (11.5,0.75) rectangle (19,7.5);
	\end{tikzpicture}
	\caption{P-Well layout}
	\label{pwell_layout}
\end{figure}

In \autoref{pwell_layout} the layout of the well and base regions on top of the active area region can be seen.

The implant values are as calculated in the documentation of the process design leading to these steps\footnote{\url{https://github.com/leviathanch/libresiliconprocess/raw/master/process_design/process_design.pdf}}.

\newpage
\subsection{N-well}\label{nwell_chapter}
In order to build CMOS on the same substrate, an N-well is required for building the complementary P-channel transistor for a NFET+PFET logic circuitry.

The cross section as well as the top view of the targeted geometry are shown in \autoref{nwell_target}

\begin{figure}[H]
	\centering
	\begin{tikzpicture}[node distance = 3cm, auto, thick,scale=\CrossAndTopSectionBig, every node/.style={transform shape}]
		\input{tikz_process_steps/nwell.a.tex}
% p-well
\fill[pwell] (11.5,0.75) rectangle (19,2);
\node at (14.25,1) {P-Well};
% n-well
\fill[nwell] (1.25,0.75) rectangle (8.5,2);
\node at (5.75,1) {N-Well};
% p-well
\fill[pwell] (11.75,0.25) rectangle (18.75,2);
\node at (15.25,1) {P-Well};
	\end{tikzpicture}
	\begin{tikzpicture}[node distance = 3cm, auto, thick,scale=\CrossAndTopSectionBig, every node/.style={transform shape}]
		% substrate
\fill[YellowOrange] (0,0) rectangle (20,12);
% trench area
\fill[DarkGray] (0,0) rectangle (1,12);
\fill[DarkGray] (8.5,0) rectangle (11.5,12);
\fill[DarkGray] (19,0) rectangle (20,12);
\fill[DarkGray] (0,0) rectangle (20,1.25);
\fill[DarkGray] (0,7.5) rectangle (20,12);
\fill[nwell] (1.25,1) rectangle (8.25,7.25);
	\end{tikzpicture}
	\caption{N-well target geometry}
	\label{nwell_target}
\end{figure}

The N-well will serve us as an island of N-doped substrate within the P-doped basis substrate.

The P-dopant concentration of our prime grade, p-type, single side polished, four inch wafers is between $8.76 \cdot 10^{14} \frac{1}{cm^3}$ and $5.23 \cdot 10^{14} \frac{1}{cm^3}$

This means we need a dose of $2.33\times10^{12}cm^{-2}$ Phosphorus at 70 keV.

The concentration will need adjustment when the used substrate has different properties!

`\section{P-well}\label{pwell_chapter}
In order to build CMOS on the same substrate, a P-well is required for building the complementary N-channel transistor for a n-p-channel logic circuitry.
The cross section as well as the top view of the targeted geometry are shown in \autoref{nwell_target}
\begin{figure}[H]
	\centering
	\begin{tikzpicture}[node distance = 3cm, auto, thick,scale=\CrossAndTopSectionBig, every node/.style={transform shape}]
		\input{tikz_process_steps/pwell.a.tex}
% n-well
\fill[nwell] (1.25,0.75) rectangle (8.5,2);
\node at (5.75,1) {N-Well};
% p-well
\fill[pwell] (11.75,0.25) rectangle (18.75,2);
\node at (15.25,1) {P-Well};
% p-well
\fill[pwell] (11.5,0.75) rectangle (19,2);
\node at (14.25,1) {P-Well};
	\end{tikzpicture}
	\begin{tikzpicture}[node distance = 3cm, auto, thick,scale=\CrossAndTopSectionBig, every node/.style={transform shape}]
		\input{tikz_process_steps/sti.b.tex}
\fill[nwell] (1.25,1) rectangle (8.25,7.25);
\fill[pwell] (11.5,1.25) rectangle (19,7.5);
	\end{tikzpicture}
	\caption{P-well target geometry}
	\label{pwell_target}
\end{figure}
The P-well will serve us as an island of higher p-doped substrate within the slightly p-doped basis substrate.

The dopant dose will be $2.5\times10^{12}cm^{-2}$ as calculated in the documentation of the process design leading to these steps\footnote{\url{https://github.com/leviathanch/libresiliconprocess/raw/master/process_design/process_design.pdf}}.

\begin{figure}[H]
	\centering
	\begin{tikzpicture}[node distance =1cm, auto, thick,scale=\VLSILayout, every node/.style={transform shape}]
		\fill[Goldenrod,opacity=0.2] (0.75,0.5) rectangle (8.75,7.75);
\fill[Goldenrod,opacity=0.2] (11.25,0.5) rectangle (19.25,7.75);

\draw[dotted] (20.5,0.5) rectangle (25,5.5);

\node at (22.25,5) {\textbf{Layers}};

\fill[Goldenrod,opacity=0.2] (21,1) rectangle (21.5,1.5);
\node at (22.25,1.25) {active};

\fill[orange,opacity=0.2] (21,1.5) rectangle (21.5,2);
\node at (22.25,1.75) {nwell};

\fill[blue,opacity=0.2] (21,2) rectangle (21.5,2.5);
\node at (22.25,2.25) {nimplant};

\fill[red,opacity=0.2] (21,2.5) rectangle (21.5,3);
\node at (22.25,2.75) {pimplant};

\fill[Emerald,opacity=0.2] (21,3) rectangle (21.5,3.5);
\node at (22.25,3.25) {gate};

\fill[Fuchsia,opacity=0.2] (21,3.5) rectangle (21.5,4);
\node at (22.25,3.75) {metal1};

\fill[DarkOrchid,opacity=0.2] (21,4) rectangle (21.5,4.5);
\node at (22.25,4.25) {via1};
\fill[pwell,opacity=\OpacityLayout] (11.5,0.75) rectangle (19,7.5);
	\end{tikzpicture}
	\caption{P-Well layout}
	\label{pwell_layout}
\end{figure}

In \autoref{pwell_layout} the layout of the P-well region on top of the active area region can be seen.

The p-well is being fit into the active area.

It should even be a little bit bigger than the active area, because of possible alignment offsets.

The layout is being automatically generated for GDS2 based on cifoutput rules, so you just have to draw you well.

\newpage

\subsection{Patterning}

The resist is being deposited spray or spin coating (spray coating is better because of the uneven surface!) and then soft baked depending on the baking time for the specific resist.
The layout for being exposed onto the resist is being extracted from the "pwell" layer within the GDS2 file onto a \textbf{bright field} mask.
The requirement is a \textbf{negative} tone resist.

\begin{figure}[H]
	\centering
	\begin{tikzpicture}[node distance = 3cm, auto, thick,scale=\CrossAndTopSection, every node/.style={transform shape}]
		% substrate
\fill[substrate] (0,0) rectangle (20,1.25);
\node at (2,0.5) {Silicon substrate};
\fill[substrate] (0.25,1.25) rectangle (19.75,2);


	\end{tikzpicture}
	\begin{tikzpicture}[node distance = 3cm, auto, thick,scale=\CrossAndTopSection, every node/.style={transform shape}]
		\input{tikz_process_steps/pwell.patterning.at.tex}
	\end{tikzpicture}
	\drawStepArrow{Mask: pwell}
	\begin{tikzpicture}[node distance = 3cm, auto, thick,scale=\CrossAndTopSection, every node/.style={transform shape}]
		% resist
\fill[resist] (0.25,2.0) rectangle (11.5,5.0);
\fill[resist] (19,2.0) rectangle (19.75,5.0);

\input{tikz_process_steps/basic.a.tex}


	\end{tikzpicture}
	\begin{tikzpicture}[node distance = 3cm, auto, thick,scale=\CrossAndTopSection, every node/.style={transform shape}]
		\input{tikz_process_steps/pwell.patterning.bt.tex}
	\end{tikzpicture}
	\caption{Cross/top view of P-well layout on resist}
\end{figure}
The thickness of the resist layer and the baking duration will variate depending on the specific equipment for which this process will be implemented with.
Also after the exposure and development, the hard baking shouldn't be forgotten!

\subsection{Implantation}\label{pwell_implant_step}
We now need to inject the carriers into the upper level of the n-channel area so that we can later on drive them into the crystal during the drive-in step.

\begin{figure}[H]
	\centering
	\begin{tikzpicture}[node distance = 3cm, auto, thick,scale=\CrossSectionOnly, every node/.style={transform shape}]
		\input{tikz_process_steps/basic.a.tex}
% boron
\shade[upper left = pwell, upper right = pwell, lower right = substrate, lower left = substrate,] (11.5,1.5) rectangle (19.0,2);


\forloop{ct}{0}{\value{ct} < 21}
{
	\draw [->] (\value{ct},5) -- (\value{ct},4);
	\node at (\value{ct},5.2) {B$^{11}$};
}
	\end{tikzpicture}
	\drawStepArrow{Boron implant}
	\begin{tikzpicture}[node distance = 3cm, auto, thick,scale=\CrossSectionOnly, every node/.style={transform shape}]
		% oxide
\fill[isolationoxide] (0,1.25) rectangle (11.5,2.75);
\fill[isolationoxide] (19,1.25) rectangle (20,2.75);

% oxide hill 1
\fill[isolationoxide] (1.25,2.75) rectangle (8.25,3.5);
\filldraw[line width=0, isolationoxide] (0.5,2.75) -- (1.25,2.75) -- (1.25,3.5);
\filldraw[line width=0, isolationoxide] (8.25,2.75) -- (8.25,3.5) -- (9.0,2.75);

% oxide hill 2
\filldraw[line width=0, isolationoxide] (11.25,2.75) -- (11.5,2.75) -- (11.5,3.0);
\filldraw[line width=0, isolationoxide] (19.0,3.0)  -- (19.0,2.75) -- (19.25,2.75);

\node at (2,2.1) {SiO2};

\input{tikz_process_steps/pwell.mask_dioxide_layer.a.tex}

% boron
\fill[pwell] (11.75,1.8) rectangle (18.75,2);
	\end{tikzpicture} \\
	\caption{Doping process}
\end{figure}

\textbf{Possible approaches}:
\begin{itemize}
	\item \textbf{"CF-3000 Implanter (IMP-3000)" from HKUST} \\
	At HKUST we have an implanter which gives us better control over the initial surface concentration. \\
	These steps are needed to arrive with the desired geometry:
	The P-well is implanted with a Boron ($B^{11}$) dose of $2.5\times10^{12}cm^{-2}$ at an energy of 100 keV
\end{itemize}

\newpage

\subsection{Resist strip}

In order to avoid contamination of the machines we need to make sure all the resist has been stripped off from the wafer.

\begin{figure}[H]
	\centering
	\begin{tikzpicture}[node distance = 3cm, auto, thick,scale=\CrossSectionOnly, every node/.style={transform shape}]
		% oxide
\fill[isolationoxide] (0,1.25) rectangle (11.5,2.75);
\fill[isolationoxide] (19,1.25) rectangle (20,2.75);

% oxide hill 1
\fill[isolationoxide] (1.25,2.75) rectangle (8.25,3.5);
\filldraw[line width=0, isolationoxide] (0.5,2.75) -- (1.25,2.75) -- (1.25,3.5);
\filldraw[line width=0, isolationoxide] (8.25,2.75) -- (8.25,3.5) -- (9.0,2.75);

% oxide hill 2
\filldraw[line width=0, isolationoxide] (11.25,2.75) -- (11.5,2.75) -- (11.5,3.0);
\filldraw[line width=0, isolationoxide] (19.0,3.0)  -- (19.0,2.75) -- (19.25,2.75);

\node at (2,2.1) {SiO2};

\input{tikz_process_steps/pwell.mask_dioxide_layer.a.tex}

% boron
\fill[pwell] (11.75,1.8) rectangle (18.75,2);

	\end{tikzpicture}
	\drawStepArrow{}
	\begin{tikzpicture}[node distance = 3cm, auto, thick,scale=\CrossSectionOnly, every node/.style={transform shape}]
		% substrate
\fill[substrate] (0,0) rectangle (20,1.25);
\node at (2,0.5) {Silicon substrate};
\fill[substrate] (0.25,1.25) rectangle (19.75,2);

% boron
\shade[upper left = pwell, upper right = pwell, lower right = substrate, lower left = substrate,] (11.5,1.5) rectangle (19.0,2);

	\end{tikzpicture}
	\caption{Resist removal}
\end{figure}
Please just use the solvent for the specific resist.

\newpage


\newpage
\subsection{P-base}\label{pbase_chapter}

In order to build BiCMOS on the same substrate, a nested P-well within the N-well (now it's twin well) is required for building the bijunction transistors.

The cross section as well as the top view of the targeted geometry are shown in \autoref{pbase_target}

\begin{figure}[H]
	\centering
	\begin{tikzpicture}[node distance = 3cm, auto, thick,scale=\CrossAndTopSectionBig, every node/.style={transform shape}]
		% substrate
\fill[substrate] (0,0) rectangle (20,2);
\node at (2,0.5) {Silicon substrate};
%trenches
\fill[isolationoxide] (0,0.75) rectangle (1,2);
\fill[isolationoxide] (8.5,0.75) rectangle (11.5,2);
\fill[isolationoxide] (19,0.75) rectangle (20,2);
\paintnwells{3.5}
\paintpwells{3.5}
\paintpbases{1.0}

	\end{tikzpicture}
	\caption{P-base cross section}
	\label{pbase_target}
\end{figure}

The P-base will serve us as an island of higher P-doped substrate within the slightly N-well basis substrate, which will result in a isolated area by forming PN junction versus PN junction.

The dopant dose will be $1.93\times10^{12}cm^{-2}$ at 40 keV, as calculated in the documentation of the process design leading to these steps\footnote{\url{https://github.com/leviathanch/libresiliconprocess/raw/master/process_design/process_design.pdf}}.

The P-base can very well cover the N-well area since the expansion mostly is vertical, but it should be kept in mind, that there is also a lateral diffusion when placing contacts also on N-well around the P-base.

The layout is being automatically generated for GDS2 based on cifoutput rules, so you just have to draw your P-base.

After the implantation we perform a drive-in in inert atmosphere at $1050\degreesC$ for one hour.

\subsection{N-base}\label{nbase_chapter}
In order to build BiCMOS on the same substrate, another N-well within the P-Base (tripple well!) is required for building the complementary isolated P-channel transistor for a n-p-channel logic circuitry as shown above in the example section.

The cross section as well as the top view of the targeted geometry are shown in \autoref{nbase_target}

\begin{figure}[H]
	\centering
	\begin{tikzpicture}[node distance = 3cm, auto, thick,scale=\CrossAndTopSectionBig, every node/.style={transform shape}]
		% substrate
\fill[substrate] (0,0) rectangle (20,1.25);
\node at (2,0.5) {Silicon substrate};
\fill[substrate] (0.25,1.25) rectangle (19.75,2);

\def\welldepthAnbase{1.25}
\def\welldepthBnbase{1.5}
\def\welldepthCnbase{1.75}

% normal wells
\shade[upper left = nwell, upper right = nwell, lower right = substrate, lower left = substrate,] (1.25,\welldepthAnbase) rectangle (8.25,2.0);
\shade[upper left = pwell, upper right = pwell, lower right = substrate, lower left = substrate,] (9.75,\welldepthAnbase) rectangle (16.75,2.0);
\shade[upper left = nwell, upper right = nwell, lower right = substrate, lower left = substrate,] (18.25,\welldepthAnbase) rectangle (25.25,2.0);
\shade[upper left = nwell, upper right = nwell, lower right = substrate, lower left = substrate,] (26.75,\welldepthAnbase) rectangle (33.75,2.0);
\shade[upper left = nwell, upper right = nwell, lower right = substrate, lower left = substrate,] (35.25,\welldepthAnbase) rectangle (42.25,2.0);

% isolation nwell pwell in nwell for SONOS flash
\shade[upper left = pbase, upper right = pbase, lower right = nwell, lower left = nwell,] (18.5,\welldepthBnbase) rectangle (25.0,2.0);

% npn - pbase
\shade[upper left = pbase, upper right = pbase, lower right = nwell, lower left = nwell,] (28.25,\welldepthBnbase) rectangle (32.0,2.0);

% pnp emitter and collector ring
\shade[upper left = pbase, upper right = pbase, lower right = nwell, lower left = nwell,] (35.5,\welldepthBnbase) rectangle (37.0,2.0);
\shade[upper left = pbase, upper right = pbase, lower right = nwell, lower left = nwell,] (38.0,\welldepthBnbase) rectangle (39.5,2.0);
\shade[upper left = pbase, upper right = pbase, lower right = nwell, lower left = nwell,] (40.5,\welldepthBnbase) rectangle (42.0,2.0);

% n base for SONOS
\shade[upper left = nbase, upper right = nbase, lower right = pbase, lower left = pbase,] (18.75,\welldepthCnbase) rectangle (24.75,2.0);

% n base for NPN emitter
\shade[upper left = nbase, upper right = nbase, lower right = pbase, lower left = pbase,] (29.5,\welldepthCnbase) rectangle (30.75,2.0);

	\end{tikzpicture}
	\caption{N-base target geometry}
	\label{nbase_target}
\end{figure}

The N-well will serve us as an island of N-doped substrate within the P-doped basis substrate.

The dopant dose will be $2.33\times10^{12}cm^{-2}$ at 70 keV.


