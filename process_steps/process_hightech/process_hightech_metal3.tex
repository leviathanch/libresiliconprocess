\subsection{Metal 3}\label{chapter_metal3}

Now we've got to build the more interconnect wires, connecting the "metal2" to the "metal3" wires,
which will provide us the contact pads exposed by the glass layer.

This surface will not be covered any further and will be the surface where we touch down with the test
probes and bond our wires onto.

\begin{figure}[H]
	\centering
	\begin{tikzpicture}[node distance = 3cm, auto, thick,scale=\CrossAndTopSectionBig, every node/.style={transform shape}]
		\fill[isolationoxide] (0.0,\LowerMoreMetal) rectangle (55.0,\LowerMoreMetalTwo);

\input{tikz_process_steps/via1.a.tex}

\fill[nitride] (0,\LowerMoreMetal) rectangle (55,\LowerMoreMetal+0.5);
\fill[isolationoxide] (0,\LowerMoreMetal) rectangle (55,\LowerMoreMetal+0.25);

\paintscaledvias{nitride}{\LowerMoreMetal+0.25}{\UpperMoreMetal+0.5}{0.75}
\paintscaledvias{isolationoxide}{\LowerMoreMetal}{\UpperMoreMetal+0.25}{0.5}

\paintscaledvias{gray}{\LowerMoreMetal}{\UpperMoreMetal}{0.25}
\paintscaledvias{brown}{\LowerMoreMetal+0.3}{\UpperMoreMetal}{0.25}

\paintscaledvias{gray}{\UpperMetal}{\LowerMoreMetal}{0.0}
\paintscaledvias{brown}{\UpperMetal+0.2}{\UpperMoreMetal}{-0.2}



\paintscaledvias{white}{\UpperMoreMetal}{\LowerMoreMetalTwo}{0}


\fill[nitride] (0,\LowerMoreMetalTwo) rectangle (55,\LowerMoreMetalTwo+0.5);
\fill[isolationoxide] (0,\LowerMoreMetalTwo) rectangle (55,\LowerMoreMetalTwo+0.25);

\paintscaledvias{nitride}{\LowerMoreMetalTwo+0.25}{\UpperMoreMetalTwo+0.5}{1.00}
\paintscaledvias{isolationoxide}{\LowerMoreMetalTwo}{\UpperMoreMetalTwo+0.25}{0.75}

\paintscaledvias{gray}{\UpperMoreMetal}{\LowerMoreMetalTwo}{0.25}
\paintscaledvias{gray}{\LowerMoreMetalTwo}{\UpperMoreMetalTwo}{0.50}

\paintscaledvias{brown}{\UpperMoreMetal+0.25}{\LowerMoreMetalTwo+0.3}{-0.1}
\paintscaledvias{brown}{\LowerMoreMetalTwo+0.3}{\UpperMoreMetalTwo}{0.5}

	\end{tikzpicture}
	\caption{Metal geometry target}
	\label{metal3_target}
\end{figure}

As can be seen in \autoref{metal3_target}, the goal of this step is purely to etch the wire structure for the additional
metal layer into the in \autoref{metal2_target} deposited metal layer, and form wires by doing so.

In a later iterations of this process we might be switching to Copper as the metal3 material for this step which
will result in a variation of this step because the usage of damascene method.

For now first we sputter 100nm Aluminum and then around 50nm Nickel for passivation, all in the same vacuum.
