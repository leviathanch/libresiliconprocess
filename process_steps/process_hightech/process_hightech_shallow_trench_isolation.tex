\section{Shallow trench isolation}\label{sti_chapter}
The geometry of a substrate with STI implemented can be seen in \autoref{sti_target}.

\begin{figure}[H]
	\centering
	\begin{tikzpicture}[node distance = 3cm, auto, thick,scale=\CrossAndTopSectionBig, every node/.style={transform shape}]
		% substrate
\fill[substrate] (0,0) rectangle (20,2);
\node at (2,0.5) {Silicon substrate};
%trenches
\fill[isolationoxide] (0,0.75) rectangle (1,2);
\fill[isolationoxide] (8.5,0.75) rectangle (11.5,2);
\fill[isolationoxide] (19,0.75) rectangle (20,2);
	\end{tikzpicture}
	\begin{tikzpicture}[node distance = 3cm, auto, thick,scale=\CrossAndTopSectionBig, every node/.style={transform shape}]
		% substrate
\fill[YellowOrange] (0,0) rectangle (20,12);
% trench area
\fill[DarkGray] (0,0) rectangle (1,12);
\fill[DarkGray] (8.5,0) rectangle (11.5,12);
\fill[DarkGray] (19,0) rectangle (20,12);
\fill[DarkGray] (0,0) rectangle (20,1.25);
\fill[DarkGray] (0,7.5) rectangle (20,12);
	\end{tikzpicture}
	\caption{Shallow trench isolation target geometry}
	\label{sti_target}
\end{figure}

As can be seen in \autoref{tripple_well_target}, the STI trench is supposed to have approximately half the depth of the N-well.

Because the N-well will be $\approx 4 \mu m$ in depth, so we have to match this with our trench depth.

I order to allow a sufficiently low resistance of the ESD diode but at the same time a sufficient isolation of between the standard cells a trade-off has been done.

The targeted depth of the box isolation is $\approx 2 \mu m$.

The STI area will be everywhere, where no well areas are.

We use a dry etching method for cutting into the silicon substrate and making the active area become islands with trenches in between.

After that we fill the trenches with LTO and polish the wafer until the LTO surface and the silicon island surface are sufficiently on the same level.

Our minimum width and height as well as the space between the active areas comes from the line space constrain of the silicon etcher and of course the optical limitations of the stepper which are as well 0.5\um.

\newpage

\subsection{CMP end stop}\label{sti_end_stop}

In order to prevent irreversible damage to the crystal lattice of the active area, we need to provide a CMP end stop on top of those areas.

\begin{figure}[H]
	\centering
	\begin{tikzpicture}[node distance = 3cm, auto, thick,scale=\CrossSectionOnly, every node/.style={transform shape}]
		\fill[substrate] (0,0) rectangle (55,\STIIslandSurface);
\node at (2,0.5) {Silicon substrate};

	\end{tikzpicture}
	\drawStepArrow{}
	\begin{tikzpicture}[node distance = 3cm, auto, thick,scale=\CrossSectionOnly, every node/.style={transform shape}]
		\fill[substrate] (0,0) rectangle (55,\STIIslandSurface);
\node at (2,0.5) {Silicon substrate};

\fill[isolationoxide] (0,\STIIslandSurface) rectangle (55,\STIIslandSurface+0.25);
\fill[nitride] (0,\STIIslandSurface+0.25) rectangle (55,\STIIslandSurface+1.0);

	\end{tikzpicture}
	\caption{End stop}
\end{figure}

The wafer is cleaned by being put into sulfuric acid at 120\degreesC and afterwards HF dipped in order to remove the native oxide.
After that a more uniform and very thin film of thermal oxide is being grown, the thickness can be around 10nm, this can be achieved by putting the wafer into a furnace at 1000\degreesC for 15 minutes in an $O_2$ environment (dry oxidation).
On top of this pad oxide, a layer of around 100nm of nitride is being deposited, using chemical vapor deposition.

\subsection{Silicon etching}\label{sti_trench_etch}

The trench depth has to be at least 2 microns and less than 4 microns deep, in order to have a sufficiently good isolation for preventing latchup effects and at the same time still good enough ESD diode behaviour.

\begin{figure}[H]
	\centering
	\begin{tikzpicture}[node distance = 3cm, auto, thick,scale=\CrossSectionOnly, every node/.style={transform shape}]
		\input{tikz_process_steps/pwell.a.tex}
% n-well
\fill[nwell] (1.25,0.75) rectangle (8.5,2);
\node at (5.75,1) {N-Well};
% p-well
\fill[pwell] (11.75,0.25) rectangle (18.75,2);
\node at (15.25,1) {P-Well};

% substrate islands
\fill[resist] (1.25,2.0) rectangle (8.25,4.0);
\fill[resist] (11.75,2.0) rectangle (18.75,4.0);


	\end{tikzpicture}
	\drawStepArrow{}
	\begin{tikzpicture}[node distance = 3cm, auto, thick,scale=\CrossSectionOnly, every node/.style={transform shape}]
		\input{tikz_process_steps/sti.silicon_etch.b.tex}
	\end{tikzpicture}
	\caption{Trench etching}
\end{figure}

After patterning the STI layout the resist is being hard baked and the nitride+pad oxide is being etched, using plasma etching for nitride+oxide.

After etching the nitride and oxide we use a DRIE etcher and set the number of cycles in a way that it results in around 2 microns trench depth.

Adding to the over etch from the previous etch step, this will result in a depth a little bit deeper than 2 microns.

\newpage

\subsection{Liner oxide}\label{sti_liner_oxide}

In order to improve the interface properties of the LTO deposited in \autoref{sti_lto_deposition} to the side walls of the silicon islands a thin layer of thermal oxide is being grown after DRIE etching.

\begin{figure}[H]
	\centering
	\begin{tikzpicture}[node distance = 3cm, auto, thick,scale=\CrossSectionOnly, every node/.style={transform shape}]
		% substrate
\fill[substrate] (0,0) rectangle (55,\trenchBottom);
\node at (2,0.5) {Silicon substrate};

% normal wells
\fill[substrate] (1.25,\trenchBottom) rectangle (8.25,\STIIslandSurface);
\fill[substrate] (9.75,\trenchBottom) rectangle (16.75,\STIIslandSurface);
\fill[substrate] (18.25,\trenchBottom) rectangle (25.25,\STIIslandSurface);
\fill[substrate] (26.75,\trenchBottom) rectangle (33.75,\STIIslandSurface);
\fill[substrate] (35.25,\trenchBottom) rectangle (42.25,\STIIslandSurface);

\fill[isolationoxide] ( 1.25,\STIIslandSurface)      rectangle ( 8.25,\STIIslandSurface+0.25);
\fill[nitride]        ( 1.25,\STIIslandSurface+0.25) rectangle ( 8.25,\STIIslandSurface+1.0);

\fill[isolationoxide] ( 9.75,\STIIslandSurface)      rectangle (16.75,\STIIslandSurface+0.25);
\fill[nitride]        ( 9.75,\STIIslandSurface+0.25) rectangle (16.75,\STIIslandSurface+1.0);

\fill[isolationoxide] (18.25,\STIIslandSurface)      rectangle (25.25,\STIIslandSurface+0.25);
\fill[nitride]        (18.25,\STIIslandSurface+0.25) rectangle (25.25,\STIIslandSurface+1.0);

\fill[isolationoxide] (26.75,\STIIslandSurface)      rectangle (33.75,\STIIslandSurface+0.25);
\fill[nitride]        (26.75,\STIIslandSurface+0.25) rectangle (33.75,\STIIslandSurface+1.0);

\fill[isolationoxide] (35.25,\STIIslandSurface)      rectangle (42.25,\STIIslandSurface+0.25);
\fill[nitride]        (35.25,\STIIslandSurface+0.25) rectangle (42.25,\STIIslandSurface+1.0);



	\end{tikzpicture}
	\drawStepArrow{}
	\begin{tikzpicture}[node distance = 3cm, auto, thick,scale=\CrossSectionOnly, every node/.style={transform shape}]
		\fill[isolationoxide] (1.00,\trenchBottom) rectangle (8.50,\STIIslandSurface+0.25);
\fill[isolationoxide] (9.50,\trenchBottom) rectangle (17.00,\STIIslandSurface+0.25);
\fill[isolationoxide] (18.00,\trenchBottom) rectangle (25.50,\STIIslandSurface+0.25);
\fill[isolationoxide] (26.50,\trenchBottom) rectangle (34.00,\STIIslandSurface+0.25);
\fill[isolationoxide] (35.00,\trenchBottom) rectangle (42.50,\STIIslandSurface+0.25);
\fill[isolationoxide] (0,0) rectangle (55.00,\trenchBottom+0.25);

% substrate
\fill[substrate] (0,0) rectangle (55,\trenchBottom);
\node at (2,0.5) {Silicon substrate};

% normal wells
\fill[substrate] (1.25,\trenchBottom) rectangle (8.25,\STIIslandSurface);
\fill[substrate] (9.75,\trenchBottom) rectangle (16.75,\STIIslandSurface);
\fill[substrate] (18.25,\trenchBottom) rectangle (25.25,\STIIslandSurface);
\fill[substrate] (26.75,\trenchBottom) rectangle (33.75,\STIIslandSurface);
\fill[substrate] (35.25,\trenchBottom) rectangle (42.25,\STIIslandSurface);

\fill[isolationoxide] ( 1.25,\STIIslandSurface)      rectangle ( 8.25,\STIIslandSurface+0.25);
\fill[nitride]        ( 1.25,\STIIslandSurface+0.25) rectangle ( 8.25,\STIIslandSurface+1.0);

\fill[isolationoxide] ( 9.75,\STIIslandSurface)      rectangle (16.75,\STIIslandSurface+0.25);
\fill[nitride]        ( 9.75,\STIIslandSurface+0.25) rectangle (16.75,\STIIslandSurface+1.0);

\fill[isolationoxide] (18.25,\STIIslandSurface)      rectangle (25.25,\STIIslandSurface+0.25);
\fill[nitride]        (18.25,\STIIslandSurface+0.25) rectangle (25.25,\STIIslandSurface+1.0);

\fill[isolationoxide] (26.75,\STIIslandSurface)      rectangle (33.75,\STIIslandSurface+0.25);
\fill[nitride]        (26.75,\STIIslandSurface+0.25) rectangle (33.75,\STIIslandSurface+1.0);

\fill[isolationoxide] (35.25,\STIIslandSurface)      rectangle (42.25,\STIIslandSurface+0.25);
\fill[nitride]        (35.25,\STIIslandSurface+0.25) rectangle (42.25,\STIIslandSurface+1.0);




	\end{tikzpicture}
	\caption{Liner oxide}
\end{figure}

The interface oxide, as the pad oxide, only has to be a few nanometers in thickness, this can be achieved by putting the wafer again into a furnace at 1000\degreesC for 15 minutes in an $O_2$ environment (dry oxidation).

\subsection{LTO deposition}\label{sti_lto_deposition}

Now we fill up the trenches we've etched before with LTO for further planarization in \autoref{sti_cmp_step}

\begin{figure}[H]
	\centering
	\begin{tikzpicture}[node distance = 3cm, auto, thick,scale=\CrossSectionOnly, every node/.style={transform shape}]
		\fill[isolationoxide] (1.00,\trenchBottom) rectangle (8.50,\STIIslandSurface+0.25);
\fill[isolationoxide] (9.50,\trenchBottom) rectangle (17.00,\STIIslandSurface+0.25);
\fill[isolationoxide] (18.00,\trenchBottom) rectangle (25.50,\STIIslandSurface+0.25);
\fill[isolationoxide] (26.50,\trenchBottom) rectangle (34.00,\STIIslandSurface+0.25);
\fill[isolationoxide] (35.00,\trenchBottom) rectangle (42.50,\STIIslandSurface+0.25);
\fill[isolationoxide] (0,0) rectangle (55.00,\trenchBottom+0.25);

% substrate
\fill[substrate] (0,0) rectangle (55,\trenchBottom);
\node at (2,0.5) {Silicon substrate};

% normal wells
\fill[substrate] (1.25,\trenchBottom) rectangle (8.25,\STIIslandSurface);
\fill[substrate] (9.75,\trenchBottom) rectangle (16.75,\STIIslandSurface);
\fill[substrate] (18.25,\trenchBottom) rectangle (25.25,\STIIslandSurface);
\fill[substrate] (26.75,\trenchBottom) rectangle (33.75,\STIIslandSurface);
\fill[substrate] (35.25,\trenchBottom) rectangle (42.25,\STIIslandSurface);

\fill[isolationoxide] ( 1.25,\STIIslandSurface)      rectangle ( 8.25,\STIIslandSurface+0.25);
\fill[nitride]        ( 1.25,\STIIslandSurface+0.25) rectangle ( 8.25,\STIIslandSurface+1.0);

\fill[isolationoxide] ( 9.75,\STIIslandSurface)      rectangle (16.75,\STIIslandSurface+0.25);
\fill[nitride]        ( 9.75,\STIIslandSurface+0.25) rectangle (16.75,\STIIslandSurface+1.0);

\fill[isolationoxide] (18.25,\STIIslandSurface)      rectangle (25.25,\STIIslandSurface+0.25);
\fill[nitride]        (18.25,\STIIslandSurface+0.25) rectangle (25.25,\STIIslandSurface+1.0);

\fill[isolationoxide] (26.75,\STIIslandSurface)      rectangle (33.75,\STIIslandSurface+0.25);
\fill[nitride]        (26.75,\STIIslandSurface+0.25) rectangle (33.75,\STIIslandSurface+1.0);

\fill[isolationoxide] (35.25,\STIIslandSurface)      rectangle (42.25,\STIIslandSurface+0.25);
\fill[nitride]        (35.25,\STIIslandSurface+0.25) rectangle (42.25,\STIIslandSurface+1.0);





	\end{tikzpicture}
	\drawStepArrow{CVD}
	\begin{tikzpicture}[node distance = 3cm, auto, thick,scale=\CrossSectionOnly, every node/.style={transform shape}]
		\fill[isolationoxide] (0.00,\trenchBottom) rectangle (55,\STIIslandSurface+0.25);
\fill[isolationoxide] (0.00,\trenchBottom) rectangle (47,\STIIslandSurface+4.5);
\fill[isolationoxide] (1.00,\trenchBottom) rectangle (8.50,\STIIslandSurface+0.25);
\fill[isolationoxide] (9.50,\trenchBottom) rectangle (17.00,\STIIslandSurface+0.25);
\fill[isolationoxide] (18.00,\trenchBottom) rectangle (25.50,\STIIslandSurface+0.25);
\fill[isolationoxide] (26.50,\trenchBottom) rectangle (34.00,\STIIslandSurface+0.25);
\fill[isolationoxide] (35.00,\trenchBottom) rectangle (42.50,\STIIslandSurface+0.25);
\fill[isolationoxide] (0,0) rectangle (55.00,\trenchBottom+0.25);

% substrate
\fill[substrate] (0,0) rectangle (55,\trenchBottom);
\node at (2,0.5) {Silicon substrate};

% normal wells
\fill[substrate] (1.25,\trenchBottom) rectangle (8.25,\STIIslandSurface);
\fill[substrate] (9.75,\trenchBottom) rectangle (16.75,\STIIslandSurface);
\fill[substrate] (18.25,\trenchBottom) rectangle (25.25,\STIIslandSurface);
\fill[substrate] (26.75,\trenchBottom) rectangle (33.75,\STIIslandSurface);
\fill[substrate] (35.25,\trenchBottom) rectangle (42.25,\STIIslandSurface);

\fill[isolationoxide] ( 1.25,\STIIslandSurface)      rectangle ( 8.25,\STIIslandSurface+0.25);
\fill[nitride]        ( 1.25,\STIIslandSurface+0.25) rectangle ( 8.25,\STIIslandSurface+1.0);

\fill[isolationoxide] ( 9.75,\STIIslandSurface)      rectangle (16.75,\STIIslandSurface+0.25);
\fill[nitride]        ( 9.75,\STIIslandSurface+0.25) rectangle (16.75,\STIIslandSurface+1.0);

\fill[isolationoxide] (18.25,\STIIslandSurface)      rectangle (25.25,\STIIslandSurface+0.25);
\fill[nitride]        (18.25,\STIIslandSurface+0.25) rectangle (25.25,\STIIslandSurface+1.0);

\fill[isolationoxide] (26.75,\STIIslandSurface)      rectangle (33.75,\STIIslandSurface+0.25);
\fill[nitride]        (26.75,\STIIslandSurface+0.25) rectangle (33.75,\STIIslandSurface+1.0);

\fill[isolationoxide] (35.25,\STIIslandSurface)      rectangle (42.25,\STIIslandSurface+0.25);
\fill[nitride]        (35.25,\STIIslandSurface+0.25) rectangle (42.25,\STIIslandSurface+1.0);





	\end{tikzpicture}
	\caption{Oxide deposition}
\end{figure}

The easiest method is to put the wafer into a CVD furnace in order to deposit around 2 microns of LTO.

Better uniformity of the LTO film can be achieved by getting the boat out after every deposited 500nm, rotating it 90 degrees and putting it back in for another deposition round.

Also remember to measure the thickness of the deposited LTO under a spectroscope, in order to calculate the approximate CMP time!

\newpage

\subsection{CMP step}\label{sti_cmp_step}

Now the LTO needs to be planarized until a sufficiently low height differential between the oxide surface and the silicon surface is being met.

\begin{figure}[H]
	\centering
	\begin{tikzpicture}[node distance = 3cm, auto, thick,scale=\CrossSectionOnly, every node/.style={transform shape}]
		\fill[isolationoxide] (0.00,\trenchBottom) rectangle (55,\STIIslandSurface+0.25);
\fill[isolationoxide] (0.00,\trenchBottom) rectangle (47,\STIIslandSurface+4.5);
\fill[isolationoxide] (1.00,\trenchBottom) rectangle (8.50,\STIIslandSurface+0.25);
\fill[isolationoxide] (9.50,\trenchBottom) rectangle (17.00,\STIIslandSurface+0.25);
\fill[isolationoxide] (18.00,\trenchBottom) rectangle (25.50,\STIIslandSurface+0.25);
\fill[isolationoxide] (26.50,\trenchBottom) rectangle (34.00,\STIIslandSurface+0.25);
\fill[isolationoxide] (35.00,\trenchBottom) rectangle (42.50,\STIIslandSurface+0.25);
\fill[isolationoxide] (0,0) rectangle (55.00,\trenchBottom+0.25);

\input{tikz_process_steps/sti.liner_oxide.a.tex}



	\end{tikzpicture}
	\drawStepArrow{CMP}
	\begin{tikzpicture}[node distance = 3cm, auto, thick,scale=\CrossSectionOnly, every node/.style={transform shape}]
		% substrate
\fill[isolationoxide] (0,0) rectangle (55,\STIIslandSurface);

\newcommand{\leftslope}[1]{
\filldraw[line width=0, isolationoxide] (#1-1.0,\STIIslandSurface) -- (#1,\STIIslandSurface) -- (#1,\STIIslandSurface+1.0);
}
\newcommand{\rightslope}[1]{
\filldraw[line width=0, isolationoxide] (#1,\STIIslandSurface) -- (#1+1.0,\STIIslandSurface) -- (#1,\STIIslandSurface+1.0);
}

\leftslope{1.25}
\leftslope{9.75}
\leftslope{18.25}
\leftslope{26.75}
\leftslope{35.25}

\rightslope{8.25}
\rightslope{16.75}
\rightslope{25.25}
\rightslope{33.75}
\rightslope{42.25}

% substrate
\fill[substrate] (0,0) rectangle (55,\trenchBottom);
\node at (2,0.5) {Silicon substrate};

% normal wells
\fill[substrate] (1.25,\trenchBottom) rectangle (8.25,\STIIslandSurface);
\fill[substrate] (9.75,\trenchBottom) rectangle (16.75,\STIIslandSurface);
\fill[substrate] (18.25,\trenchBottom) rectangle (25.25,\STIIslandSurface);
\fill[substrate] (26.75,\trenchBottom) rectangle (33.75,\STIIslandSurface);
\fill[substrate] (35.25,\trenchBottom) rectangle (42.25,\STIIslandSurface);

\fill[isolationoxide] ( 1.25,\STIIslandSurface)      rectangle ( 8.25,\STIIslandSurface+0.25);
\fill[nitride]        ( 1.25,\STIIslandSurface+0.25) rectangle ( 8.25,\STIIslandSurface+1.0);

\fill[isolationoxide] ( 9.75,\STIIslandSurface)      rectangle (16.75,\STIIslandSurface+0.25);
\fill[nitride]        ( 9.75,\STIIslandSurface+0.25) rectangle (16.75,\STIIslandSurface+1.0);

\fill[isolationoxide] (18.25,\STIIslandSurface)      rectangle (25.25,\STIIslandSurface+0.25);
\fill[nitride]        (18.25,\STIIslandSurface+0.25) rectangle (25.25,\STIIslandSurface+1.0);

\fill[isolationoxide] (26.75,\STIIslandSurface)      rectangle (33.75,\STIIslandSurface+0.25);
\fill[nitride]        (26.75,\STIIslandSurface+0.25) rectangle (33.75,\STIIslandSurface+1.0);

\fill[isolationoxide] (35.25,\STIIslandSurface)      rectangle (42.25,\STIIslandSurface+0.25);
\fill[nitride]        (35.25,\STIIslandSurface+0.25) rectangle (42.25,\STIIslandSurface+1.0);




\fill[isolationoxide] ( 1.25,\STIIslandSurface)      rectangle ( 8.25,\STIIslandSurface+0.25);
\fill[nitride]        ( 1.25,\STIIslandSurface+0.25) rectangle ( 8.25,\STIIslandSurface+1.0);

\fill[isolationoxide] ( 9.75,\STIIslandSurface)      rectangle (16.75,\STIIslandSurface+0.25);
\fill[nitride]        ( 9.75,\STIIslandSurface+0.25) rectangle (16.75,\STIIslandSurface+1.0);

\fill[isolationoxide] (18.25,\STIIslandSurface)      rectangle (25.25,\STIIslandSurface+0.25);
\fill[nitride]        (18.25,\STIIslandSurface+0.25) rectangle (25.25,\STIIslandSurface+1.0);

\fill[isolationoxide] (26.75,\STIIslandSurface)      rectangle (33.75,\STIIslandSurface+0.25);
\fill[nitride]        (26.75,\STIIslandSurface+0.25) rectangle (33.75,\STIIslandSurface+1.0);

\fill[isolationoxide] (35.25,\STIIslandSurface)      rectangle (42.25,\STIIslandSurface+0.25);
\fill[nitride]        (35.25,\STIIslandSurface+0.25) rectangle (42.25,\STIIslandSurface+1.0);

\draw[<->] (42.25,\STIIslandSurface+1.0) -- (42.25,\STIIslandSurface);
\node at (42.5,\STIIslandSurface+0.5) {$\Delta h$};


	\end{tikzpicture}
	\drawStepArrow{Nitride strip}
	\begin{tikzpicture}[node distance = 3cm, auto, thick,scale=\CrossSectionOnly, every node/.style={transform shape}]
		\input{tikz_process_steps/sti.cmp.c.tex}
	\end{tikzpicture}
	\caption{After CMP}
\end{figure}

A CMP is performed, based on a rough time calculation, based on the thickness measurement from \autoref{sti_lto_deposition}, until a height differential (See $\Delta h$ in graphics) below 200nm is being reached, which can be determined by using a surface profiler.

If available the slurry "SRS-985" should be used because it can significantly increase the yield by reducing the dishing as a study has found\footnote{\url{https://download.libresilicon.com/papers/10.1.1.567.8814.pdf}}.

After the planarization the wafer needs to be cleaned in hot ammonia and with RCA solution and wafers should kept wet in DI water after CMPing and should not dry out before being cleaned, because this would make particles get stuck in the oxide permanently and will destroy the sample.

After cleaning the LTO has to be annealed in order to increase the etching time when removing the pad oxide: The sample is being put into a furnace for 30 minutes at 850\degreesC in an inert atmosphere ($N_2$/$Ar$).

The densification is being performed before nitride strip, because the Phosphoric acid also attacks oxide, and by hardening the LTO first, we can reduce the amount of oxide, which is being removed alongside the nitride.

After the annealing the nitride is first stripped in a suitable nitride etchant like Phosphoric acid or the like.

After that the pad oxide beneath is being removed by being put into BOE for a few minutes until the pad oxide has been removed.

In this process, the sharp corners of the craters in the oxide, where the nitride used to be will also be smothened out, which is a nice side effect.

