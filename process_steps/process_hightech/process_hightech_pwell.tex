\subsection{P-well}\label{pwell_chapter}

In order to build CMOS on the same substrate, a P-well is required for building the complementary N-channel transistor for a NFET+PFET logic circuitry.

The cross section as well as the top view of the targeted geometry are shown in \autoref{nwell_target}

\begin{figure}[H]
	\centering
	\begin{tikzpicture}[node distance = 3cm, auto, thick,scale=\CrossAndTopSectionBig, every node/.style={transform shape}]
		\input{tikz_process_steps/nwell.a.tex}
% p-well
\fill[pwell] (11.5,0.75) rectangle (19,2);
\node at (14.25,1) {P-Well};
% n-well
\fill[nwell] (1.25,0.75) rectangle (8.5,2);
\node at (5.75,1) {N-Well};
% p-well
\fill[pwell] (11.75,0.25) rectangle (18.75,2);
\node at (15.25,1) {P-Well};
% p-well
\fill[pwell] (11.5,0.75) rectangle (19,2);
\node at (14.25,1) {P-Well};
	\end{tikzpicture}
	\begin{tikzpicture}[node distance = 3cm, auto, thick,scale=\CrossAndTopSectionBig, every node/.style={transform shape}]
		% substrate
\fill[YellowOrange] (0,0) rectangle (20,12);
% trench area
\fill[DarkGray] (0,0) rectangle (1,12);
\fill[DarkGray] (8.5,0) rectangle (11.5,12);
\fill[DarkGray] (19,0) rectangle (20,12);
\fill[DarkGray] (0,0) rectangle (20,1.25);
\fill[DarkGray] (0,7.5) rectangle (20,12);
\fill[nwell] (1.25,1) rectangle (8.25,7.25);
\fill[pwell] (11.5,1.25) rectangle (19,7.5);
	\end{tikzpicture}
	\caption{P-well target geometry}
	\label{pwell_target}
\end{figure}

The "P-well" will serve us as an island of higher p-doped substrate within the slightly p-doped basis substrate and gives us more flexibility with suppliers, because we can just adjust the doping in case the concentration might be different with another supplier.

The P-dopant concentration of our prime grade, p-type, single side polished, four inch wafers is between $8.76 \cdot 10^14 \frac{1}{cm^3}$ and $5.23 \cdot 10^14 \frac{1}{cm^3}$

This means we need a dose of $1.93\times10^{12}cm^{-2}$ Boron atoms at 40 keV, as calculated in the documentation of the process design leading to these steps\footnote{\url{https://github.com/leviathanch/libresiliconprocess/raw/master/process_design/process_design.pdf}}.

The concentration will need adjustment when the used substrate has different properties!

\begin{figure}[H]
	\centering
	\begin{tikzpicture}[node distance =1cm, auto, thick,scale=\VLSILayout, every node/.style={transform shape}]
		\fill[Goldenrod,opacity=0.2] (0.75,0.5) rectangle (8.75,7.75);
\fill[Goldenrod,opacity=0.2] (11.25,0.5) rectangle (19.25,7.75);

\draw[dotted] (20.5,0.5) rectangle (25,5.5);

\node at (22.25,5) {\textbf{Layers}};

\fill[Goldenrod,opacity=0.2] (21,1) rectangle (21.5,1.5);
\node at (22.25,1.25) {active};

\fill[orange,opacity=0.2] (21,1.5) rectangle (21.5,2);
\node at (22.25,1.75) {nwell};

\fill[blue,opacity=0.2] (21,2) rectangle (21.5,2.5);
\node at (22.25,2.25) {nimplant};

\fill[red,opacity=0.2] (21,2.5) rectangle (21.5,3);
\node at (22.25,2.75) {pimplant};

\fill[Emerald,opacity=0.2] (21,3) rectangle (21.5,3.5);
\node at (22.25,3.25) {gate};

\fill[Fuchsia,opacity=0.2] (21,3.5) rectangle (21.5,4);
\node at (22.25,3.75) {metal1};

\fill[DarkOrchid,opacity=0.2] (21,4) rectangle (21.5,4.5);
\node at (22.25,4.25) {via1};
\fill[pwell,opacity=\OpacityLayout] (11.5,0.75) rectangle (19,7.5);
	\end{tikzpicture}
	\caption{P-Well layout}
	\label{pwell_layout}
\end{figure}

In \autoref{pwell_layout} the layout of the P-well region on top of the active area region can be seen.

The p-well is being fit into the active area.

It should even be a little bit bigger than the active area, because of possible alignment offsets.

The layout is being automatically generated for GDS2 based on cifoutput rules, so you just have to draw you well.

After the implantation we perform a drive-in in inert atmosphere at $1050\degreesC$ for two hours and don't have to worry about the substrate anymore.
