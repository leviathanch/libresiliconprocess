\section{SONOS}\label{sonos_chapter}

Before we can construct the gate, we have to put some pads of oxide-nitride-oxide sandwich roughly in the area where the SONOS (silicon oxide nitride oxide silicon) gates will be located.

Later on, during etching of the polysilicon gates, the SONOS gate oxide sandwich will be automatically aligned with the gate because excess SONOS sandwich material will be etched away with the normal gate oxide.

\begin{figure}[H]
	\centering
	\begin{tikzpicture}[node distance = 3cm, auto, thick,scale=\CrossAndTopSectionBig, every node/.style={transform shape}]
		\def\welldepthAfox{0.75}
\def\welldepthBfox{1.25}
\def\welldepthCfox{1.5}

\newcommand{\stopper}[1]{
\filldraw[line width=0, isolationoxide] (#1,2.0) -- (#1+0.75,2.0) -- (#1+0.75,2.75);
\fill[isolationoxide] (#1+0.75,2.0) rectangle (#1+1.25,2.75);
\filldraw[line width=0, isolationoxide] (#1+1.25,2.75) -- (#1+1.25,2.0) -- (#1+2.0,2.0);
}

\newcommand{\bjtstopper}[1]{
\filldraw[line width=0, isolationoxide] (#1,2.0) -- (#1+0.25,2.0) -- (#1+0.25,2.75);
\fill[isolationoxide] (#1+0.25,2.0) rectangle (#1+0.5,2.75);
\filldraw[line width=0, isolationoxide] (#1+0.5,2.75) -- (#1+0.5,2.0) -- (#1+0.75,2.0);
}

% oxide
\fill[isolationoxide] (0,1.25) rectangle (55.0,2.0);

\fill[isolationoxide] (0,2.0) rectangle (0.75,2.75);
\filldraw[line width=0, isolationoxide] (0.75,2.75) -- (0.75,2.0) -- (1.5,2.0);

%separates n+ from p+
\stopper{2.5}
\stopper{8.0}

%separates n+ from p+
\stopper{13.5}

\stopper{16.5}
\stopper{17.0}

\stopper{20.0}

\stopper{24.5}
\stopper{25.0}

\bjtstopper{27.75}
\bjtstopper{29.0}
\bjtstopper{30.5}
\bjtstopper{31.75}

\stopper{33.5}
\stopper{33.75}

\bjtstopper{36.5}
\bjtstopper{37.75}
\bjtstopper{39.0}
\bjtstopper{40.25}

\filldraw[line width=0, isolationoxide] (41.75,2.0) -- (42.5,2.0) -- (42.5,2.75);
\fill[isolationoxide] (42.5,2.0) rectangle (55.0,2.75);

% substrate
\fill[substrate] (0,0) rectangle (20,2);
\node at (2,0.5) {Silicon substrate};
%trenches
\fill[isolationoxide] (0,0.75) rectangle (1,2);
\fill[isolationoxide] (8.5,0.75) rectangle (11.5,2);
\fill[isolationoxide] (19,0.75) rectangle (20,2);

% substrate
\fill[substrate] (0,0) rectangle (55,1.25);
\node at (2,0.5) {Silicon substrate};

% normal wells
\shade[upper left = nwell, upper right = nwell, lower right = substrate, lower left = substrate,] (1.25,\welldepthAfox) rectangle (8.25,2.0);
\shade[upper left = pwell, upper right = pwell, lower right = substrate, lower left = substrate,] (9.75,\welldepthAfox) rectangle (16.75,2.0);
\shade[upper left = nwell, upper right = nwell, lower right = substrate, lower left = substrate,] (18.25,\welldepthAfox) rectangle (25.25,2.0);
\shade[upper left = nwell, upper right = nwell, lower right = substrate, lower left = substrate,] (26.75,\welldepthAfox) rectangle (33.75,2.0);
\shade[upper left = nwell, upper right = nwell, lower right = substrate, lower left = substrate,] (35.25,\welldepthAfox) rectangle (42.25,2.0);

% p base
\shade[upper left = pbase, upper right = pbase, lower right = nwell, lower left = nwell,] (18.5,\welldepthBfox) rectangle (25.0,2.0);

% npn - pbase
\shade[upper left = pbase, upper right = pbase, lower right = nwell, lower left = nwell,] (28.25,\welldepthBfox) rectangle (32.0,2.0);

% pnp emitter and collector ring
\shade[upper left = pbase, upper right = pbase, lower right = nwell, lower left = nwell,] (35.5,\welldepthBfox) rectangle (37.0,2.0);
\shade[upper left = pbase, upper right = pbase, lower right = nwell, lower left = nwell,] (38.0,\welldepthBfox) rectangle (39.5,2.0);
\shade[upper left = pbase, upper right = pbase, lower right = nwell, lower left = nwell,] (40.5,\welldepthBfox) rectangle (42.0,2.0);

% n base for SONOS
\shade[upper left = nbase, upper right = nbase, lower right = pbase, lower left = pbase,] (18.75,\welldepthCfox) rectangle (24.75,2.0);

% n base for NPN emitter
\shade[upper left = nbase, upper right = nbase, lower right = pbase, lower left = pbase,] (29.5,\welldepthCfox) rectangle (30.75,2.0);



\fill[gateoxide] (22.6,\STIIslandSurface) rectangle (23.9,\SONOStopONE);
\fill[nitride] (22.6,\SONOStopONE) rectangle (23.9,\SONOStopTWO);
\fill[gateoxide] (22.6,\SONOStopTWO) rectangle (23.9,\SONOStopTHREE);

	\end{tikzpicture}
	\caption{SONOS sandwich pad}
\end{figure}

The line spacing of the SONOS shape has to be at least 0.5\um because of the resolution of the stepper and also because of the etching process which has 0.5\um as the minimum line spacing.

Also there has to be at least one lambda on each site to compensate for offsets, so the SONOS mask is bloated by 0.5\um.

The SONOS gate oxide is comprised of a stack of oxide covered with nitride covered with oxide.

The upper and lower oxide layers are the so called tunnel oxides which allow electrons during the programming phase to be tunneled into the nitride, where they get trapped and shift the threshold
voltage of the transistor. This way information can be stored and erased by tunelling the electrons back out of the nitride.

This ONO (oxide nitride oxide) pad will prevent the additional thin oxide from forming during the gate oxide formation in \autoref{step_growing_gate_oxide} and instead the LTO under and above the nitride
will be densified during the process step.

\newpage

\subsection{Lower oxide deposition}\label{step_depositing_sonos_lower_lto}

Now we have to deposit the lower part of the SONOS sandwich by depositing LTO.

As designed in the process design document, the layer will be around 5nm thick.

\begin{figure}[H]
	\centering
	\begin{tikzpicture}[node distance = 3cm, auto, thick,scale=\CrossSectionOnly, every node/.style={transform shape}]
		\input{tikz_process_steps/sonos.lto1_depositon.a.tex}
	\end{tikzpicture}
	\drawStepArrow{LTO deposition}
	\begin{tikzpicture}[node distance = 3cm, auto, thick,scale=\CrossSectionOnly, every node/.style={transform shape}]
		\coveringlayer{gateoxide}{0.2}

\def\welldepthAfox{0.75}
\def\welldepthBfox{1.25}
\def\welldepthCfox{1.5}

\newcommand{\stopper}[1]{
\filldraw[line width=0, isolationoxide] (#1,2.0) -- (#1+0.75,2.0) -- (#1+0.75,2.75);
\fill[isolationoxide] (#1+0.75,2.0) rectangle (#1+1.25,2.75);
\filldraw[line width=0, isolationoxide] (#1+1.25,2.75) -- (#1+1.25,2.0) -- (#1+2.0,2.0);
}

\newcommand{\bjtstopper}[1]{
\filldraw[line width=0, isolationoxide] (#1,2.0) -- (#1+0.25,2.0) -- (#1+0.25,2.75);
\fill[isolationoxide] (#1+0.25,2.0) rectangle (#1+0.5,2.75);
\filldraw[line width=0, isolationoxide] (#1+0.5,2.75) -- (#1+0.5,2.0) -- (#1+0.75,2.0);
}

% oxide
\fill[isolationoxide] (0,1.25) rectangle (55.0,2.0);

\fill[isolationoxide] (0,2.0) rectangle (0.75,2.75);
\filldraw[line width=0, isolationoxide] (0.75,2.75) -- (0.75,2.0) -- (1.5,2.0);

%separates n+ from p+
\stopper{2.5}
\stopper{8.0}

%separates n+ from p+
\stopper{13.5}

\stopper{16.5}
\stopper{17.0}

\stopper{20.0}

\stopper{24.5}
\stopper{25.0}

\bjtstopper{27.75}
\bjtstopper{29.0}
\bjtstopper{30.5}
\bjtstopper{31.75}

\stopper{33.5}
\stopper{33.75}

\bjtstopper{36.5}
\bjtstopper{37.75}
\bjtstopper{39.0}
\bjtstopper{40.25}

\filldraw[line width=0, isolationoxide] (41.75,2.0) -- (42.5,2.0) -- (42.5,2.75);
\fill[isolationoxide] (42.5,2.0) rectangle (55.0,2.75);

% substrate
\fill[substrate] (0,0) rectangle (20,2);
\node at (2,0.5) {Silicon substrate};
%trenches
\fill[isolationoxide] (0,0.75) rectangle (1,2);
\fill[isolationoxide] (8.5,0.75) rectangle (11.5,2);
\fill[isolationoxide] (19,0.75) rectangle (20,2);

% substrate
\fill[substrate] (0,0) rectangle (55,1.25);
\node at (2,0.5) {Silicon substrate};

% normal wells
\shade[upper left = nwell, upper right = nwell, lower right = substrate, lower left = substrate,] (1.25,\welldepthAfox) rectangle (8.25,2.0);
\shade[upper left = pwell, upper right = pwell, lower right = substrate, lower left = substrate,] (9.75,\welldepthAfox) rectangle (16.75,2.0);
\shade[upper left = nwell, upper right = nwell, lower right = substrate, lower left = substrate,] (18.25,\welldepthAfox) rectangle (25.25,2.0);
\shade[upper left = nwell, upper right = nwell, lower right = substrate, lower left = substrate,] (26.75,\welldepthAfox) rectangle (33.75,2.0);
\shade[upper left = nwell, upper right = nwell, lower right = substrate, lower left = substrate,] (35.25,\welldepthAfox) rectangle (42.25,2.0);

% p base
\shade[upper left = pbase, upper right = pbase, lower right = nwell, lower left = nwell,] (18.5,\welldepthBfox) rectangle (25.0,2.0);

% npn - pbase
\shade[upper left = pbase, upper right = pbase, lower right = nwell, lower left = nwell,] (28.25,\welldepthBfox) rectangle (32.0,2.0);

% pnp emitter and collector ring
\shade[upper left = pbase, upper right = pbase, lower right = nwell, lower left = nwell,] (35.5,\welldepthBfox) rectangle (37.0,2.0);
\shade[upper left = pbase, upper right = pbase, lower right = nwell, lower left = nwell,] (38.0,\welldepthBfox) rectangle (39.5,2.0);
\shade[upper left = pbase, upper right = pbase, lower right = nwell, lower left = nwell,] (40.5,\welldepthBfox) rectangle (42.0,2.0);

% n base for SONOS
\shade[upper left = nbase, upper right = nbase, lower right = pbase, lower left = pbase,] (18.75,\welldepthCfox) rectangle (24.75,2.0);

% n base for NPN emitter
\shade[upper left = nbase, upper right = nbase, lower right = pbase, lower left = pbase,] (29.5,\welldepthCfox) rectangle (30.75,2.0);




	\end{tikzpicture}
	\caption{Thin oxide}
\end{figure}

This might be difficult depending on the CVD machine used. Typically a reduced pressure can reduce the deposition rate, do not reduce the temperature however, since this can cause the formation of grains.

\subsection{Silicon nitride deposition}\label{step_depositing_sonos_nitride}

Now we need to add the nitride layer for forming the SONOS sandwich.

\begin{figure}[H]
	\centering
	\begin{tikzpicture}[node distance = 3cm, auto, thick,scale=\CrossSectionOnly, every node/.style={transform shape}]
		\input{tikz_process_steps/sonos.nitride_depositon.a.tex}
	\end{tikzpicture}
	\drawStepArrow{Nitride deposition}
	\begin{tikzpicture}[node distance = 3cm, auto, thick,scale=\CrossSectionOnly, every node/.style={transform shape}]
		\input{tikz_process_steps/sonos.nitride_depositon.b.tex}
	\end{tikzpicture}
	\caption{Polysilicon}
\end{figure}

We use the LPCVD machine and deposit a layer of around 7nm nitride.

Since the deposition rate of nitride in a CVD furnace is by nature quite slow (usually 2nm per minute) we have a much better control over this thickness.

\newpage

\subsection{Upper oxide deposition}\label{step_growing_gate_oxide}

Now we have to deposit the upper part of the SONOS sandwich by depositing LTO.

As designed in the process design document, the layer will be around 5nm thick.

\begin{figure}[H]
	\centering
	\begin{tikzpicture}[node distance = 3cm, auto, thick,scale=\CrossSectionOnly, every node/.style={transform shape}]
		\coveringlayer{nitride}{0.4}{0.75}
\coveringlayer{gateoxide}{0.2}{0.75}

\def\welldepthAfox{0.75}
\def\welldepthBfox{1.25}
\def\welldepthCfox{1.5}

\newcommand{\stopper}[1]{
\filldraw[line width=0, isolationoxide] (#1,2.0) -- (#1+0.75,2.0) -- (#1+0.75,2.75);
\fill[isolationoxide] (#1+0.75,2.0) rectangle (#1+1.25,2.75);
\filldraw[line width=0, isolationoxide] (#1+1.25,2.75) -- (#1+1.25,2.0) -- (#1+2.0,2.0);
}

\newcommand{\bjtstopper}[1]{
\filldraw[line width=0, isolationoxide] (#1,2.0) -- (#1+0.25,2.0) -- (#1+0.25,2.75);
\fill[isolationoxide] (#1+0.25,2.0) rectangle (#1+0.5,2.75);
\filldraw[line width=0, isolationoxide] (#1+0.5,2.75) -- (#1+0.5,2.0) -- (#1+0.75,2.0);
}

% oxide
\fill[isolationoxide] (0,1.25) rectangle (55.0,2.0);

\fill[isolationoxide] (0,2.0) rectangle (0.75,2.75);
\filldraw[line width=0, isolationoxide] (0.75,2.75) -- (0.75,2.0) -- (1.5,2.0);

%separates n+ from p+
\stopper{2.5}
\stopper{8.0}

%separates n+ from p+
\stopper{13.5}

\stopper{16.5}
\stopper{17.0}

\stopper{20.0}

\stopper{24.5}
\stopper{25.0}

\bjtstopper{27.75}
\bjtstopper{29.0}
\bjtstopper{30.5}
\bjtstopper{31.75}

\stopper{33.5}
\stopper{33.75}

\bjtstopper{36.5}
\bjtstopper{37.75}
\bjtstopper{39.0}
\bjtstopper{40.25}

\filldraw[line width=0, isolationoxide] (41.75,2.0) -- (42.5,2.0) -- (42.5,2.75);
\fill[isolationoxide] (42.5,2.0) rectangle (55.0,2.75);

% substrate
\fill[substrate] (0,0) rectangle (20,2);
\node at (2,0.5) {Silicon substrate};
%trenches
\fill[isolationoxide] (0,0.75) rectangle (1,2);
\fill[isolationoxide] (8.5,0.75) rectangle (11.5,2);
\fill[isolationoxide] (19,0.75) rectangle (20,2);

% substrate
\fill[substrate] (0,0) rectangle (55,1.25);
\node at (2,0.5) {Silicon substrate};

% normal wells
\shade[upper left = nwell, upper right = nwell, lower right = substrate, lower left = substrate,] (1.25,\welldepthAfox) rectangle (8.25,2.0);
\shade[upper left = pwell, upper right = pwell, lower right = substrate, lower left = substrate,] (9.75,\welldepthAfox) rectangle (16.75,2.0);
\shade[upper left = nwell, upper right = nwell, lower right = substrate, lower left = substrate,] (18.25,\welldepthAfox) rectangle (25.25,2.0);
\shade[upper left = nwell, upper right = nwell, lower right = substrate, lower left = substrate,] (26.75,\welldepthAfox) rectangle (33.75,2.0);
\shade[upper left = nwell, upper right = nwell, lower right = substrate, lower left = substrate,] (35.25,\welldepthAfox) rectangle (42.25,2.0);

% p base
\shade[upper left = pbase, upper right = pbase, lower right = nwell, lower left = nwell,] (18.5,\welldepthBfox) rectangle (25.0,2.0);

% npn - pbase
\shade[upper left = pbase, upper right = pbase, lower right = nwell, lower left = nwell,] (28.25,\welldepthBfox) rectangle (32.0,2.0);

% pnp emitter and collector ring
\shade[upper left = pbase, upper right = pbase, lower right = nwell, lower left = nwell,] (35.5,\welldepthBfox) rectangle (37.0,2.0);
\shade[upper left = pbase, upper right = pbase, lower right = nwell, lower left = nwell,] (38.0,\welldepthBfox) rectangle (39.5,2.0);
\shade[upper left = pbase, upper right = pbase, lower right = nwell, lower left = nwell,] (40.5,\welldepthBfox) rectangle (42.0,2.0);

% n base for SONOS
\shade[upper left = nbase, upper right = nbase, lower right = pbase, lower left = pbase,] (18.75,\welldepthCfox) rectangle (24.75,2.0);

% n base for NPN emitter
\shade[upper left = nbase, upper right = nbase, lower right = pbase, lower left = pbase,] (29.5,\welldepthCfox) rectangle (30.75,2.0);



	\end{tikzpicture}
	\drawStepArrow{LTO deposition}
	\begin{tikzpicture}[node distance = 3cm, auto, thick,scale=\CrossSectionOnly, every node/.style={transform shape}]
		\coveringlayer{gateoxide}{0.6}
\coveringlayer{nitride}{0.4}
\coveringlayer{gateoxide}{0.2}

\def\welldepthAfox{0.75}
\def\welldepthBfox{1.25}
\def\welldepthCfox{1.5}

\newcommand{\stopper}[1]{
\filldraw[line width=0, isolationoxide] (#1,2.0) -- (#1+0.75,2.0) -- (#1+0.75,2.75);
\fill[isolationoxide] (#1+0.75,2.0) rectangle (#1+1.25,2.75);
\filldraw[line width=0, isolationoxide] (#1+1.25,2.75) -- (#1+1.25,2.0) -- (#1+2.0,2.0);
}

\newcommand{\bjtstopper}[1]{
\filldraw[line width=0, isolationoxide] (#1,2.0) -- (#1+0.25,2.0) -- (#1+0.25,2.75);
\fill[isolationoxide] (#1+0.25,2.0) rectangle (#1+0.5,2.75);
\filldraw[line width=0, isolationoxide] (#1+0.5,2.75) -- (#1+0.5,2.0) -- (#1+0.75,2.0);
}

% oxide
\fill[isolationoxide] (0,1.25) rectangle (55.0,2.0);

\fill[isolationoxide] (0,2.0) rectangle (0.75,2.75);
\filldraw[line width=0, isolationoxide] (0.75,2.75) -- (0.75,2.0) -- (1.5,2.0);

%separates n+ from p+
\stopper{2.5}
\stopper{8.0}

%separates n+ from p+
\stopper{13.5}

\stopper{16.5}
\stopper{17.0}

\stopper{20.0}

\stopper{24.5}
\stopper{25.0}

\bjtstopper{27.75}
\bjtstopper{29.0}
\bjtstopper{30.5}
\bjtstopper{31.75}

\stopper{33.5}
\stopper{33.75}

\bjtstopper{36.5}
\bjtstopper{37.75}
\bjtstopper{39.0}
\bjtstopper{40.25}

\filldraw[line width=0, isolationoxide] (41.75,2.0) -- (42.5,2.0) -- (42.5,2.75);
\fill[isolationoxide] (42.5,2.0) rectangle (55.0,2.75);

% substrate
\fill[substrate] (0,0) rectangle (20,2);
\node at (2,0.5) {Silicon substrate};
%trenches
\fill[isolationoxide] (0,0.75) rectangle (1,2);
\fill[isolationoxide] (8.5,0.75) rectangle (11.5,2);
\fill[isolationoxide] (19,0.75) rectangle (20,2);

% substrate
\fill[substrate] (0,0) rectangle (55,1.25);
\node at (2,0.5) {Silicon substrate};

% normal wells
\shade[upper left = nwell, upper right = nwell, lower right = substrate, lower left = substrate,] (1.25,\welldepthAfox) rectangle (8.25,2.0);
\shade[upper left = pwell, upper right = pwell, lower right = substrate, lower left = substrate,] (9.75,\welldepthAfox) rectangle (16.75,2.0);
\shade[upper left = nwell, upper right = nwell, lower right = substrate, lower left = substrate,] (18.25,\welldepthAfox) rectangle (25.25,2.0);
\shade[upper left = nwell, upper right = nwell, lower right = substrate, lower left = substrate,] (26.75,\welldepthAfox) rectangle (33.75,2.0);
\shade[upper left = nwell, upper right = nwell, lower right = substrate, lower left = substrate,] (35.25,\welldepthAfox) rectangle (42.25,2.0);

% p base
\shade[upper left = pbase, upper right = pbase, lower right = nwell, lower left = nwell,] (18.5,\welldepthBfox) rectangle (25.0,2.0);

% npn - pbase
\shade[upper left = pbase, upper right = pbase, lower right = nwell, lower left = nwell,] (28.25,\welldepthBfox) rectangle (32.0,2.0);

% pnp emitter and collector ring
\shade[upper left = pbase, upper right = pbase, lower right = nwell, lower left = nwell,] (35.5,\welldepthBfox) rectangle (37.0,2.0);
\shade[upper left = pbase, upper right = pbase, lower right = nwell, lower left = nwell,] (38.0,\welldepthBfox) rectangle (39.5,2.0);
\shade[upper left = pbase, upper right = pbase, lower right = nwell, lower left = nwell,] (40.5,\welldepthBfox) rectangle (42.0,2.0);

% n base for SONOS
\shade[upper left = nbase, upper right = nbase, lower right = pbase, lower left = pbase,] (18.75,\welldepthCfox) rectangle (24.75,2.0);

% n base for NPN emitter
\shade[upper left = nbase, upper right = nbase, lower right = pbase, lower left = pbase,] (29.5,\welldepthCfox) rectangle (30.75,2.0);



	\end{tikzpicture}
	\caption{Thin oxide}
\end{figure}

It's exactly the same recipe as for the first oxide deposition step.

\subsection{Etching}

Now we've got to etch the SONOS structures.

\begin{figure}[H]
	\centering
	\begin{tikzpicture}[node distance = 3cm, auto, thick,scale=\CrossAndTopSection, every node/.style={transform shape}]
		\input{tikz_process_steps/sonos.etching.a.tex}
	\end{tikzpicture}
	\drawStepArrow{}
	\begin{tikzpicture}[node distance = 3cm, auto, thick,scale=\CrossAndTopSection, every node/.style={transform shape}]
		\fill[gateoxide] (21.40,\STIIslandSurface) rectangle (23.40,\SONOStopONE);
\fill[nitride] (21.40,\SONOStopONE) rectangle (23.40,\SONOStopTWO);
\fill[gateoxide] (21.40,\SONOStopTWO) rectangle (23.40,\SONOStopTHREE);

\def\welldepthAfox{0.75}
\def\welldepthBfox{1.25}
\def\welldepthCfox{1.5}

\newcommand{\stopper}[1]{
\filldraw[line width=0, isolationoxide] (#1,2.0) -- (#1+0.75,2.0) -- (#1+0.75,2.75);
\fill[isolationoxide] (#1+0.75,2.0) rectangle (#1+1.25,2.75);
\filldraw[line width=0, isolationoxide] (#1+1.25,2.75) -- (#1+1.25,2.0) -- (#1+2.0,2.0);
}

\newcommand{\bjtstopper}[1]{
\filldraw[line width=0, isolationoxide] (#1,2.0) -- (#1+0.25,2.0) -- (#1+0.25,2.75);
\fill[isolationoxide] (#1+0.25,2.0) rectangle (#1+0.5,2.75);
\filldraw[line width=0, isolationoxide] (#1+0.5,2.75) -- (#1+0.5,2.0) -- (#1+0.75,2.0);
}

% oxide
\fill[isolationoxide] (0,1.25) rectangle (55.0,2.0);

\fill[isolationoxide] (0,2.0) rectangle (0.75,2.75);
\filldraw[line width=0, isolationoxide] (0.75,2.75) -- (0.75,2.0) -- (1.5,2.0);

%separates n+ from p+
\stopper{2.5}
\stopper{8.0}

%separates n+ from p+
\stopper{13.5}

\stopper{16.5}
\stopper{17.0}

\stopper{20.0}

\stopper{24.5}
\stopper{25.0}

\bjtstopper{27.75}
\bjtstopper{29.0}
\bjtstopper{30.5}
\bjtstopper{31.75}

\stopper{33.5}
\stopper{33.75}

\bjtstopper{36.5}
\bjtstopper{37.75}
\bjtstopper{39.0}
\bjtstopper{40.25}

\filldraw[line width=0, isolationoxide] (41.75,2.0) -- (42.5,2.0) -- (42.5,2.75);
\fill[isolationoxide] (42.5,2.0) rectangle (55.0,2.75);

% substrate
\fill[substrate] (0,0) rectangle (20,2);
\node at (2,0.5) {Silicon substrate};
%trenches
\fill[isolationoxide] (0,0.75) rectangle (1,2);
\fill[isolationoxide] (8.5,0.75) rectangle (11.5,2);
\fill[isolationoxide] (19,0.75) rectangle (20,2);

% substrate
\fill[substrate] (0,0) rectangle (55,1.25);
\node at (2,0.5) {Silicon substrate};

% normal wells
\shade[upper left = nwell, upper right = nwell, lower right = substrate, lower left = substrate,] (1.25,\welldepthAfox) rectangle (8.25,2.0);
\shade[upper left = pwell, upper right = pwell, lower right = substrate, lower left = substrate,] (9.75,\welldepthAfox) rectangle (16.75,2.0);
\shade[upper left = nwell, upper right = nwell, lower right = substrate, lower left = substrate,] (18.25,\welldepthAfox) rectangle (25.25,2.0);
\shade[upper left = nwell, upper right = nwell, lower right = substrate, lower left = substrate,] (26.75,\welldepthAfox) rectangle (33.75,2.0);
\shade[upper left = nwell, upper right = nwell, lower right = substrate, lower left = substrate,] (35.25,\welldepthAfox) rectangle (42.25,2.0);

% p base
\shade[upper left = pbase, upper right = pbase, lower right = nwell, lower left = nwell,] (18.5,\welldepthBfox) rectangle (25.0,2.0);

% npn - pbase
\shade[upper left = pbase, upper right = pbase, lower right = nwell, lower left = nwell,] (28.25,\welldepthBfox) rectangle (32.0,2.0);

% pnp emitter and collector ring
\shade[upper left = pbase, upper right = pbase, lower right = nwell, lower left = nwell,] (35.5,\welldepthBfox) rectangle (37.0,2.0);
\shade[upper left = pbase, upper right = pbase, lower right = nwell, lower left = nwell,] (38.0,\welldepthBfox) rectangle (39.5,2.0);
\shade[upper left = pbase, upper right = pbase, lower right = nwell, lower left = nwell,] (40.5,\welldepthBfox) rectangle (42.0,2.0);

% n base for SONOS
\shade[upper left = nbase, upper right = nbase, lower right = pbase, lower left = pbase,] (18.75,\welldepthCfox) rectangle (24.75,2.0);

% n base for NPN emitter
\shade[upper left = nbase, upper right = nbase, lower right = pbase, lower left = pbase,] (29.5,\welldepthCfox) rectangle (30.75,2.0);



	\end{tikzpicture}
	\caption{Resist}
\end{figure}

The etching time depends on the dry etcher and recipe used.

