\subsection{Contacts}\label{chapter_contact}

Now we have to build the first set of vias connecting the first metal layer to the active area.
These vias are in the fringe between front-end and back-end process.

\begin{figure}[H]
	\centering
	\begin{tikzpicture}[node distance = 3cm, auto, thick,scale=\CrossAndTopSectionBig, every node/.style={transform shape}]
		\fill[isolationoxide] (0,0) rectangle (55.0,\LowerMetal);

\paintcontacts{white}{white}{white}

\input{tikz_process_steps/nimplant.a.tex}
\fill[pimplant] (3.5,1) rectangle (5,2);
\node at (4,1.5) {p+};
\fill[pimplant] (6.5,1) rectangle (8,2);
\node at (7,1.5) {p+};
\fill[pimplant] (17,1) rectangle (18.5,2);
\node at (18,1.5) {p+};

\filldraw[line width=0, isolationoxide] (5.75,2.0) -- (5.5,2.0) -- (5.75,3.0);
\filldraw[line width=0, isolationoxide] (6.75,2.0) -- (7.0,2.0) -- (6.75,3.0);

\filldraw[line width=0, isolationoxide] (11.25,2.0) -- (11.0,2.0) -- (11.25,3.0);
\filldraw[line width=0, isolationoxide] (12.5,2.0) -- (12.25,2.0) -- (12.25,3.0);

\filldraw[line width=0, isolationoxide] (22.75,2.4) -- (22.6,2.4) -- (22.75,3.4);
\filldraw[line width=0, isolationoxide] (23.75,2.4) -- (23.9,2.4) -- (23.75,3.4);

\fill[silicide] (1.5,1.9) rectangle (2.5,2);
\fill[silicide] (4.5,1.9) rectangle (5.5,2);
\fill[silicide] (5.75,2.9) rectangle (6.75,3.0);
\fill[silicide] (7,1.9) rectangle (8,2);

\fill[silicide] (10.0,1.9) rectangle (11.0,2);
\fill[silicide] (11.25,2.9) rectangle (12.25,3.0);
\fill[silicide] (12.5,1.9) rectangle (13.5,2);
\fill[silicide] (15.5,1.9) rectangle (16.5,2);

\fill[silicide] (19.0,1.9) rectangle (20.0,2);

\fill[silicide] (22.0,1.9) rectangle (22.6,2.0);
\fill[silicide] (22.75,3.3) rectangle (23.75,3.4);
\fill[silicide] (23.9,1.9) rectangle (24.5,2.0);

\fill[silicide] (27.0,1.9) rectangle (27.75,2.0);
\fill[silicide] (28.5,1.9) rectangle (29.0,2.0);
\fill[silicide] (29.75,1.9) rectangle (30.5,2.0);
\fill[silicide] (31.25,1.9) rectangle (31.75,2.0);
\fill[silicide] (32.5,1.9) rectangle (33.5,2.0);

\fill[silicide] (35.75,1.9) rectangle (36.5,2.0);
\fill[silicide] (37.25,1.9) rectangle (37.75,2.0);
\fill[silicide] (38.5,1.9) rectangle (39.0,2.0);
\fill[silicide] (39.75,1.9) rectangle (40.25,2.0);
\fill[silicide] (41.0,1.9) rectangle (41.75,2.0);

% diode contacts
\fill[silicide] (43.0,3.4) rectangle (44.0,3.5);
\fill[silicide] (47.0,3.4) rectangle (48.0,3.5);

% resistor contacts
\fill[silicide] (49.0,3.4) rectangle (49.5,3.5);
\fill[silicide] (53.5,3.4) rectangle (54.0,3.5);


	\end{tikzpicture}
	\caption{Contact geometry target}
	\label{contact_cross_subsections}
\end{figure}

As can be seen in \autoref{contact_cross_subsections}, the goal of this step is purely to deposit a layer of isolation oxide, get the holes into it, down to the silicide and polyside in order to form wires later on.
We do not wanna etch down anywhere else than the silicide/polycide areas because these function as etch stoppers, while everywhere else we might etch further than desired with small variations in etching time which might result in a drastic variation in sheet resistance of the junctions and gate.
In later iterations of this process we might be switching to Tungsten as the metal material for this step.
