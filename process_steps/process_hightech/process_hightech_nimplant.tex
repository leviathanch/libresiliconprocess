\section{n+ Implant}\label{nimplant}
For the bulk of the PMOS transistors and for the source and drain of the NMOS transistors highly doped  n+ areas are required.
In this step we're going to build these.

\begin{figure}[H]
	\centering
	\begin{tikzpicture}[node distance = 3cm, auto, thick,scale=\CrossAndTopSectionBig, every node/.style={transform shape}]
		% substrate
\fill[YellowOrange] (0,0) rectangle (20,2);
\node at (2,0.5) {Si (p-type)};
% n-well
\fill[Goldenrod] (1.25,0.75) rectangle (8.25,2);
\node at (5.75,1) {N-Well};
% body
\fill[ProcessBlue] (1.5,1) rectangle (3,2);
\node at (2,1.5) {n+};
% source
\fill[RedOrange] (3.5,1) rectangle (5,2);
\node at (4,1.5) {p+};
% drain
\fill[RedOrange] (6.5,1) rectangle (8,2);
\node at (7,1.5) {p+};
%% gate:
% gate oxide
\fill[LightGray] (4.8,2) rectangle (6.7,2.1);
% gate poly
\fill[BrickRed] (4.8,2.1) rectangle (6.7,2.2);

%field oxides:
\fill[DarkGray] (0,2) rectangle (1,4);
\fill[DarkGray] (8.5,2) rectangle (11.5,4);
\fill[DarkGray] (19,2) rectangle (20,4);

\fill[RedOrange] (0,1.5) rectangle (1,2);
\fill[RedOrange] (8.5,1.5) rectangle (11.5,2);
\fill[RedOrange] (19,1.5) rectangle (20,2);

\node at (0.5,1.75) {p+};
\node at (9.5,1.75) {p+};
\node at (19.5,1.75) {p+};

%%% nmos:
% body
\fill[RedOrange] (17,1) rectangle (18.5,2);
\node at (18,1.5) {p+};
% source
\fill[ProcessBlue] (15,1) rectangle (16.5,2);
\node at (16,1.5) {n+};
% drain
\fill[ProcessBlue] (12,1) rectangle (13.5,2);
\node at (13,1.5) {n+};

%% gate:
% gate oxide
\fill[LightGray] (13.3,2) rectangle (15.2,2.1);
% gate poly
\fill[BrickRed] (13.3,2.1) rectangle (15.2,2.2);
	\end{tikzpicture}
	\begin{tikzpicture}[node distance = 3cm, auto, thick,scale=\CrossAndTopSectionBig, every node/.style={transform shape}]
		\fill[isolationoxide] (0,0) rectangle (20,10);

% n-well
\fill[nwell] (1,1.25) rectangle (8.5,7.5);

% p-well
\fill[pwell] (11.5,1.25) rectangle (19,7.5);

% gate metal
\fill[gatemetal] (5,0) rectangle (6.5,9);
\fill[gatemetal] (13.5,0) rectangle (15,9);
\fill[gatemetal] (5,8) rectangle (15,10);

% n+
\fill[nimplant] (1.5,2) rectangle (3,6.5);
\fill[nimplant] (12,2) rectangle (13.5,6.5);
\fill[nimplant] (15,2) rectangle (16.5,6.5);

	\end{tikzpicture}
	\caption{N+ implant geometry target}
\end{figure}

The tricky thing here is to have a reasonable implant depth but not too deep because the deeper the junction, the higher the junction capacity which in turn limits the switching performance of the CMOS circuitry.

\begin{figure}[H]
	\centering
	\begin{tikzpicture}[node distance =1cm, auto, thick,scale=\VLSILayout, every node/.style={transform shape}]
		\input{tikz_process_steps/nimplant.layout.tex}

% p+
\fill[pimplant,opacity=\OpacityLayout] (3.5,2) rectangle (5,6.5);
\fill[pimplant,opacity=\OpacityLayout] (6.5,2) rectangle (8,6.5);
\fill[pimplant,opacity=\OpacityLayout] (17,2) rectangle (18.5,6.5);
% gate metal
\fill[gatemetal,opacity=\OpacityLayout] (4.8,1.75) rectangle (6.7,8);
\fill[gatemetal,opacity=\OpacityLayout] (13.3,1.75) rectangle (15.2,8);
\fill[gatemetal,opacity=\OpacityLayout] (4.8,8) rectangle (15.2,10);


% n+
\fill[nimplant,opacity=\OpacityLayout] (1.5,2) rectangle (3,6.5);
\fill[nimplant,opacity=\OpacityLayout] (12,2) rectangle (13.5,6.5);
\fill[nimplant,opacity=\OpacityLayout] (15,2) rectangle (16.5,6.5);

	\end{tikzpicture}
	\caption{N+ layout}
	\label{nimplant_layout}
\end{figure}

An example layout of p-implants can be seen in \autoref{nimplant_layout}, the mask is being extracted from the layer "n\_plus\_select".

Also important to notice is that this example layout is just for demonstration purposes only, please have a look at the standard cell documentation for the actual layouts. 

\newpage

\subsection{Patterning}

The resist is being deposited using spin coating and then baked depending on the baking time for the specific resist.
The layout for being exposed onto the resist is being extracted from the "n\_plus\_select" layer within the GDS2 file onto a \textbf{bright field} mask.
The requirement is a \textbf{negative} tone resist.

\begin{figure}[H]
	\centering
	\begin{tikzpicture}[node distance = 3cm, auto, thick,scale=\CrossAndTopSection, every node/.style={transform shape}]
		\input{tikz_process_steps/nwell.a.tex}
% p-well
\fill[pwell] (11.5,0.75) rectangle (19,2);
\node at (14.25,1) {P-Well};
\fill[gateoxide] (4.8,2) rectangle (6.7,2.3);
\fill[gateoxide] (13.3,2) rectangle (15.2,2.3);
\fill[gatemetal] (4.8,2.3) rectangle (6.7,2.6);
\fill[gatemetal] (13.3,2.3) rectangle (15.2,2.6);

	\end{tikzpicture}
	\begin{tikzpicture}[node distance = 3cm, auto, thick,scale=\CrossAndTopSection, every node/.style={transform shape}]
		\input{tikz_process_steps/nimplant.patterning.at.tex}
	\end{tikzpicture}
	\drawStepArrow{Mask: nselect}
	\begin{tikzpicture}[node distance = 3cm, auto, thick,scale=\CrossAndTopSection, every node/.style={transform shape}]
		% resist
\fill[resist] (0,2.0) rectangle (1.25,6.0);
\fill[resist] (3,2.0) rectangle (11.75,6.0);
\fill[resist] (17,2.0) rectangle (20,6.0);

\input{tikz_process_steps/nimplant.oxide_growth.b.tex}

	\end{tikzpicture}
	\begin{tikzpicture}[node distance = 3cm, auto, thick,scale=\CrossAndTopSection, every node/.style={transform shape}]
		\fill[resist] (0,0) rectangle (20,12);

% n+
\fill[isolationoxide] (1.5,2) rectangle (3,6.5);
\fill[isolationoxide] (12,2) rectangle (13.5,6.5);
\fill[isolationoxide] (15,2) rectangle (16.5,6.5);
	\end{tikzpicture}
	\caption{N+ region resist mask}
\end{figure}

The thickness of the resist layer and the baking duration will variate depending on the specific equipment for which this process will be implemented with.
Also after the exposure and development, the hard baking shouldn't be forgotten!

\newpage

\subsection{Implantation}\label{nimplant_implant_step}

We now need to bring in the carriers in order to build the n-junctions.

\begin{figure}[H]
	\centering
	\begin{tikzpicture}[node distance = 3cm, auto, thick,scale=\CrossSectionOnly, every node/.style={transform shape}]
		% oxide
\fill[isolationoxide] (0,2) rectangle (1.25,3.5);
\fill[isolationoxide] (3,2) rectangle (11.75,3.5);
\fill[isolationoxide] (17,2) rectangle (20,3.5);

\forloop{ct}{0}{\value{ct} < 21}
{
	\draw [->] (\value{ct},5) -- (\value{ct},4);
	\node at (\value{ct},5.2) {P$^{31}$};
}

\input{tikz_process_steps/nwell.a.tex}
% p-well
\fill[pwell] (11.5,0.75) rectangle (19,2);
\node at (14.25,1) {P-Well};
\fill[gateoxide] (4.8,2) rectangle (6.7,2.3);
\fill[gateoxide] (13.3,2) rectangle (15.2,2.3);
\fill[gatemetal] (4.8,2.3) rectangle (6.7,2.6);
\fill[gatemetal] (13.3,2.3) rectangle (15.2,2.6);
	\end{tikzpicture}
	\drawStepArrow{Boron implant}
	\begin{tikzpicture}[node distance = 3cm, auto, thick,scale=\CrossSectionOnly, every node/.style={transform shape}]
		% resist
\fill[resist] (0,2.0) rectangle (0.75,5.0);
\fill[resist] (3.25,2.0) rectangle (11.25,5.0);
\fill[resist] (16.25,2.0) rectangle (20,5.0);

\input{tikz_process_steps/pwell.a.tex}
\fill[gateoxide] (4.8,2) rectangle (6.7,2.3);
\fill[gateoxide] (13.3,2) rectangle (15.2,2.3);
\fill[gatemetal] (4.8,2.3) rectangle (6.7,2.6);
\fill[gatemetal] (13.3,2.3) rectangle (15.2,2.6);

\shade[upper left = nimplant, upper right = nimplant, lower right = nwell, lower left = nwell,] (1.5,1.5) rectangle (2.5,2);
\node at (2,1.65) {n+};
\shade[upper left = nimplant, upper right = nimplant, lower right = pwell, lower left = pwell,] (12.0,1.5) rectangle (13.0,2);
\node at (12.5,1.65) {n+};
\shade[upper left = nimplant, upper right = nimplant, lower right = pwell, lower left = pwell,] (14.5,1.5) rectangle (15.5,2);
\node at (15,1.65) {n+};



	\end{tikzpicture} \\
	\caption{N+ implant process}
\end{figure}

\textbf{Possible approaches}:
\begin{itemize}
	\item \textbf{"CF-3000 Implanter (IMP-3000)" from HKUST} \\
	At HKUST we have an implanter which gives us better control over the initial surface concentration. \\
	These steps are needed to arrive with the desired geometry:
	The nselect is implanted with a Phosphorus ($P^{31}$) dose of $2.5\times10^{12}cm^{-2}$ at an energy of 35 keV (43nm$\pm$18nm deep)
\end{itemize}

\subsection{Resist strip}

Now we need to remove the contaminants for further processing.

\begin{figure}[H]
	\centering
	\begin{tikzpicture}[node distance = 3cm, auto, thick,scale=\CrossSectionOnly, every node/.style={transform shape}]
		% resist
\fill[resist] (0,2.0) rectangle (0.75,5.0);
\fill[resist] (3.25,2.0) rectangle (11.25,5.0);
\fill[resist] (16.25,2.0) rectangle (20,5.0);

\input{tikz_process_steps/gate.a.tex}

\shade[upper left = nimplant, upper right = nimplant, lower right = nwell, lower left = nwell,] (1.5,1.5) rectangle (2.5,2);
\node at (2,1.65) {n+};
\shade[upper left = nimplant, upper right = nimplant, lower right = pwell, lower left = pwell,] (12.0,1.5) rectangle (13.0,2);
\node at (12.5,1.65) {n+};
\shade[upper left = nimplant, upper right = nimplant, lower right = pwell, lower left = pwell,] (14.5,1.5) rectangle (15.5,2);
\node at (15,1.65) {n+};




	\end{tikzpicture}
	\drawStepArrow{}
	\begin{tikzpicture}[node distance = 3cm, auto, thick,scale=\CrossSectionOnly, every node/.style={transform shape}]
		\input{tikz_process_steps/nwell.a.tex}
% p-well
\fill[pwell] (11.5,0.75) rectangle (19,2);
\node at (14.25,1) {P-Well};
\fill[gateoxide] (4.8,2) rectangle (6.7,2.3);
\fill[gateoxide] (13.3,2) rectangle (15.2,2.3);
\fill[gatemetal] (4.8,2.3) rectangle (6.7,2.6);
\fill[gatemetal] (13.3,2.3) rectangle (15.2,2.6);

\shade[upper left = nimplant, upper right = nimplant, lower right = nwell, lower left = nwell,] (1.5,1.5) rectangle (2.5,2);
\node at (2,1.65) {n+};
\shade[upper left = nimplant, upper right = nimplant, lower right = pwell, lower left = pwell,] (12.0,1.5) rectangle (13.0,2);
\node at (12.5,1.65) {n+};
\shade[upper left = nimplant, upper right = nimplant, lower right = pwell, lower left = pwell,] (14.5,1.5) rectangle (15.5,2);
\node at (15,1.65) {n+};


	\end{tikzpicture}
	\caption{Resist removal}
\end{figure}

We strip the resist, rinse and perform sulfuric cleaning.

\newpage

