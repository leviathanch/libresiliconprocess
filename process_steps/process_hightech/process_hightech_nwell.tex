\subsection{N-well}\label{nwell_chapter}
In order to build CMOS on the same substrate, an N-well is required for building the complementary P-channel transistor for a NFET+PFET logic circuitry.

The cross section as well as the top view of the targeted geometry are shown in \autoref{nwell_target}

\begin{figure}[H]
	\centering
	\begin{tikzpicture}[node distance = 3cm, auto, thick,scale=\CrossAndTopSectionBig, every node/.style={transform shape}]
		\input{tikz_process_steps/pwell.a.tex}
% n-well
\fill[nwell] (1.25,0.75) rectangle (8.5,2);
\node at (5.75,1) {N-Well};
% p-well
\fill[pwell] (11.75,0.25) rectangle (18.75,2);
\node at (15.25,1) {P-Well};
% p-well
\fill[pwell] (11.5,0.75) rectangle (19,2);
\node at (14.25,1) {P-Well};
% n-well
\fill[nwell] (1.25,0.75) rectangle (8.5,2);
\node at (5.75,1) {N-Well};
% p-well
\fill[pwell] (11.75,0.25) rectangle (18.75,2);
\node at (15.25,1) {P-Well};
	\end{tikzpicture}
	\begin{tikzpicture}[node distance = 3cm, auto, thick,scale=\CrossAndTopSectionBig, every node/.style={transform shape}]
		% substrate
\fill[YellowOrange] (0,0) rectangle (20,12);
% trench area
\fill[DarkGray] (0,0) rectangle (1,12);
\fill[DarkGray] (8.5,0) rectangle (11.5,12);
\fill[DarkGray] (19,0) rectangle (20,12);
\fill[DarkGray] (0,0) rectangle (20,1.25);
\fill[DarkGray] (0,7.5) rectangle (20,12);
\fill[nwell] (1.25,1) rectangle (8.25,7.25);
	\end{tikzpicture}
	\caption{N-well target geometry}
	\label{nwell_target}
\end{figure}

The N-well will serve us as an island of N-doped substrate within the P-doped basis substrate.

The P-dopant concentration of our prime grade, p-type, single side polished, four inch wafers is between $8.76 \cdot 10^{14} \frac{1}{cm^3}$ and $5.23 \cdot 10^{14} \frac{1}{cm^3}$

This means we need a dose of $2.33\times10^{12}cm^{-2}$ Phosphorus at 70 keV.

The concentration will need adjustment when the used substrate has different properties!
