\section{P-well}\label{pwell_chapter}
In order to build CMOS on the same substrate, an P-well is required for building the complementary P-channel transistor for a n-p-channel logic circuitry as shown above in the example section.
The cross section as well as the top view of the targeted geometry are shown in \autoref{nwell_target}
\begin{figure}[H]
	\centering
	\begin{tikzpicture}[node distance = 3cm, auto, thick,scale=\CrossAndTopSectionBig, every node/.style={transform shape}]
		\input{tikz_process_steps/nwell.a.tex}
% p-well
\fill[pwell] (11.5,0.75) rectangle (19,2);
\node at (14.25,1) {P-Well};
% n-well
\fill[nwell] (1.25,0.75) rectangle (8.5,2);
\node at (5.75,1) {N-Well};
% p-well
\fill[pwell] (11.75,0.25) rectangle (18.75,2);
\node at (15.25,1) {P-Well};
% p-well
\fill[pwell] (11.5,0.75) rectangle (19,2);
\node at (14.25,1) {P-Well};
	\end{tikzpicture}
	\begin{tikzpicture}[node distance = 3cm, auto, thick,scale=\CrossAndTopSectionBig, every node/.style={transform shape}]
		% substrate
\fill[YellowOrange] (0,0) rectangle (20,12);
% trench area
\fill[DarkGray] (0,0) rectangle (1,12);
\fill[DarkGray] (8.5,0) rectangle (11.5,12);
\fill[DarkGray] (19,0) rectangle (20,12);
\fill[DarkGray] (0,0) rectangle (20,1.25);
\fill[DarkGray] (0,7.5) rectangle (20,12);
\fill[nwell] (1.25,1) rectangle (8.25,7.25);
\fill[pwell] (11.5,1.25) rectangle (19,7.5);
	\end{tikzpicture}
	\caption{P-well target geometry}
	\label{pwell_target}
\end{figure}
The P-well will serve us as an island of higher p-doped substrate within the slightly p-doped basis substrate.

The dopant dose will be: $2.5\times10^{12}cm^{-2}$

\begin{figure}[H]
	\centering
	\begin{tikzpicture}[node distance =1cm, auto, thick,scale=\VLSILayout, every node/.style={transform shape}]
		\fill[Goldenrod,opacity=0.2] (0.75,0.5) rectangle (8.75,7.75);
\fill[Goldenrod,opacity=0.2] (11.25,0.5) rectangle (19.25,7.75);

\draw[dotted] (20.5,0.5) rectangle (25,5.5);

\node at (22.25,5) {\textbf{Layers}};

\fill[Goldenrod,opacity=0.2] (21,1) rectangle (21.5,1.5);
\node at (22.25,1.25) {active};

\fill[orange,opacity=0.2] (21,1.5) rectangle (21.5,2);
\node at (22.25,1.75) {nwell};

\fill[blue,opacity=0.2] (21,2) rectangle (21.5,2.5);
\node at (22.25,2.25) {nimplant};

\fill[red,opacity=0.2] (21,2.5) rectangle (21.5,3);
\node at (22.25,2.75) {pimplant};

\fill[Emerald,opacity=0.2] (21,3) rectangle (21.5,3.5);
\node at (22.25,3.25) {gate};

\fill[Fuchsia,opacity=0.2] (21,3.5) rectangle (21.5,4);
\node at (22.25,3.75) {metal1};

\fill[DarkOrchid,opacity=0.2] (21,4) rectangle (21.5,4.5);
\node at (22.25,4.25) {via1};
\fill[pwell,opacity=\OpacityLayout] (11.5,0.75) rectangle (19,7.5);
	\end{tikzpicture}
	\caption{P-Well layout}
	\label{pwell_layout}
\end{figure}

In \autoref{pwell_layout} the layout of the P-well region on top of the active area region can be seen.

The p-well is being fit into the active area.

It should even be a little bit bigger than the active area, because of possible alignment offsets

\newpage

\subsection{Mask dioxide layer}
In order to selectively inject charge carrying atoms into the crystalline structure a protective dioxide ($SiO_2$) layer needs to be grown on top of a p-type substrate.
\begin{figure}[H]
	\centering
	\begin{tikzpicture}[node distance = 3cm, auto, thick,scale=\CrossSectionOnly, every node/.style={transform shape}]
		\input{tikz_process_steps/pwell.1.a.tex}
	\end{tikzpicture} \\
	\includegraphics[scale=0.01]{down_arrow.png} \\
	\begin{tikzpicture}[node distance = 3cm, auto, thick,scale=\CrossSectionOnly, every node/.style={transform shape}]
		\input{tikz_process_steps/nwell.a.tex}
% p-well
\fill[pwell] (11.5,0.75) rectangle (19,2);
\node at (14.25,1) {P-Well};
% n-well
\fill[nwell] (1.25,0.75) rectangle (8.5,2);
\node at (5.75,1) {N-Well};
% p-well
\fill[pwell] (11.75,0.25) rectangle (18.75,2);
\node at (15.25,1) {P-Well};

% oxide
\fill[isolationoxide] (0,2) rectangle (20,2.6);
\node at (2,2.1) {SiO2};
	\end{tikzpicture}
	\caption{Dioxide layer growth}
\end{figure}
The industrial best practice is a layer of around (500nm$\approx$5000\normalfont\AA) thickness or more.
For this purpose the wafer is being oxidized for at least 90 minutes at 1000\degree C using wet oxidation which results in a dioxide layer at least 500nm($\approx$5000\normalfont\AA) in thickness.

\subsection{Patterning}
The resist is being deposited using spin coating and then baked depending on the baking time for the specific resist.
The layout for being exposed onto the resist is being extracted from the "pwell" layer within the GDS2 file.
\begin{figure}[H]
	\centering
	\begin{tikzpicture}[node distance = 3cm, auto, thick,scale=\CrossAndTopSection, every node/.style={transform shape}]
		\input{tikz_process_steps/nwell.a.tex}
% p-well
\fill[pwell] (11.5,0.75) rectangle (19,2);
\node at (14.25,1) {P-Well};
% n-well
\fill[nwell] (1.25,0.75) rectangle (8.5,2);
\node at (5.75,1) {N-Well};
% p-well
\fill[pwell] (11.75,0.25) rectangle (18.75,2);
\node at (15.25,1) {P-Well};
% oxide
\fill[isolationoxide] (0,2) rectangle (20,2.6);
	\end{tikzpicture}
	\begin{tikzpicture}[node distance = 3cm, auto, thick,scale=\CrossAndTopSection, every node/.style={transform shape}]
		\input{tikz_process_steps/pwell.2.at.tex}
	\end{tikzpicture} \\
	\includegraphics[scale=0.01]{down_arrow.png} \\
	\begin{tikzpicture}[node distance = 3cm, auto, thick,scale=\CrossAndTopSection, every node/.style={transform shape}]
		\input{tikz_process_steps/pwell.2.b.tex}
	\end{tikzpicture}
	\begin{tikzpicture}[node distance = 3cm, auto, thick,scale=\CrossAndTopSection, every node/.style={transform shape}]
		\input{tikz_process_steps/pwell.2.bt.tex}
	\end{tikzpicture}
	\caption{Cross/top view of P-well layout on resist}
\end{figure}
The thickness of the resist layer and the baking duration will variate depending on the specific equipment for which this process will be implemented with.

\subsection{Etching}
We now need to open a window in the dioxide layer, through which we will inject carrier atoms into the silicon crystal structure.
\begin{figure}[H]
	\centering
	\begin{tikzpicture}[node distance = 3cm, auto, thick,scale=\CrossAndTopSection, every node/.style={transform shape}]
		\input{tikz_process_steps/pwell.3.a.tex}
	\end{tikzpicture}
	\begin{tikzpicture}[node distance = 3cm, auto, thick,scale=\CrossAndTopSection, every node/.style={transform shape}]
		\input{tikz_process_steps/pwell.3.at.tex}
	\end{tikzpicture} \\
	\includegraphics[scale=0.01]{down_arrow.png} \\
	\begin{tikzpicture}[node distance = 3cm, auto, thick,scale=\CrossAndTopSection, every node/.style={transform shape}]
		\input{tikz_process_steps/pwell.3.b.tex}
	\end{tikzpicture}
	\begin{tikzpicture}[node distance = 3cm, auto, thick,scale=\CrossAndTopSection, every node/.style={transform shape}]
		\input{tikz_process_steps/pwell.3.bt.tex}
	\end{tikzpicture}
	\caption{Cross/top view of P-well oxide window}
\end{figure}
Since the silicon dioxide layer is 500nm thick and we wanna reach the silicon below we can use wet etching as described in the chemistry chapter.
Using BHF (6:1) (\autoref{BHF_six_to_one}) we can etch with a speed of approximately 2 nm/s at 25 \degree C, we can calculate the etching time to be $\frac{500nm}{2nm/s}$=250s=4 minutes 10 seconds (or make it rather 30 seconds instead of 10)

\subsection{Cleaning}
In order to avoid contamination of the machines we need to make sure all the resist has been stripped off from the wafer.
\begin{figure}[H]
	\centering
	\begin{tikzpicture}[node distance = 3cm, auto, thick,scale=\CrossSectionOnly, every node/.style={transform shape}]
		\input{tikz_process_steps/nwell.a.tex}
% p-well
\fill[pwell] (11.5,0.75) rectangle (19,2);
\node at (14.25,1) {P-Well};
% n-well
\fill[nwell] (1.25,0.75) rectangle (8.5,2);
\node at (5.75,1) {N-Well};
% p-well
\fill[pwell] (11.75,0.25) rectangle (18.75,2);
\node at (15.25,1) {P-Well};

% oxide
\fill[isolationoxide] (0,2) rectangle (11.5,2.6);
\fill[isolationoxide] (19,2) rectangle (20,2.6);
% resist
\fill[resist] (0,2.6) rectangle (11.5,3.2);
\fill[resist] (19,2.6) rectangle (20,3.2);
	\end{tikzpicture} \\
	\includegraphics[scale=0.01]{down_arrow.png} \\
	\begin{tikzpicture}[node distance = 3cm, auto, thick,scale=\CrossSectionOnly, every node/.style={transform shape}]
		\input{tikz_process_steps/nwell.a.tex}
% p-well
\fill[pwell] (11.5,0.75) rectangle (19,2);
\node at (14.25,1) {P-Well};
% n-well
\fill[nwell] (1.25,0.75) rectangle (8.5,2);
\node at (5.75,1) {N-Well};
% p-well
\fill[pwell] (11.75,0.25) rectangle (18.75,2);
\node at (15.25,1) {P-Well};

% oxide
\fill[isolationoxide] (0,2) rectangle (11.5,2.6);
\fill[isolationoxide] (19,2) rectangle (20,2.6);
	\end{tikzpicture}
	\caption{Resist removal}
\end{figure}
Please just use the solvent for the specific resist.

\subsection{Injection}
We now need to inject the carriers into the upper level of the n-channel area so that we can later on drive them into the crystal during the drive-in step.
\begin{figure}[H]
	\centering
	\begin{tikzpicture}[node distance = 3cm, auto, thick,scale=\CrossSectionOnly, every node/.style={transform shape}]
		\input{tikz_process_steps/nwell.a.tex}
% p-well
\fill[pwell] (11.5,0.75) rectangle (19,2);
\node at (14.25,1) {P-Well};
% n-well
\fill[nwell] (1.25,0.75) rectangle (8.5,2);
\node at (5.75,1) {N-Well};
% p-well
\fill[pwell] (11.75,0.25) rectangle (18.75,2);
\node at (15.25,1) {P-Well};

% oxide
\fill[isolationoxide] (0,2) rectangle (11.5,2.6);
\fill[isolationoxide] (19,2) rectangle (20,2.6);

\forloop{ct}{0}{\value{ct} < 21}
{
	\draw [->] (\value{ct},4) -- (\value{ct},3);
	\node at (\value{ct},4.2) {P$^{31}$};
}
	\end{tikzpicture} \\
	\includegraphics[scale=0.01]{down_arrow.png} \\
	\begin{tikzpicture}[node distance = 3cm, auto, thick,scale=\CrossSectionOnly, every node/.style={transform shape}]
		\input{tikz_process_steps/pwell.5.b.tex}
	\end{tikzpicture}
	\caption{Doping process}
\end{figure}
The P-well is implanted with a Boron ($B^{11}$) dose of $2.5\times10^{12}cm^{-2}$ at an energy of 100 KeV.
The P-well is then annealed.

\subsection{Oxide for drive-in}
Now we need to cover the now doped and annealed areas with an oxide layer for the drive-in phase.
\begin{figure}[H]
	\centering
	\begin{tikzpicture}[node distance = 3cm, auto, thick,scale=\CrossSectionOnly, every node/.style={transform shape}]
		\input{tikz_process_steps/pwell.6.a.tex}
	\end{tikzpicture} \\
	\includegraphics[scale=0.01]{down_arrow.png} \\
	\begin{tikzpicture}[node distance = 3cm, auto, thick,scale=\CrossSectionOnly, every node/.style={transform shape}]
		\input{tikz_process_steps/pwell.6.b.tex}
	\end{tikzpicture}
	\caption{Oxide growth}
\end{figure}
The wafer is being oxidized for 32 minutes at 1000\degree C in order to achieve a cover silicon layer of 250nm thickness ($\approx$2500\normalfont\AA).

\subsection{Drive-in}
In order to drive the carrier atoms deeper into the crystalline structure the wafer needs to be driven in after predeposition.
\begin{figure}[H]
	\centering
	\begin{tikzpicture}[node distance = 3cm, auto, thick,scale=\CrossSectionOnly, every node/.style={transform shape}]
		\input{tikz_process_steps/pwell.7.a.tex}
	\end{tikzpicture} \\
	\includegraphics[scale=0.01]{down_arrow.png} \\
	\begin{tikzpicture}[node distance = 3cm, auto, thick,scale=\CrossSectionOnly, every node/.style={transform shape}]
		\input{tikz_process_steps/pwell.7.b.tex}
	\end{tikzpicture}
	\caption{Drive-in process}
\end{figure}
In this step the wafer is  driven-in for 96 minutes at 1150\degree C in an inert ambient.

\subsection{Oxide mask removal}
We want to remove the silicon mask from the wafer so that the P-well becomes accessible for the further process steps but we don't want to etch "way too much" of the trench material.
\begin{figure}[H]
	\centering
	\begin{tikzpicture}[node distance = 3cm, auto, thick,scale=\CrossSectionOnly, every node/.style={transform shape}]
		\input{tikz_process_steps/pwell.8.a.tex}
	\end{tikzpicture} \\
	\includegraphics[scale=0.01]{down_arrow.png} \\
	\begin{tikzpicture}[node distance = 3cm, auto, thick,scale=\CrossSectionOnly, every node/.style={transform shape}]
		% substrate
\fill[substrate] (0,0) rectangle (20,2);
\node at (2,0.5) {Silicon substrate};
%trenches
\fill[isolationoxide] (0,0.75) rectangle (1,2);
\fill[isolationoxide] (8.5,0.75) rectangle (11.5,2);
\fill[isolationoxide] (19,0.75) rectangle (20,2);

% n-well
\fill[nwell] (1,0.75) rectangle (8.5,2);
\node at (4.75,1) {N-Well};
	\end{tikzpicture}
	\caption{Oxide removal}
\end{figure}
Since the silicon dioxide layer is 750nm (500nm+250nm) thick and we wanna reach the silicon below we can use wet etching as described in the chemistry chapter.
Using BHF (6:1) (\autoref{BHF_six_to_one}) we can etch with a speed of approximately 2 nm/s at 25 \degree C.
We can calculate the etching time to be $\frac{750nm}{2nm/s}$=375s=6 Minutes and 15 Seconds.

Etching away a "little bit too much" of the oxide isn't that bad, because the oxide within the trenches will be "filled up" again during the later steps.
