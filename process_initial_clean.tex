\subsection{Initial cleaning}
In order to remove the initial naturally grown silicon dioxide from the wafer, acid is being applied to the wafer which leads to a pure silicon substrate wafer as in the process illustration shown below.

\begin{center}
	\begin{figure}[h]
		\begin{center}
			\begin{tikzpicture}[node distance = 3cm, auto, thick,scale=0.3, every node/.style={transform shape}]
				\input{tikz_process_steps/1.a.tex}
			\end{tikzpicture} \\
			\includegraphics[scale=0.01]{down_arrow.png} \\
			\begin{tikzpicture}[node distance = 3cm, auto, thick,scale=0.3, every node/.style={transform shape}]
				\input{tikz_process_steps/1.b.tex}
			\end{tikzpicture}
		\end{center}
		\caption{Initial cleaning}
	\end{figure}
\end{center}

This needs to be done because the naturally grown initially existing silicon oxide is not pure and may contain contamination which may render the final product unusable.

\subsubsection{Sulfuric Cleaning}
The sulfuric acid mixture, $H_2 S O_4 + H_2 O_2$ is being applied to the wafer for 10 minutes at a temperature of 120 \degree C.

\subsubsection{HF dip}
After the sulfuric cleaning a HF (HF:$H_2O$,1:50) dip is being performed for one minute. \\
Hydrofluoric acid (HF) is used to remove native silicon dioxide from wafers. Since it acts quickly, one needs to only expose the wafer for a short time ("dip"). \\
After that the wafer needs to be dried and quickly processed further before new uncontrolled natural oxide can build up on the wafer through the contact with air.