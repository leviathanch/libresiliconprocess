\subsection{p+ Implant}
\begin{center}
	\begin{tikzpicture}[node distance = 3cm, auto, thick,scale=\CrossAndTopSectionBig, every node/.style={transform shape}]
		% substrate
\fill[YellowOrange] (0,0) rectangle (20,2);
\node at (2,0.5) {Si (p-type)};
% n-well
\fill[Goldenrod] (1.25,0.75) rectangle (8.25,2);
\node at (5.75,1) {N-Well};
% body
\fill[ProcessBlue] (1.5,1) rectangle (3,2);
\node at (2,1.5) {n+};
% source
\fill[RedOrange] (3.5,1) rectangle (5,2);
\node at (4,1.5) {p+};
% drain
\fill[RedOrange] (6.5,1) rectangle (8,2);
\node at (7,1.5) {p+};
%% gate:
% gate oxide
\fill[LightGray] (4.8,2) rectangle (6.7,2.1);
% gate poly
\fill[BrickRed] (4.8,2.1) rectangle (6.7,2.2);

%field oxides:
\fill[DarkGray] (0,2) rectangle (1,4);
\fill[DarkGray] (8.5,2) rectangle (11.5,4);
\fill[DarkGray] (19,2) rectangle (20,4);

\fill[RedOrange] (0,1.5) rectangle (1,2);
\fill[RedOrange] (8.5,1.5) rectangle (11.5,2);
\fill[RedOrange] (19,1.5) rectangle (20,2);

\node at (0.5,1.75) {p+};
\node at (9.5,1.75) {p+};
\node at (19.5,1.75) {p+};

%%% nmos:
% body
\fill[RedOrange] (17,1) rectangle (18.5,2);
\node at (18,1.5) {p+};
% source
\fill[ProcessBlue] (15,1) rectangle (16.5,2);
\node at (16,1.5) {n+};
% drain
\fill[ProcessBlue] (12,1) rectangle (13.5,2);
\node at (13,1.5) {n+};

%% gate:
% gate oxide
\fill[LightGray] (13.3,2) rectangle (15.2,2.1);
% gate poly
\fill[BrickRed] (13.3,2.1) rectangle (15.2,2.2);
\fill[pimplant] (3.5,1) rectangle (5,2);
\node at (4,1.5) {p+};
\fill[pimplant] (6.5,1) rectangle (8,2);
\node at (7,1.5) {p+};
\fill[pimplant] (17,1) rectangle (18.5,2);
\node at (18,1.5) {p+};
	\end{tikzpicture}
	\begin{tikzpicture}[node distance = 3cm, auto, thick,scale=\CrossAndTopSectionBig, every node/.style={transform shape}]
		\fill[YellowOrange] (0,0) rectangle (20,10);

% n-well
\fill[Goldenrod] (1,1.25) rectangle (8.5,7.5);

% p+
\fill[RedOrange] (3.5,2) rectangle (5,6.5);
\fill[RedOrange] (6.5,2) rectangle (8,6.5);
\fill[RedOrange] (17,2) rectangle (18.5,6.5);

% n+
\fill[ProcessBlue] (1.5,2) rectangle (3,6.5);
\fill[ProcessBlue] (12,2) rectangle (13.5,6.5);
\fill[ProcessBlue] (15,2) rectangle (16.5,6.5);

% trench area
\fill[DarkGray] (0,0) rectangle (1,12);
\fill[DarkGray] (8.5,0) rectangle (11.5,12);
\fill[DarkGray] (19,0) rectangle (20,12);
\fill[DarkGray] (0,0) rectangle (20,1.25);
\fill[DarkGray] (0,7.5) rectangle (20,12);

% poly
\fill[BrickRed] (4.8,1.75) rectangle (6.7,9);
\fill[BrickRed] (13.3,1.75) rectangle (15.2,9);
\fill[BrickRed] (4.8,8) rectangle (15.2,9);
	\end{tikzpicture}
\end{center}

For the bulk of the NMOS transistors and for the source and drain of the PMOS transistors highly doped  p+ areas are required.
In this step we're going to build these.

\subsubsection{Dioxide layer}
\begin{center}
	\begin{tikzpicture}[node distance = 3cm, auto, thick,scale=\CrossSectionOnly, every node/.style={transform shape}]
		\input{tikz_process_steps/pimplant.1.a.tex}
	\end{tikzpicture} \\
	\includegraphics[scale=0.01]{down_arrow.png} \\
	\begin{tikzpicture}[node distance = 3cm, auto, thick,scale=\CrossSectionOnly, every node/.style={transform shape}]
		% substrate
\fill[YellowOrange] (0,0) rectangle (20,2);
\node at (2,0.5) {Si (p-type)};
% n-well
\fill[Goldenrod] (1.25,0.75) rectangle (8.25,2);
\node at (5.75,1) {N-Well};
% body
\fill[ProcessBlue] (1.5,1) rectangle (3,2);
\node at (2,1.5) {n+};
% source
\fill[RedOrange] (3.5,1) rectangle (5,2);
\node at (4,1.5) {p+};
% drain
\fill[RedOrange] (6.5,1) rectangle (8,2);
\node at (7,1.5) {p+};
%% gate:
% gate oxide
\fill[LightGray] (4.8,2) rectangle (6.7,2.1);
% gate poly
\fill[BrickRed] (4.8,2.1) rectangle (6.7,2.2);

%field oxides:
\fill[DarkGray] (0,2) rectangle (1,4);
\fill[DarkGray] (8.5,2) rectangle (11.5,4);
\fill[DarkGray] (19,2) rectangle (20,4);

\fill[RedOrange] (0,1.5) rectangle (1,2);
\fill[RedOrange] (8.5,1.5) rectangle (11.5,2);
\fill[RedOrange] (19,1.5) rectangle (20,2);

\node at (0.5,1.75) {p+};
\node at (9.5,1.75) {p+};
\node at (19.5,1.75) {p+};

%%% nmos:
% body
\fill[RedOrange] (17,1) rectangle (18.5,2);
\node at (18,1.5) {p+};
% source
\fill[ProcessBlue] (15,1) rectangle (16.5,2);
\node at (16,1.5) {n+};
% drain
\fill[ProcessBlue] (12,1) rectangle (13.5,2);
\node at (13,1.5) {n+};

%% gate:
% gate oxide
\fill[LightGray] (13.3,2) rectangle (15.2,2.1);
% gate poly
\fill[BrickRed] (13.3,2.1) rectangle (15.2,2.2);
% oxide
\fill[gray] (0,2) rectangle (20,3);
	\end{tikzpicture}
\end{center}

\subsubsection{Pattering}
\begin{center}
	\begin{tikzpicture}[node distance = 3cm, auto, thick,scale=\CrossAndTopSection, every node/.style={transform shape}]
		% substrate
\fill[YellowOrange] (0,0) rectangle (20,2);
\node at (2,0.5) {Si (p-type)};
% n-well
\fill[Goldenrod] (1.25,0.75) rectangle (8.25,2);
\node at (5.75,1) {N-Well};
% body
\fill[ProcessBlue] (1.5,1) rectangle (3,2);
\node at (2,1.5) {n+};
% source
\fill[RedOrange] (3.5,1) rectangle (5,2);
\node at (4,1.5) {p+};
% drain
\fill[RedOrange] (6.5,1) rectangle (8,2);
\node at (7,1.5) {p+};
%% gate:
% gate oxide
\fill[LightGray] (4.8,2) rectangle (6.7,2.1);
% gate poly
\fill[BrickRed] (4.8,2.1) rectangle (6.7,2.2);

%field oxides:
\fill[DarkGray] (0,2) rectangle (1,4);
\fill[DarkGray] (8.5,2) rectangle (11.5,4);
\fill[DarkGray] (19,2) rectangle (20,4);

\fill[RedOrange] (0,1.5) rectangle (1,2);
\fill[RedOrange] (8.5,1.5) rectangle (11.5,2);
\fill[RedOrange] (19,1.5) rectangle (20,2);

\node at (0.5,1.75) {p+};
\node at (9.5,1.75) {p+};
\node at (19.5,1.75) {p+};

%%% nmos:
% body
\fill[RedOrange] (17,1) rectangle (18.5,2);
\node at (18,1.5) {p+};
% source
\fill[ProcessBlue] (15,1) rectangle (16.5,2);
\node at (16,1.5) {n+};
% drain
\fill[ProcessBlue] (12,1) rectangle (13.5,2);
\node at (13,1.5) {n+};

%% gate:
% gate oxide
\fill[LightGray] (13.3,2) rectangle (15.2,2.1);
% gate poly
\fill[BrickRed] (13.3,2.1) rectangle (15.2,2.2);
% oxide
\fill[isolationoxide] (0,2) rectangle (20,3);
	\end{tikzpicture}
	\begin{tikzpicture}[node distance = 3cm, auto, thick,scale=\CrossAndTopSection, every node/.style={transform shape}]
		\fill[NormalGray] (0,0) rectangle (20,12);
	\end{tikzpicture} \\
	\includegraphics[scale=0.01]{down_arrow.png} \\
	\begin{tikzpicture}[node distance = 3cm, auto, thick,scale=\CrossAndTopSection, every node/.style={transform shape}]
		\input{tikz_process_steps/pimplant.2.b.tex}
	\end{tikzpicture}
	\begin{tikzpicture}[node distance = 3cm, auto, thick,scale=\CrossAndTopSection, every node/.style={transform shape}]
		\fill[orange] (0,0) rectangle (20,12);

% p+
\fill[gray] (3.5,2) rectangle (5,6.5);
\fill[gray] (6.5,2) rectangle (8,6.5);
\fill[gray] (17,2) rectangle (18.5,6.5);
	\end{tikzpicture}
\end{center}

\subsubsection{Etching}
\begin{center}
	\begin{tikzpicture}[node distance = 3cm, auto, thick,scale=\CrossAndTopSection, every node/.style={transform shape}]
		% oxide
\fill[isolationoxide] (0,2) rectangle (20,3.5);
% resist
\fill[resist] (0,3.5) rectangle (3.5,4.1);
\fill[resist] (8,3.5) rectangle (17,4.1);
\fill[resist] (18.5,3.5) rectangle (20,4.1);

% substrate
\fill[YellowOrange] (0,0) rectangle (20,2);
\node at (2,0.5) {Si (p-type)};
% n-well
\fill[Goldenrod] (1.25,0.75) rectangle (8.25,2);
\node at (5.75,1) {N-Well};
% body
\fill[ProcessBlue] (1.5,1) rectangle (3,2);
\node at (2,1.5) {n+};
% source
\fill[RedOrange] (3.5,1) rectangle (5,2);
\node at (4,1.5) {p+};
% drain
\fill[RedOrange] (6.5,1) rectangle (8,2);
\node at (7,1.5) {p+};
%% gate:
% gate oxide
\fill[LightGray] (4.8,2) rectangle (6.7,2.1);
% gate poly
\fill[BrickRed] (4.8,2.1) rectangle (6.7,2.2);

%field oxides:
\fill[DarkGray] (0,2) rectangle (1,4);
\fill[DarkGray] (8.5,2) rectangle (11.5,4);
\fill[DarkGray] (19,2) rectangle (20,4);

\fill[RedOrange] (0,1.5) rectangle (1,2);
\fill[RedOrange] (8.5,1.5) rectangle (11.5,2);
\fill[RedOrange] (19,1.5) rectangle (20,2);

\node at (0.5,1.75) {p+};
\node at (9.5,1.75) {p+};
\node at (19.5,1.75) {p+};

%%% nmos:
% body
\fill[RedOrange] (17,1) rectangle (18.5,2);
\node at (18,1.5) {p+};
% source
\fill[ProcessBlue] (15,1) rectangle (16.5,2);
\node at (16,1.5) {n+};
% drain
\fill[ProcessBlue] (12,1) rectangle (13.5,2);
\node at (13,1.5) {n+};

%% gate:
% gate oxide
\fill[LightGray] (13.3,2) rectangle (15.2,2.1);
% gate poly
\fill[BrickRed] (13.3,2.1) rectangle (15.2,2.2);
	\end{tikzpicture}
	\begin{tikzpicture}[node distance = 3cm, auto, thick,scale=\CrossAndTopSection, every node/.style={transform shape}]
		\fill[orange] (0,0) rectangle (20,12);

% n+
\fill[gray] (3.5,2) rectangle (8,6.5);
\fill[gray] (17,2) rectangle (18.5,6.5);
	\end{tikzpicture} \\
	\includegraphics[scale=0.01]{down_arrow.png} \\
	\begin{tikzpicture}[node distance = 3cm, auto, thick,scale=\CrossAndTopSection, every node/.style={transform shape}]
		% oxide
\fill[isolationoxide] (0,2) rectangle (3.0,3.5);
\fill[isolationoxide] (8.5,2) rectangle (17,3.5);
\fill[isolationoxide] (18.75,2) rectangle (20,3.5);

% resist
\fill[resist] (0,3.5) rectangle (3.0,4.1);
\fill[resist] (8.5,3.5) rectangle (17,4.1);
\fill[resist] (18.75,3.5) rectangle (20,4.1);

% substrate
\fill[YellowOrange] (0,0) rectangle (20,2);
\node at (2,0.5) {Si (p-type)};
% n-well
\fill[Goldenrod] (1.25,0.75) rectangle (8.25,2);
\node at (5.75,1) {N-Well};
% body
\fill[ProcessBlue] (1.5,1) rectangle (3,2);
\node at (2,1.5) {n+};
% source
\fill[RedOrange] (3.5,1) rectangle (5,2);
\node at (4,1.5) {p+};
% drain
\fill[RedOrange] (6.5,1) rectangle (8,2);
\node at (7,1.5) {p+};
%% gate:
% gate oxide
\fill[LightGray] (4.8,2) rectangle (6.7,2.1);
% gate poly
\fill[BrickRed] (4.8,2.1) rectangle (6.7,2.2);

%field oxides:
\fill[DarkGray] (0,2) rectangle (1,4);
\fill[DarkGray] (8.5,2) rectangle (11.5,4);
\fill[DarkGray] (19,2) rectangle (20,4);

\fill[RedOrange] (0,1.5) rectangle (1,2);
\fill[RedOrange] (8.5,1.5) rectangle (11.5,2);
\fill[RedOrange] (19,1.5) rectangle (20,2);

\node at (0.5,1.75) {p+};
\node at (9.5,1.75) {p+};
\node at (19.5,1.75) {p+};

%%% nmos:
% body
\fill[RedOrange] (17,1) rectangle (18.5,2);
\node at (18,1.5) {p+};
% source
\fill[ProcessBlue] (15,1) rectangle (16.5,2);
\node at (16,1.5) {n+};
% drain
\fill[ProcessBlue] (12,1) rectangle (13.5,2);
\node at (13,1.5) {n+};

%% gate:
% gate oxide
\fill[LightGray] (13.3,2) rectangle (15.2,2.1);
% gate poly
\fill[BrickRed] (13.3,2.1) rectangle (15.2,2.2);

	\end{tikzpicture}
	\begin{tikzpicture}[node distance = 3cm, auto, thick,scale=\CrossAndTopSection, every node/.style={transform shape}]
		\fill[orange] (0,0) rectangle (20,12);

% p+
\fill[Goldenrod] (3.5,2) rectangle (5,6.5);
\fill[Goldenrod] (6.5,2) rectangle (8,6.5);
\fill[YellowOrange] (17,2) rectangle (18.5,6.5);
	\end{tikzpicture}
\end{center}

\subsubsection{Cleaning}
\begin{center}
	\begin{tikzpicture}[node distance = 3cm, auto, thick,scale=\CrossSectionOnly, every node/.style={transform shape}]
		% substrate
\fill[YellowOrange] (0,0) rectangle (20,2);
\node at (2,0.5) {Si (p-type)};
% n-well
\fill[Goldenrod] (1.25,0.75) rectangle (8.25,2);
\node at (5.75,1) {N-Well};
% body
\fill[ProcessBlue] (1.5,1) rectangle (3,2);
\node at (2,1.5) {n+};
% source
\fill[RedOrange] (3.5,1) rectangle (5,2);
\node at (4,1.5) {p+};
% drain
\fill[RedOrange] (6.5,1) rectangle (8,2);
\node at (7,1.5) {p+};
%% gate:
% gate oxide
\fill[LightGray] (4.8,2) rectangle (6.7,2.1);
% gate poly
\fill[BrickRed] (4.8,2.1) rectangle (6.7,2.2);

%field oxides:
\fill[DarkGray] (0,2) rectangle (1,4);
\fill[DarkGray] (8.5,2) rectangle (11.5,4);
\fill[DarkGray] (19,2) rectangle (20,4);

\fill[RedOrange] (0,1.5) rectangle (1,2);
\fill[RedOrange] (8.5,1.5) rectangle (11.5,2);
\fill[RedOrange] (19,1.5) rectangle (20,2);

\node at (0.5,1.75) {p+};
\node at (9.5,1.75) {p+};
\node at (19.5,1.75) {p+};

%%% nmos:
% body
\fill[RedOrange] (17,1) rectangle (18.5,2);
\node at (18,1.5) {p+};
% source
\fill[ProcessBlue] (15,1) rectangle (16.5,2);
\node at (16,1.5) {n+};
% drain
\fill[ProcessBlue] (12,1) rectangle (13.5,2);
\node at (13,1.5) {n+};

%% gate:
% gate oxide
\fill[LightGray] (13.3,2) rectangle (15.2,2.1);
% gate poly
\fill[BrickRed] (13.3,2.1) rectangle (15.2,2.2);

% oxide
\fill[isolationoxide] (0,2) rectangle (3.5,3);
\fill[isolationoxide] (5,2) rectangle (6.5,3);
\fill[isolationoxide] (8,2) rectangle (17,3);
\fill[isolationoxide] (18.5,2) rectangle (20,3);;

% resist
\fill[resist] (0,3) rectangle (3.5,3.6);
\fill[resist] (5,3) rectangle (6.5,3.6);
\fill[resist] (8,3) rectangle (17,3.6);
\fill[resist] (18.5,3) rectangle (20,3.6);;
	\end{tikzpicture} \\
	\includegraphics[scale=0.01]{down_arrow.png} \\
	\begin{tikzpicture}[node distance = 3cm, auto, thick,scale=\CrossSectionOnly, every node/.style={transform shape}]
		% substrate
\fill[YellowOrange] (0,0) rectangle (20,2);
\node at (2,0.5) {Si (p-type)};
% n-well
\fill[Goldenrod] (1.25,0.75) rectangle (8.25,2);
\node at (5.75,1) {N-Well};
% body
\fill[ProcessBlue] (1.5,1) rectangle (3,2);
\node at (2,1.5) {n+};
% source
\fill[RedOrange] (3.5,1) rectangle (5,2);
\node at (4,1.5) {p+};
% drain
\fill[RedOrange] (6.5,1) rectangle (8,2);
\node at (7,1.5) {p+};
%% gate:
% gate oxide
\fill[LightGray] (4.8,2) rectangle (6.7,2.1);
% gate poly
\fill[BrickRed] (4.8,2.1) rectangle (6.7,2.2);

%field oxides:
\fill[DarkGray] (0,2) rectangle (1,4);
\fill[DarkGray] (8.5,2) rectangle (11.5,4);
\fill[DarkGray] (19,2) rectangle (20,4);

\fill[RedOrange] (0,1.5) rectangle (1,2);
\fill[RedOrange] (8.5,1.5) rectangle (11.5,2);
\fill[RedOrange] (19,1.5) rectangle (20,2);

\node at (0.5,1.75) {p+};
\node at (9.5,1.75) {p+};
\node at (19.5,1.75) {p+};

%%% nmos:
% body
\fill[RedOrange] (17,1) rectangle (18.5,2);
\node at (18,1.5) {p+};
% source
\fill[ProcessBlue] (15,1) rectangle (16.5,2);
\node at (16,1.5) {n+};
% drain
\fill[ProcessBlue] (12,1) rectangle (13.5,2);
\node at (13,1.5) {n+};

%% gate:
% gate oxide
\fill[LightGray] (13.3,2) rectangle (15.2,2.1);
% gate poly
\fill[BrickRed] (13.3,2.1) rectangle (15.2,2.2);

% oxide
\fill[gray] (0,2) rectangle (3.5,3);
\fill[gray] (8,2) rectangle (13.3,3);
\fill[gray] (13.3,2.6) rectangle (15.2,3);
\fill[gray] (15.2,2) rectangle (17,3);
\fill[gray] (18.5,2) rectangle (20,3);
	\end{tikzpicture}
\end{center}

\subsubsection{Predeposition}
\begin{center}
	\begin{tikzpicture}[node distance = 3cm, auto, thick,scale=\CrossSectionOnly, every node/.style={transform shape}]
		% oxide
\fill[isolationoxide] (0,2) rectangle (3.5,3.5);
\fill[isolationoxide] (8,2) rectangle (17,3.5);
\fill[isolationoxide] (18.5,2) rectangle (20,3.5);

\forloop{ct}{0}{\value{ct} < 21}
{
	\draw [->] (\value{ct},5) -- (\value{ct},4);
	\node at (\value{ct},5.2) {B$^{11}$};
}

% substrate
\fill[YellowOrange] (0,0) rectangle (20,2);
\node at (2,0.5) {Si (p-type)};
% n-well
\fill[Goldenrod] (1.25,0.75) rectangle (8.25,2);
\node at (5.75,1) {N-Well};
% body
\fill[ProcessBlue] (1.5,1) rectangle (3,2);
\node at (2,1.5) {n+};
% source
\fill[RedOrange] (3.5,1) rectangle (5,2);
\node at (4,1.5) {p+};
% drain
\fill[RedOrange] (6.5,1) rectangle (8,2);
\node at (7,1.5) {p+};
%% gate:
% gate oxide
\fill[LightGray] (4.8,2) rectangle (6.7,2.1);
% gate poly
\fill[BrickRed] (4.8,2.1) rectangle (6.7,2.2);

%field oxides:
\fill[DarkGray] (0,2) rectangle (1,4);
\fill[DarkGray] (8.5,2) rectangle (11.5,4);
\fill[DarkGray] (19,2) rectangle (20,4);

\fill[RedOrange] (0,1.5) rectangle (1,2);
\fill[RedOrange] (8.5,1.5) rectangle (11.5,2);
\fill[RedOrange] (19,1.5) rectangle (20,2);

\node at (0.5,1.75) {p+};
\node at (9.5,1.75) {p+};
\node at (19.5,1.75) {p+};

%%% nmos:
% body
\fill[RedOrange] (17,1) rectangle (18.5,2);
\node at (18,1.5) {p+};
% source
\fill[ProcessBlue] (15,1) rectangle (16.5,2);
\node at (16,1.5) {n+};
% drain
\fill[ProcessBlue] (12,1) rectangle (13.5,2);
\node at (13,1.5) {n+};

%% gate:
% gate oxide
\fill[LightGray] (13.3,2) rectangle (15.2,2.1);
% gate poly
\fill[BrickRed] (13.3,2.1) rectangle (15.2,2.2);

	\end{tikzpicture} \\
	\includegraphics[scale=0.01]{down_arrow.png} \\
	\begin{tikzpicture}[node distance = 3cm, auto, thick,scale=\CrossSectionOnly, every node/.style={transform shape}]
		% substrate
\fill[YellowOrange] (0,0) rectangle (20,2);
\node at (2,0.5) {Si (p-type)};
% n-well
\fill[Goldenrod] (1.25,0.75) rectangle (8.25,2);
\node at (5.75,1) {N-Well};
% body
\fill[ProcessBlue] (1.5,1) rectangle (3,2);
\node at (2,1.5) {n+};
% source
\fill[RedOrange] (3.5,1) rectangle (5,2);
\node at (4,1.5) {p+};
% drain
\fill[RedOrange] (6.5,1) rectangle (8,2);
\node at (7,1.5) {p+};
%% gate:
% gate oxide
\fill[LightGray] (4.8,2) rectangle (6.7,2.1);
% gate poly
\fill[BrickRed] (4.8,2.1) rectangle (6.7,2.2);

%field oxides:
\fill[DarkGray] (0,2) rectangle (1,4);
\fill[DarkGray] (8.5,2) rectangle (11.5,4);
\fill[DarkGray] (19,2) rectangle (20,4);

\fill[RedOrange] (0,1.5) rectangle (1,2);
\fill[RedOrange] (8.5,1.5) rectangle (11.5,2);
\fill[RedOrange] (19,1.5) rectangle (20,2);

\node at (0.5,1.75) {p+};
\node at (9.5,1.75) {p+};
\node at (19.5,1.75) {p+};

%%% nmos:
% body
\fill[RedOrange] (17,1) rectangle (18.5,2);
\node at (18,1.5) {p+};
% source
\fill[ProcessBlue] (15,1) rectangle (16.5,2);
\node at (16,1.5) {n+};
% drain
\fill[ProcessBlue] (12,1) rectangle (13.5,2);
\node at (13,1.5) {n+};

%% gate:
% gate oxide
\fill[LightGray] (13.3,2) rectangle (15.2,2.1);
% gate poly
\fill[BrickRed] (13.3,2.1) rectangle (15.2,2.2);
% oxide
\fill[isolationoxide] (0,2) rectangle (3.5,3);
\fill[isolationoxide] (5,2) rectangle (6.5,3);
\fill[isolationoxide] (8,2) rectangle (17,3);
\fill[isolationoxide] (18.5,2) rectangle (20,3);

\fill[pimplant] (3.5,1) rectangle (5,2);
\node at (4,1.5) {p+};
\fill[pimplant] (6.5,1) rectangle (8,2);
\node at (7,1.5) {p+};
\fill[pimplant] (17,1) rectangle (18.5,2);
\node at (18,1.5) {p+};
	\end{tikzpicture}
\end{center}

The p+ islands are implanted with a Boron ($B^{11}$) dose of $4\times10^{11}cm^{-2}$ at an energy of 35 KeV.

\subsubsection{Sacrificial oxide}
\begin{center}
	\begin{tikzpicture}[node distance = 3cm, auto, thick,scale=\CrossSectionOnly, every node/.style={transform shape}]
		% oxide
\fill[isolationoxide] (0,2) rectangle (3.0,3.5);
\fill[isolationoxide] (8.5,2) rectangle (17,3.5);
\fill[isolationoxide] (18.75,2) rectangle (20,3.5);

% substrate
\fill[YellowOrange] (0,0) rectangle (20,2);
\node at (2,0.5) {Si (p-type)};
% n-well
\fill[Goldenrod] (1.25,0.75) rectangle (8.25,2);
\node at (5.75,1) {N-Well};
% body
\fill[ProcessBlue] (1.5,1) rectangle (3,2);
\node at (2,1.5) {n+};
% source
\fill[RedOrange] (3.5,1) rectangle (5,2);
\node at (4,1.5) {p+};
% drain
\fill[RedOrange] (6.5,1) rectangle (8,2);
\node at (7,1.5) {p+};
%% gate:
% gate oxide
\fill[LightGray] (4.8,2) rectangle (6.7,2.1);
% gate poly
\fill[BrickRed] (4.8,2.1) rectangle (6.7,2.2);

%field oxides:
\fill[DarkGray] (0,2) rectangle (1,4);
\fill[DarkGray] (8.5,2) rectangle (11.5,4);
\fill[DarkGray] (19,2) rectangle (20,4);

\fill[RedOrange] (0,1.5) rectangle (1,2);
\fill[RedOrange] (8.5,1.5) rectangle (11.5,2);
\fill[RedOrange] (19,1.5) rectangle (20,2);

\node at (0.5,1.75) {p+};
\node at (9.5,1.75) {p+};
\node at (19.5,1.75) {p+};

%%% nmos:
% body
\fill[RedOrange] (17,1) rectangle (18.5,2);
\node at (18,1.5) {p+};
% source
\fill[ProcessBlue] (15,1) rectangle (16.5,2);
\node at (16,1.5) {n+};
% drain
\fill[ProcessBlue] (12,1) rectangle (13.5,2);
\node at (13,1.5) {n+};

%% gate:
% gate oxide
\fill[LightGray] (13.3,2) rectangle (15.2,2.1);
% gate poly
\fill[BrickRed] (13.3,2.1) rectangle (15.2,2.2);

\fill[pimplant] (3.0,1.5) rectangle (5,2);
\fill[pimplant] (6.5,1.5) rectangle (8.5,2);
\fill[pimplant] (17,1.5) rectangle (18.75,2);
	\end{tikzpicture} \\
	\includegraphics[scale=0.01]{down_arrow.png} \\
	\begin{tikzpicture}[node distance = 3cm, auto, thick,scale=\CrossSectionOnly, every node/.style={transform shape}]
		% substrate
\fill[YellowOrange] (0,0) rectangle (20,2);
\node at (2,0.5) {Si (p-type)};
% n-well
\fill[Goldenrod] (1.25,0.75) rectangle (8.25,2);
\node at (5.75,1) {N-Well};
% body
\fill[ProcessBlue] (1.5,1) rectangle (3,2);
\node at (2,1.5) {n+};
% source
\fill[RedOrange] (3.5,1) rectangle (5,2);
\node at (4,1.5) {p+};
% drain
\fill[RedOrange] (6.5,1) rectangle (8,2);
\node at (7,1.5) {p+};
%% gate:
% gate oxide
\fill[LightGray] (4.8,2) rectangle (6.7,2.1);
% gate poly
\fill[BrickRed] (4.8,2.1) rectangle (6.7,2.2);

%field oxides:
\fill[DarkGray] (0,2) rectangle (1,4);
\fill[DarkGray] (8.5,2) rectangle (11.5,4);
\fill[DarkGray] (19,2) rectangle (20,4);

\fill[RedOrange] (0,1.5) rectangle (1,2);
\fill[RedOrange] (8.5,1.5) rectangle (11.5,2);
\fill[RedOrange] (19,1.5) rectangle (20,2);

\node at (0.5,1.75) {p+};
\node at (9.5,1.75) {p+};
\node at (19.5,1.75) {p+};

%%% nmos:
% body
\fill[RedOrange] (17,1) rectangle (18.5,2);
\node at (18,1.5) {p+};
% source
\fill[ProcessBlue] (15,1) rectangle (16.5,2);
\node at (16,1.5) {n+};
% drain
\fill[ProcessBlue] (12,1) rectangle (13.5,2);
\node at (13,1.5) {n+};

%% gate:
% gate oxide
\fill[LightGray] (13.3,2) rectangle (15.2,2.1);
% gate poly
\fill[BrickRed] (13.3,2.1) rectangle (15.2,2.2);

\fill[RedOrange] (3.5,1) rectangle (5,2);
\node at (4,1.5) {p+};
\fill[RedOrange] (6.5,1) rectangle (8,2);
\node at (7,1.5) {p+};
\fill[RedOrange] (17,1) rectangle (18.5,2);
\node at (18,1.5) {p+};
	\end{tikzpicture}
\end{center}

\subsubsection{Drive-in}
\begin{center}
	\begin{tikzpicture}[node distance = 3cm, auto, thick,scale=\CrossSectionOnly, every node/.style={transform shape}]
		\input{tikz_process_steps/pimplant.7.a.tex}
	\end{tikzpicture} \\
	\includegraphics[scale=0.01]{down_arrow.png} \\
	\begin{tikzpicture}[node distance = 3cm, auto, thick,scale=\CrossSectionOnly, every node/.style={transform shape}]
		\input{tikz_process_steps/pimplant.7.b.tex}
	\end{tikzpicture}
\end{center}

\subsubsection{Oxide removal}
In the final step before the  next mask we need to remove the impure oxide layer for further processing of the wafer.

\begin{center}
	\begin{tikzpicture}[node distance = 3cm, auto, thick,scale=\CrossSectionOnly, every node/.style={transform shape}]
		\input{tikz_process_steps/pimplant.8.a.tex}
	\end{tikzpicture} \\
	\includegraphics[scale=0.01]{down_arrow.png} \\
	\begin{tikzpicture}[node distance = 3cm, auto, thick,scale=\CrossSectionOnly, every node/.style={transform shape}]
		\input{tikz_process_steps/pimplant.8.b.tex}
	\end{tikzpicture}
\end{center}

For this purpose we use a wet etching method selectively etching away the before grown, each approximately 250nm thick layers of oxide, together 500nm.
Because we don't want to reduce the thick oxide of the field oxide area too much we limit the etching time  here, so that we at worst etch maybe a 10% of it.
Since we have grown it to be around 750nm we can afford to loose around 75nm or so.

